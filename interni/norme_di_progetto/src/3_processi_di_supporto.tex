\section{Processi di supporto}
\label{sec:processi_di_supporto}

\subsection{Documentazione}
\label{sub:doc}

\subsubsection{Descrizione}
\label{ssub:documentazione:descrizione}

Il processo di documentazione aggrega le diverse attività che il gruppo \emph{SpaghettiCode} dovrà svolgere al fine di fornire la documentazione necessaria a corredo del software \emph{HD Viz} del proponente \emph{Zucchetti S.p.A.} e dei committenti \emph{prof. Tullio Vardanega} e \emph{prof. Riccardo Cardin}.

Il processo di documentazione adottato dal gruppo SpaghettiCode consiste nelle seguenti attività definite dallo standard
di processo ISO/IEC 12207:1997, opportunamente adattate alle necessità del progetto didattico:
\begin{itemize}
    \item \textbf{Creazione};
    \item \textbf{Stesura};
    \item \textbf{Revisione};
    \item \textbf{Approvazione}.
\end{itemize}
Essendo il modello di sviluppo adottato dal gruppo di tipo incrementale, ci saranno continue modifiche dei documenti, ad eccezione dei verbali.
Dopo l'approvazione infatti, ogni documento potrà essere esteso, dando quindi inizio ad un nuovo incremento che corrisponderà ad una nuova versione di quel documento.

\subsubsection{Attività}
\label{ssub:documentazione:attivita}

\paragraph{Creazione}
\label{par:creazione}
\subparagraph{Scopo}
\label{par:creazione:scopo}
Identificati i destinatari, il titolo e lo scopo del documento, questo viene creato a partire da un template, fornito del registro delle modifiche e degli elenchi di figure e tabelle.
\subparagraph{Compiti}
\label{par:creazione:compiti}
Questa attività è composta dal seguente compito:
\begin{itemize}
    \item \textbf{Creazione}: si identificano le caratteristiche del documento e lo si crea a partire da un template.
\end{itemize}



\paragraph{Stesura}
\label{par:stesura}
\subparagraph{Scopo}
\label{par:stesura:scopo}
Il documento viene scritto da più membri del gruppo, riportando i vari incrementi nel registro delle modifiche.
\subparagraph{Compiti}
\label{par:stesura:compiti}
Questa attività è composta dal seguente compito:
\begin{itemize}
    \item \textbf{Stesura}: si procede nella scrittura del documento, tracciando i cambiamenti attraverso il registro delle modifiche.
\end{itemize}

\paragraph{Revisione}
\label{par:revisione}
\subparagraph{Scopo}
\label{par:revizione:scopo}
In seguito a ciascun incremento del documento vengono revisionate le modifiche da uno o più \emph{verificatori}, i quali valutano se il documento rispetta le norme stabilite; è obbligatorio che un redattore non verifichi un incremento da lui svolto
\subparagraph{Compiti}
\label{par:revisione:compiti}
Questa attività è composta dal seguente compito:
\begin{itemize}
    \item \textbf{Revisione}: si procede con la verifica la conformità del documento prodotto rispetto agli standard di qualità imposti.
\end{itemize}

\paragraph{Approvazione}
\label{par:approvazione}
\subparagraph{Scopo}
\label{par:approvazione:scopo}
A seguito di ciascuna revisione il \emph{Responsabile} può potenzialmente approvare il documento; una versione del documento approvata può essere rilasciata.
\subparagraph{Compiti}
\label{par:approvazione:compiti}
Questa attività è composta dal seguente compito:
\begin{itemize}
    \item \textbf{Approvazione}: il \emph{Responsabile} può potenzialmente approvare il documento.
\end{itemize}


% TODO: duplicato di 3.1.8?
\subsubsection{Template}
\label{ssub:template}

Il gruppo ha deciso di utilizzare per la stesura dei documenti del progetto il linguaggio \LaTeX. Al fine di
uniformare i documenti e velocizzare il processo di creazione e strutturazione degli stessi è stato definito un
template su cui ogni documento si basa.

\subsubsection{Documenti prodotti}
\label{ssub:documenti_prodotti}

I documenti prodotti vengono suddivisi in due macrocategorie:
\begin{itemize}
    \item \textbf{Informali}: in questa macrocategoria rientrano i documenti non soggetti a versionamento; in
    particolare fanno parte di questa categoria i verbali, i quali fungono da resoconto degli incontri. I verbali
    vengono a loro volta distinti in verbali interni ed esterni:
    \begin{itemize}
        \item \textbf{Verbali Interni}: a questa categoria appartengono i verbali degli incontri interni del gruppo;
        \item \textbf{Verbali Esterni}: a questa categoria appartengono i verbali degli incontri con membri esterni al
        gruppo, quali potrebbero essere i committenti e/o i proponenti.
    \end{itemize}

    \item \textbf{Formali}: in questa macrocategoria rientrano tutti i documenti soggetti a versionamento; questi devono
    essere approvati ad ogni loro versione dal \emph{Responsabile} di progetto e viene considerata come corrente la
    versione più recentemente approvata. Un'ulteriore distinzione viene fatta tra:
    \begin{itemize}
        \item \textbf{Documenti Interni}: appartengono a questa categoria i documenti destinati ad uso interno al
        gruppo e non sono quindi di diretto interesse per il proponente e i committenti;
        \item \textbf{Documenti Esterni}: appartengono a questa categoria i documenti destinati ad essere consegnati,
        nella loro ultima versione approvata, al proponente e ai committenti.
    \end{itemize}
    Di seguito viene riportato un elenco dei documenti formali prodotti:
    \begin{itemize}
        \item \textbf{Norme di progetto}: documento interno che racchiude tutte le norme che regolamentano il progetto
        nella sua interezza;
        \item \textbf{Glossario}: documento esterno che elenca tutti i termini che, secondo il gruppo, necessitano di
        una spiegazione esplicita;
        \item \textbf{Studio di fattibilità}: documento interno che raccoglie i singoli studi di fattibilità sui
        capitolati, secondo lo schema descritto in questo documento al paragrafo \refSec{par:studio_fattibilita};
        \item \textbf{Piano di progetto}: documento esterno che espone la pianificazione delle attività di progetto
        previste dal gruppo e il preventivo degli impegni orari e dei costi;
        \item \textbf{Piano di qualifica}: documento esterno che espone i criteri di valutazione della qualità del
        progetto;
        \item \textbf{Analisi dei requisiti}: documento esterno che raccoglie tutti i requisiti del progetto e che il
        prodotto software deve avere.
  \end{itemize}

\end{itemize}

\subsubsection{Cartella dei documenti}
\label{ssub:cartella_doc}

Seguendo le convenzioni in \refSec{par:attribuzione_nome} per ogni documento viene creata una cartella, all'interno della quale viene posto
un file .tex rappresentante il documento stesso, assieme ad un'eventuale cartella \emph{src} contenente le diverse risorse necessarie.
La cartella \emph{src} può contenere altre sottodirectory per meglio organizzare le dipendenze, suddividendole prima per tipologia di file
(e.g. img) e poi per tipologia di contenuto (e.g. diagrammi\_uc).

\subsubsection{Struttura dei documenti}
\label{ssub:struttura_doc}

Tutti i documenti prodotti si rifanno ad un template predefinito scritto in formato .tex che ne definisce la struttura
e le parti comuni. Ogni documento è caratterizzato nel frontespizio da una serie di elementi, dall'alto verso il basso:
\begin{itemize}
    \item Il logo a colori del gruppo;
    \item Il nome del gruppo;
    \item Il contatto e-mail del gruppo;
    \item Il nome del documento;
    \item Una tabella riportante la versione, l'approvatore, i redattori, i verificatori, l'uso e i destinatari;
    \item Una breve descrizione del contenuto del documento.
\end{itemize}
A seguire ci sono una o più pagine che riportano il registro delle modifiche, in cui vengono specificati:
\begin{itemize}
    \item Versione del documento;
    \item Cognome e nome di chi ha apportato la modifica;
    \item Ruolo all'interno del gruppo;
    \item Data della modifica;
    \item Descrizione della modifica.
\end{itemize}
Nella descrizione della modifica potrebbero comparire i caratteri \textbf{??} qualora fosse stata rimossa una sezione dal documento
precedentemente riferita.
A seguire si trova la tabella dei contenuti e, se previste dal documento, una lista delle tabelle e una delle figure presenti.\\
Segue poi il corpo del documento, sezione caratteristica di ciascun documento considerato. Nonostante ciò la struttura
di ogni singola pagina rimane fissa ed è così strutturata:
\begin{itemize}
    \item In alto a sinistra ci sono il nome del gruppo e il nome del documento;
    \item In alto a destra c'è il logo del gruppo nella sua versione a colori;
    \item Al di sotto di questi due elementi c'è una linea nera che li separa dal contenuto della pagina;
    \item Il contenuto della pagina;
    \item A piè di pagina si trova, sulla destra, il numero della pagina corrente e il numero di pagine totali.
\end{itemize}
I verbali invece si presentano nella seguente struttura:
\begin{itemize}
    \item Frontespizio;
    \item Registro delle modifiche;
    \item Indice;
    \item Informazioni generali riguardanti luogo, data ed orario dell'incontro e partecipanti;
    \item Ordine del giorno;
    \item Resoconto;
    \item Tabella riassuntiva delle decisioni prese e relativo codice identificativo.
\end{itemize}

\subsubsection{Norme tipografiche}
\label{ssub:norme_tipografiche}

\paragraph{Convenzioni sui nomi di file e cartelle}
\label{par:attribuzione_nome}

Per i nomi dei file e delle cartelle viene seguita la convenzione \glossario{snake\_case}; essa prevede la scrittura di tutti i nomi in stampatello
minuscolo e la separazione delle parole mediante il carattere \glossario{underscore}.

\paragraph{Stili di testo}
\label{par:stili_testo}

Di seguito sono indicati i differenti stili del testo adottati nei documenti, con il relativo significato:
\begin{itemize}
    \item \textbf{Grassetto}: utilizzato per singole parole da enfatizzare;
    \item \textbf{Corsivo}: utilizzato per i nomi propri o per evidenziare frasi complete;
    \item \textbf{Monospace}: utilizzato per riportare frammenti di codice;
    \item \textbf{Stampatello maiuscolo}: utilizzato per le sigle o per riferimenti a documenti.
\end{itemize}

\paragraph{Elenchi puntati}
\label{par:elenchi}

Per indicare gli elementi di un elenco puntato viene utilizzato il \glossario{punto elenco}, mentre nel caso in cui l'elenco
sia annidato viene utilizzato il \glossario{trattino}.

\paragraph{Formati di date}
\label{par:date}

Le date vengono indicate secondo lo standard \textsc{ISO 8601}, con il formato:
\begin{center}
    \textbf{[YYYY]-[MM]-[DD]}
\end{center}
In questa notazione YYYY indica l'anno, MM indica il mese e DD indica il giorno.

\paragraph{Sigle}
\label{par:sigle}

Vengono riportate di seguito le diverse sigle utilizzate.
Per quanto riguarda i documenti vengono utilizzate le sigle:
\begin{itemize}
    \item AdR: Analisi dei Requisiti;
    \item PdP: Piano di Progetto;
    \item PdQ: Piano di Qualifica;
    \item SdF: Studio di Fattibilità;
    \item NdP: Norme di Progetto;
    \item VI: Verbali Interni;
    \item VE: Verbali Esterni.
\end{itemize}

Per quanto riguarda le revisioni di progetto le sigle utilizzate sono:
\begin{itemize}
    \item RR: Revisione dei Requisiti;
    \item RP: Revisione di Progettazione;
    \item RQ: Revisione di Qualifica;
    \item RA: Revisione di Accettazione.
\end{itemize}

\subsubsection{Elementi grafici}
\label{ssub:elementi_grafici}

\paragraph{Immagini}
\label{par:immagini}

Le immagini devono essere centrate nella pagina e devono avere un'opportuna didascalia.

\paragraph{Grafici UML}
\label{par:uml}

I grafici UML vengono inseriti come immagini.

\paragraph{Tabelle}
\label{par:tabelle}

Ogni tabella, ad eccezione del registro delle modifiche, è accompagnata da una didascalia e dev'essere centrata nella
pagina.

\subsubsection{Strumenti}
\label{ssub:strumenti}

Di seguito vengono indicati gli strumenti utilizzati per il corretto svolgimento del processo:
\begin{itemize}
	\item \LaTeX: linguaggio di markup per la preparazione di testi altamente personalizzabile;
	\item \textbf{Draw.io}: applicazione web molto versatile per disegnare grafici UML che offre la possibilità di lavorare
		contemporaneamente;
	\item \textbf{Gantt project}: software per la gestione di progetto che permette di creare diagrammi di Gantt in maniera rapida ed \
		intuitiva.
\end{itemize}

\subsection{Gestione delle configurazioni}
\label{sub:gestione_configurazioni}

\subsubsection{Descrizione}
\label{ssub:gestione_configurazione:descrizione}

Questo processo ha lo scopo di gestire in maniera ordinata e sistematica la produzione di elementi sottoposti a controllo di versione,
quali documenti e sistemi software.
\\
\\
\textbf{NOTE}: all'interno di \refSec{sub:gestione_configurazioni} con \emph{oggetto} si intenderà un qualsiasi elemento sottoposto a
controllo di versione.
\label{par:gestione_configurazione:compiti}
Il processo di gestione delle configurazioni adottato dal gruppo \emph{SpaghettiCode} consiste nelle seguenti attività definite dallo standard di processo \emph{ISO/IEC 12207:1997}, opportunamente adattate alle necessità del progetto didattico:
\begin{itemize}
    \item \textbf{Pianificazione};
    \item \textbf{Controllo della configurazione};
    \item \textbf{Valutazione della configurazione}.
\end{itemize}

\subsubsection{Attività}

\paragraph{Pianificazione}
\label{par:pianificazione}
\subparagraph{Scopo}
\label{par:pianificazione:scopo}
In questa attività si stabiliscono quali e come procedure da seguire e si decide quale software di versionamento usare.
\subparagraph{Compiti}
\label{par:pianificazione:compiti}
Questa attività è composta dal seguente compito:
\begin{itemize}
    \item \textbf{Pianificazione}: si stabilisce la pianificazione per la gestione di configurazione.
\end{itemize}

\paragraph{Controllo della configurazione}
\label{par:controllo_della_configurazione}
\subparagraph{Scopo}
\label{par:controllo_della_configurazione:scopo}
In questa attività si identificano e si registrano le richieste di cambiamento, si analizzano, e conseguentemente, si accettano o rifiutano i cambiamenti.
\subparagraph{Compiti}
\label{par:controllo_della_configurazione:compiti}
Questa attività è composta dal seguente compito:
\begin{itemize}
    \item \textbf{Controllo della configurazione}: Si controlla il cambiamento del prodotto. Si effettua verifica, implementazione e rilascio dei cambiamenti. Si storicizzano i cambiamenti e le richieste di cambiamento.
\end{itemize}

\paragraph{Valutazione della configurazione}
\label{par:valutazione_della_configurazione}
\subparagraph{Scopo}
\label{par:valutazione_della_configurazione:scopo}
In questa attività si valuta la completezza degli oggetti rispetto ai requisiti ed alla completezza fisica.
\subparagraph{Compiti}
\label{par:valutazione_della_configurazione:compiti}
Questa attività è composta dal seguente compito:
\begin{itemize}
    \item \textbf{Valutazione}: si verifica se gli oggetti sono aggiornati rispetto alla loro codifica e design.
\end{itemize}


\subsubsection{Versionamento}
\label{ssub:gestione_configurazione:versionamento}

\paragraph{Scopo}

L'attività di versionamento ha come obiettivo fornire una gestione e uno storico delle diverse versioni degli oggetti. Di seguito vengono
definite le norme e le convenzioni a cui il gruppo aderisce per poter facilitare ed automatizzare le modifiche agli oggetti.
Il sistema di versionamento adottato è il \glossario{versionamento semantico}, ed aderisce alle convenzioni stabilite, reperibili all'indirizzo: \url{https://semver.org/lang/it/}.

\paragraph{Codice di versione}
\label{par:codice_versione}

I documenti ed il prodotto hanno sempre associato un codice di versione, che si presenta nella forma:
\begin{center}
    \textbf{[X].[Y].[Z]}
\end{center}
dove \textbf{X}, \textbf{Y} e \textbf{Z} sono interi non negativi.\\
Ciascun codice di versione viene assegnato ad una correzione o ad un incremento pianificato e a seconda dell'entit\`{a} della modifica
pu\`{o} essere aumentata:
\begin{itemize}
	\item \textbf{X}: incrementa quando viene effettuata una \emph{modifica non retrocompatibile};
	\item \textbf{Y}: incrementa quando si effettuano \emph{aggiunte  retrocompatibili}; il valore Z viene azzerato;
	\item \textbf{Z}: incrementa quando vengono effettuate \emph{modifiche retrocompatibili}.
\end{itemize}
Ciascun incremento di \textbf{X}, \textbf{Y} e \textbf{Z} deve essere unitario e ciascuna modifica ad un documento o ad un prodotto
dev'essere rilasciata come una nuova versione.
Lo scatto di versione verrà effettuato solo in seguito alla verifica da parte di un verificatore della effettiva bontà della aggiunta/modifica effettuata.

\paragraph{Codice di versione unitario}
Il prodotto software avrà a sua volta un suo codice identificativo che ne indichi lo stato di versione. Nello specifico il codice verrà espresso come segue:\\
\begin{center}
    \textbf{[A].[B]}
\end{center}

dove:
\begin{itemize}
\item \textbf{B} scatterà di versione ogni volta che verrà implementato un incremento pianificato nel \textsc{Piano di Progetto}; questo valore numerico partirà da 0 e crescerà. Non verrà mai azzerato.
\item \textbf{A} scatterà ogni volta che il software verrà testato e ritenuto un prodotto stabile e rilasciabile. Anche questo valore partirà da 0 e non verrà mai azzerato.
\end{itemize}

\paragraph{Repository}
\label{par:repo}

Il gruppo ha creato per il progetto un'organizzazione su \emph{GitHub} e ha provveduto a creare tre repository:
\begin{itemize}
    \item \textbf{swe-docs}: contenente tutti i documenti prodotti dal gruppo ed individuabile al link
    	\url{https://github.com/Spaghetti-Code-G6/swe-docs};
    \item \textbf{HD-Viz}: contenente il codice del prodotto \emph{HD Viz}, individuabile al link \url{https://github.com/Spaghetti-Code-G6/HD-Viz};
    \item \textbf{HD-Viz-PoC}: contenente il \emph{Proof of Concept} del prodotto, sviluppato dal gruppo per la \emph{Technology Baseline}.
\end{itemize}

\paragraph{Struttura dei repository}
\label{par:struttura_repo}

Il repository \emph{swe-docs} è strutturato come segue:
\begin{itemize}
    \item \textbf{esterni}: directory contenente tutti i documenti destinati ad uso esterno, ovvero
		indirizzati al proponente e ai committenti. All'interno della cartella si trova una directory per ciascun documento, la quale
		segue le regole di struttura e nomenclatura descritte nel paragrafo \refSec{ssub:cartella_doc};
    \item \textbf{interni}: questa directory contiene tutti i documenti destinati ad uso interno ed ha una struttura analoga a quella dei
    	documenti esterni;
    \item \textbf{config}: in questa directory sono presenti i file \emph{config.tex} e \emph{template.tex}
    	che definiscono il template dei documenti, inoltre è presente una sottodirectory \emph{src} contenente tutte le risorse necessarie
    	al template (immagini, loghi, ecc.);
    \item \textbf{utils}: directory contenente script di utilità come compilazione dei sorgenti e rimozione dei file intermedi di
    	compilazione.
    \item \textbf{.gitignore}
    \item \textbf{README.md}
\end{itemize}

All'interno di \emph{swe-docs} sono presenti i seguenti tipi di file:
\begin{itemize}
    \item File \textbf{.tex}: file sorgenti \LaTeX\ con i contenuti dei documenti;
    \item Immagini da inserire nei documenti;
    \item File \textbf{.md}: file utilizzati a scopo di documentazione all'interno del repository;
    \item \textbf{.gitignore}: file contenente un riferimento a tutte le tipologie di file non versionati;
    \item \textbf{.sh}: script da linea di comando di utilità.
\end{itemize}


Il repository \emph{HD-Viz-PoC}, essendo un sistema prototipale in continuo sviluppo, non possiede una struttura rigida.

Il repository \emph{HD-Viz} al momento è inutilizzato e pertanto la sua struttura verrà normata successivamente.
% TODO: normare struttura del repo
% private/
%
% public/
%	html/
% 	javascript/
%	css/

\paragraph{Branch}
\label{par:gestione_configurazione:branching}

Al fine di poter lavorare parallelamente sugli stessi oggetti si è deciso di adottare il \glossario{workflow} \emph{feature branch}, in
particolare sono stati previsti i seguenti branch:
\begin{itemize}
	\item \textbf{main}: contenente gli oggetti all'ultima versione rilasciata;
	\item \textbf{develop}: contenente gli oggetti rilasciati, oppure in fase di rilascio;
	\item \textbf{config}: utilizzato esclusivamente per sviluppare strumenti di supporto e gestione della configurazione del repository;
	\item \textbf{feature/}: categoria di branch all'interno dei quali vengono effettuati incrementi specifici agli oggetti;
	\item \textbf{fix/}: categoria di branch all'interno dei quali vengono fatte delle correzioni specifiche o estese ad un oggetto.
\end{itemize}
Nei branch \textbf{feature/} e \textbf{fix/}, si aggiunge un suffisso che indica l'elemento di interesse della modifica ed esso si può trovare
nella forma del nome esteso dell'oggetto oppure mediante la sua sigla se presente in \refSec{par:sigle}. L'area della modifica può essere
maggiormente specificata separando i livelli gerarchici con il carattere \emph{/} (e.g. feature/NdP/processi\_di\_supporto/gestione\_configurazione).

Il repository \emph{swe-docs} prevede inoltre la categoria di branch \textbf{doc/}; vi è un branch per documento e ciascuno di
essi mantiene l'ultima versione stabile del relativo documento. Nei branch appartenenti alla categoria viene aggiunto come suffisso la
sigla di un documento in \refSec{par:sigle}, essa specifica il documento di riferimento del branch.

\paragraph{Gestione delle modifiche}
\label{par:gestione_modifiche}
Ogni membro del gruppo può modificare gli oggetti nei repository; le modifiche possono essere effettuate liberamente nei branch
\emph{feature/} e \emph{fix/}, mentre in \emph{develop} possono essere effettuati \glossario{hotfix} in fase di rilascio. Nei restanti
branch invece le modifiche vengono implementate esclusivamente mediante merge da \glossario{pull request}.

Una modifica prevede i seguenti passi:
\begin{enumerate}
	\item Creazione di un branch di lavoro secondo \refSec{spar:creazione_branch}, utilizzando come base \emph{develop} o il
		branch del documento di interesse \emph{doc/} nel repository \emph{swe-docs};
	\item Implementazione della modifica e/o dell'incremento e aggiunta al repository secondo \refSec{spar:aggiunta_modifiche};
	\item Apertura della pull request, sul branch base di partenza, da parte dell'assegnatario del ticket della modifica;
	\item Verifica da parte di un \emph{verificatore} delle modifiche effettuate;
	\item Accettazione della pull request, con conseguente merge delle modifiche ed eliminazione del branch di lavoro da parte del
		\emph{verificatore}.
\end{enumerate}

\paragraph{Comandi Git}
\label{par:comandi_git}

Vengono di seguito presentati i comandi da eseguire nel terminale per effettuare le seguenti operazioni:
\begin{itemize}
	\item Creazione di un branch di lavoro;
	\item Aggiunta delle modifiche nel repository locale e remoto.
\end{itemize}

\subparagraph{Creazione di un branch di lavoro}
\label{spar:creazione_branch}

Per creare un nuovo branch di lavoro, assumendo che l'albero delle modifiche locale sia vuoto, svolgere i seguenti passi:
\begin{enumerate}
	\item Aprire il terminale e spostarsi sulla cartella del repository su cui si vuole operare;
	\item Spostarsi sul branch di base (normalmente \emph{develop} oppure un branch \emph{doc/}):\\
			\texttt{git checkout [branch base]};
	\item Includere in locale le eventuali modifiche nel branch remoto:\\
			\texttt{git pull}
	\item In caso di conflitti:
		\begin{itemize}
			\item Importare le modifiche nel riferimento al branch remoto locale:\\
					\texttt{git fetch origin [branch base]};
			\item Forzare la versione dell'oggetto del branch remoto come corrente:\\
					\texttt{git reset --hard origin/[branch base]}
		\end{itemize}
	\item Creare il branch di lavoro sulla base del \emph{branch base}, seguendo le regole in \refSec{par:gestione_configurazione:branching}:\\
			\texttt{git checkout -b [branch di lavoro] [branch base]}
	\item Aggiungere il branch di lavoro creato in locale al repository pubblico:\\
			\texttt{git push -u origin [branch di lavoro]}
\end{enumerate}
\textbf{NOTE}: \texttt{[branch base]} e \texttt{[branch di lavoro]} sono dei placeholder generici che indicano il ruolo del branch di cui
bisogna inserire il nome; il branch effettivamente indicato dovrà obbligatoriamente essere presente in \refSec{par:gestione_configurazione:branching}.

\subparagraph{Aggiunta delle modifiche nei repository locale e remoto}
\label{spar:aggiunta_modifiche}

Per aggiungere le modifiche effettuate al repository locale e a quello remoto svolgere i seguenti passi:
\begin{enumerate}
	\item Aprire il terminale e spostarsi sulla cartella del repository su cui si vuole operare;
	\item Sincronizzare il repository locale con quello pubblico:\\
		\texttt{git pull}
	\item Aggiungere il tracciamento dei file modificati:\\
		\texttt{git add *}
	\item Aggiungere le modifiche al repository locale accompagnandole con un messaggio esplicativo delle stesse:\\
		\texttt{git commit -m "[messaggio riguardante le modifiche effettuate]"}
	\item Aggiungere le modifiche effettuate al repository pubblico:\\
		\texttt{git push}
\end{enumerate}

\subsubsection{Strumenti}
\label{sub:gestione_configurazione:strumenti}

Gli strumenti utilizzati a supporto delle attività del processo sono i seguenti:
\begin{itemize}
	\item \textbf{Git}: software di controllo di versione distribuito;
	\item \textbf{GitHub}: sistema di controllo di versione cloud;
	\item \textbf{GitHub Desktop}: client desktop dell'omonimo sistema di versionamento;
\end{itemize}



\subsection{Accertamento della qualità}
\label{sub:accertamento_qualita}

\subsubsection{Descrizione}
L'accertamento della qualità è un processo che racchiude all'interno diverse attività volte a provare e garantire oggettivamente il
soddisfacimento degli obiettivi di qualità di un prodotto o di un processo. All'interno del documento \textsc{Piano di Qualifica}
andranno inseriti gli obiettivi di qualità stabiliti, le
relative metriche individuate e un'istantanea del livello di qualità dei diversi elementi presi in esame misurata sulla base delle
metriche scelte. Le attività aggregate dal processo in questione sono le seguenti:
\begin{itemize}
    \item \textbf{Qualità del prodotto};
    \item \textbf{Qualità del processo};
\end{itemize}

\subsubsection{Attività}

\paragraph{Qualità del prodotto}
\label{par:qualita_del_prodotto}
\subparagraph{Scopo}
\label{par:qualita_del_prodotto:scopo}
In questa attività ci si assicura che il prodotto software e la documentazione ad esso associata rispetti gli standard di qualità.
\subparagraph{Compiti}
\label{par:qualita_del_prodotto:compiti}
Questa attività è composta dai seguenti compiti:
\begin{itemize}
    \item \textbf{Controllo della documentazione}: si controlla che tutta la documentazione richiesta dia fornita e che rispetti le convenzioni stabilite dal contratto;
    \item \textbf{Controllo del prodotto software}: si controlla che il prodotto software rispetti i requisiti richiesti dal contratto;
    \item \textbf{Preparazione alla consegna}: si controlla in preparazione alla consegna che tutti i prodotti abbiano soddisfatto a pieno i requisiti contrattuali.
\end{itemize}

\paragraph{Qualità del processo}
\label{par:qualita_del_processo}
\subparagraph{Scopo}
\label{par:qualita_del_processo:scopo}
In questa attività ci si assicura che ogni processo svolto rispetti gli standard di qualità.
\subparagraph{Compiti}
\label{par:qualita_del_processo:compiti}
Questa attività è composta dai seguenti compiti:
\begin{itemize}
    \item \textbf{Controllo del ciclo di vita}: si controlla che il ciclo di vita scelto sia applicato correttamente al prodotto;
    \item \textbf{Controllo dell'ambiente di sviluppo}: si controlla che l'ambiente di sviluppo (pratiche ingegneristiche adottate, ambiente di test, librerie esterne...) rispettino la qualità richiesta dal contratto;
    \item \textbf{Controllo delle metriche}: si controlla che le metriche di qualità siano in accordo con quanto stabilito;
    \item \textbf{Controllo del personale}: si controlla che il personale tecnico abbia le competenze necessarie allo svolgimento del progetto e che ricevano la formazione adeguata.
\end{itemize}

Di seguito vengono definiti gli strumenti utilizzati a supporto delle attività del processo, le norme e la struttura
degli obiettivi di qualità e delle metriche.

\subsubsection{Piano di Qualità}
\label{ssub:pianificazione_qualita}

\paragraph{Scopo}

In questa attività viene stabilito il concetto di qualità dell'elemento preso in esame; ciò avviene mediante l'individuazione di obiettivi
di qualità che si desidera raggiungere nello stesso. Inoltre vengono stabilite le metriche con le quali viene misurato il livello di
raggiungimento degli obiettivi di qualità imposti, viene così definito implicitamente come perseguire la qualità.

\paragraph{Aspettative}
Le aspettative di questo processo sono:
\begin{itemize}
    \item conseguimento della qualità nel prodotto secondo le richieste del proponente;
    \item prova oggettiva della qualità del prodotto;
    \item conseguimento della qualità nell'organizzazione delle attività e dei processi;
    \item raggiungimento della soddisfazione del proponente.
\end{itemize}

\paragraph{Attività}
Le attività principali da svolgere sono:
\begin{itemize}
    \item \textbf{Pianificazione:} occorre porsi degli obiettivi di qualità, definire le strategie per raggiungerli, disporre delle risorse nel modo migliore;
    \item \textbf{Valutazione:} mettere in atto la pianificazione monitorando i risultati;
    \item \textbf{Reazione:} sulla base dei risultati rivedere le strategie applicate, se dovessero esser risultate non conformi.
\end{itemize}

\paragraph{Obiettivo di qualità}
\label{par:obiettivo_qualita}

Ciascun obiettivo di qualità è composto dai seguenti elementi:
\begin{itemize}
	\item codice: permette l'identificazione univoca;
	\item titolo: sintetizza il contenuto;
	\item descrizione: fornisce una spiegazione esaustiva.
\end{itemize}

Il codice di un obiettivo di qualità deve avere la seguente forma:
\begin{center}
	\textbf{O[Tipologia][ID]}
\end{center}
All'interno del codice \textbf{Tipologia} indica a quale tipologia di elemento l'obiettivo sia associato e può assumere i seguenti valori:
\begin{itemize}
	\item \textbf{PR}: l'obiettivo è relativo ad un processo;
	\item \textbf{S}: l'obiettivo è relativo ad un prodotto software;
    \item \textbf{D}: l'obiettivo è relativo ad un prodotto documentale.
\end{itemize}
\textbf{ID} invece è un intero positivo che distingue univocamente gli obiettivi relativi a tipologie di elementi uguali.

\paragraph{Metrica di qualità}
\label{par:metrica_qualita}

Ciascuna metrica di qualità è associata ad uno e un solo obiettivo ed è composta dai seguenti elementi:
\begin{itemize}
	\item codice: permette l'identificazione univoca;
	\item titolo: sintetizza il contenuto;
	\item descrizione: fornisce una spiegazione esaustiva di cosa misura la metrica e come viene effettuata la misura;
	\item valore accettabile: definisce il valore della metrica con il quale l'obiettivo viene ritenuto conseguito;
	\item valore preferibile: definisce il valore della metrica che il gruppo punta a raggiungere.
\end{itemize}

Il codice di una metrica di qualità deve avere la seguente forma:
\begin{center}
    \textbf{M[Tipologia][ID]}
\end{center}
All'interno del codice \textbf{[Tipologia]} indica a quale tipologia di elemento la metrica è associata, questo valore deve coincidere con
quello del codice dell'obiettivo associato e può assumere i seguenti valori:
\begin{itemize}
	\item \textbf{PR}: la metrica è relativa ad un processo;
	\item \textbf{S}: la metrica è relativa ad un prodotto software;
    \item \textbf{D}: la metrica è relativa ad un prodotto documentale.
\end{itemize}
\textbf{ID} invece è un intero positivo che distingue univocamente le metriche relative a tipologie di elementi uguali.

\paragraph{Controllo di Qualità}
\label{par:controllo_qualita}

In questa attività vengono calcolate le metriche di controllo precedentemente stabilite dell'elemento preso in esame e ne viene verificato
il soddisfacimento o meno dei valori minimi e preferibili.

\paragraph{Strumenti}
\label{par:gestione_qualita:strumenti}

Di seguito vengono indicati gli strumenti utilizzati nell'attuazione del processo:
\begin{itemize}
	\item \textbf{Farfalla project}: applicazione open source che fornisce strumenti di accessibilità, in particolare presenta uno
		strumento per il calcolo della leggibilità del testo al link \url{https://farfalla-project.org/readability_static/};
	\item \textbf{Aspell}: strumento da linea di comando per la correzione degli errori ortografici.
\end{itemize}



\subsection{Verifica}
\label{sub:verifica}

\subsubsection{Descrizione}
Questo processo racchiude un insieme di attività volte a fornire una prova oggettiva del soddisfacimento dei requisiti e delle norme dei
processi per la fase in esame. Ciò avviene prendendo in ingresso i prodotti dei processi attuati, controllando che essi siano completi e corretti
rispetto ai requisiti degli stessi e producendo un report delle valutazioni effettuate nel documento \textsc{Piano di Qualifica}; in
generale le operazioni di verifica sui prodotti devono essere chiare, oggettive ed affidabili.

Il processo di verifica adottato dal gruppo \emph{SpaghettiCode} consiste nelle seguenti attività definite dallo standard di processo \emph{ISO/IEC 12207:1997}, opportunamente adattate alle necessità del progetto didattico:
\begin{itemize}
	\item \textbf{Implementazione};
	\item \textbf{Verifica}.
\end{itemize}

\subsubsection{Attività}

\paragraph{Implementazione}
\label{par:implementazione}
\subparagraph{Scopo}
\label{par:implementazione:scopo}
In questa attività si definiscono le criticità del processo di verifica e si stabilisce come svolgere al meglio il compito di verifica\\
\subparagraph{Compiti}
\label{par:implementazione:compiti}
Questa attività è composta dai seguenti compiti:
\begin{itemize}
    \item \textbf{Definizione del processo di verifica}: si analizzano i requisiti del progetto e si pianifica il processo di verifica;
    \item \textbf{Definizione delle attività di verifica}: si definiscono i metodi, le tecniche ed i mezzi per eseguire il compito di verifica;
    \item \textbf{Documentazione delle attività di verifica}: si documenta il processo di verifica;
    \item \textbf{Implementazione delle attività di verifica}: si implementano le attività di verifica. Le non conformità al modello stilato devono venire documentate, analizzate e risolte.
\end{itemize}


\paragraph{Verifica}
\label{par:verifica}

\subparagraph{Scopo}
Questa attività ha il compito di valutare la correttezza e la conformità dei prodotti di processo alle regole; essa può essere svolta sia
sui prodotti software che sui prodotti documentali del progetto, in quanto valuta caratteristiche scisse dall'esecuzione del prodotto e
dunque immutabili. Essa verrà svolta in parte automaticamente e in parte manualmente; in maniera automatica verranno verificati gli
aspetti riguardanti la formattazione e la presenza degli elementi nei prodotti, mentre manualmente verranno verificati i contenuti dei
prodotti seguendo un metodo di lettura \glossario{Inspection}, ossia valutando attentamente gli incrementi apportati in seguito alla
precedente verifica e gli elementi strettamente legati ad essi.

\subparagraph{Compiti}
\label{par:verifica:compiti}

\subparagraph{Analisi dinamica}
\label{sub:verifica:analisi_dinamica}

\subparagraph{Scopo}

Questa attività ha il compito di studiare e verificare il corretto comportamento del prodotto in un insieme finito di casi; ciò richiede
l'esecuzione dell'oggetto di verifica, che deve obbligatoriamente essere un prodotto software ed utilizza un insieme di prove denominate
test.

\subparagraph{Test}
\label{par:verifica:test}

I test sono una componente fondamentale dell'analisi dinamica in quanto permettono di valutare il comportamento del prodotto software e
verificarne la correttezza.\\
Ciascun test è costituito dai seguenti elementi:
\begin{itemize}
    \item \textbf{Oggetto}: elemento di cui si verifica il comportamento;
    \item \textbf{Input richiesti}: dati in ingresso con il quale viene eseguito il test;
    \item \textbf{Output attesi}: output attesi in relazione agli input;
    \item \textbf{Ambiente}: ambiente nel quale viene eseguito il test;
    \item \textbf{Stato iniziale}: stato iniziale dell'ambiente;
    \item \textbf{Passi di esecuzione}: passaggi che si desidera eseguire per giungere all'output atteso.
\end{itemize}

Inoltre i test vengono descritti da un codice identificativo univoco che permette di distinguerli e ne viene fornito lo stato che può
assumere uno dei seguenti valori:
\begin{itemize}
    \item \textbf{I}: il test è stato implementato;
    \item \textbf{NI}: il test non è ancora stato implementato;
    \item \textbf{S}: il test è stato eseguito e superato;
    \item \textbf{NS}: il test è stato eseguito ma non è stato superato.
\end{itemize}

Il codice identificativo si presenta nella forma:
\begin{center}
    \textbf{T[Tipologia][ID]}
\end{center}
Nella forma appena descritta \textbf{Tipologia} indica la tipologia del test e può assumere uno tra i seguenti valori:
\begin{itemize}
	\item \textbf{S}: Sistema. \\ Testano l’applicazione nella sua interezza. Viene effettuato un controllo sulle componenti del sistema che devono risultare compatibili fra loro e conformi nelle loro interazioni.
    Questi test caratterizzano la validazione del prodotto finale.
	\item \textbf{I}: Integrazione. \\ Viene verificato il funzionamento di diverse parti del sistema legate tra di loro da una definita relazione.
    Rappresentano l’estensione logica dei test di unità e sono i predecessori del test di sistema. La strategia che i test d’integrazione adottano è una strategia di tipo incrementale: si basa nella prosecuzione per passi, aggiungendo parti fino al completamento.
	\item \textbf{U}: Unità. \\ Servono a verificare la correttezza di piccoli componenti. L’obiettivo è di isolare la parte più piccola di software testabile, chiamata unità, e poter stabilire il corretto funzionamento.
    Ogni singola unità deve essere testata prima di essere integrata. Per farlo si possono usare simulatori o driver.

\end{itemize}

\textbf{ID} invece è un valore intero positivo che permette di identificare univocamente i test della stessa tipologia in base ad una
numerazione progressiva.

\subparagraph{Strumenti}
\label{ssub:verifica:strumenti}

Di seguito vengono riportati gli strumenti utilizzati nelle diverse attività del processo di verifica:
\begin{itemize}
	\item \textbf{ESLint}: strumento open source di analisi statica di software in linguaggio JavaScript;
	\item \textbf{Jest}: framework di testing per il linguaggio JavaScript incentrato sulla semplicità e l'espressività;
	\item \textbf{Selenium}: framework per il testing di applicazioni web;
	\item \textbf{W3C HTML Validator}: strumento di analisi di codice HTML che verifica la validità dei file sorgenti, individuabile al
		link \url{https://validator.w3.org};
	\item \textbf{W3C CSS Validator}: strumento di analisi di codice CSS che verifica la validità dei file sorgenti, individuabile al
		link \url{https://jigsaw.w3.org/css-validator/}.
\end{itemize}

\subsection{Validazione}
\label{sub:validazione}

\subsubsection{Descrizione}
Questo processo ha il compito di verificare la correttezza e la completezza del prodotto software rispetto alle attese del committente e
del proponente. Successivamente alla validazione si garantisce che il software rispetta i requisiti richiesti dal committente.
Il processo viene attuato al termine del processo di sviluppo, riceve in ingresso il prodotto software nella sua interezza
e produce il rilascio dello stesso. Se i test danno esiti positivi, il \emph{Responsabile} può accettare ed approvare il prodotto oppure chiedere ulteriori verifiche con opportune indicazioni.\\

Il processo di validazione adottato dal gruppo \emph{SpaghettiCode} consiste nelle seguenti attività definite dallo standard di processo \emph{ISO/IEC 12207:1997}, opportunamente adattate alle necessità del progetto didattico:
\begin{itemize}
	\item \textbf{Implementazione};
	\item \textbf{validazione}.
\end{itemize}

\subsubsection{Attività}

\paragraph{Implementazione}
\label{par:Implementazione}
\subparagraph{Scopo}
\label{par:Implementazione:scopo}
In questa attività si definiscono le criticità del processo di validazione e si stabilisce come svolgere al meglio il compito di verifica\\
\subparagraph{Compiti}
\label{par:Implementazione:compiti}
Questa attività è composta dai seguenti compiti:
\begin{itemize}
    \item \textbf{Definizione del processo di validazione}: si analizzano i requisiti del progetto e si pianifica il processo di validazione;
    \item \textbf{Definizione delle attività di validazione}: si definiscono i metodi, le tecniche ed i mezzi per eseguire il compito di validazione;
    \item \textbf{Documentazione delle attività di verifica}: si documenta il processo di validazione;
    \item \textbf{Implementazione delle attività di validazione}: si implementano le attività di validazione. Le non conformità al modello stilato devono venire documentate, analizzate e risolte.
\end{itemize}

\paragraph{Validazione}
\label{par:validazione}
\subparagraph{Scopo}
\label{par:validazione:scopo}
In questa attività si definiscono le criticità del processo di verifica e si stabilisce come svolgere al meglio il compito di verifica\\
\subparagraph{Compiti}
\label{par:validazione:compiti}
Questa attività è composta dai seguenti compiti:
\begin{itemize}
    \item \textbf{Preparazione dei test}: si selezionano i requisiti e le specifiche per analizzare i test e i risultati dei test;
    \item \textbf{Verifica dei requisiti dei test}: si verifica che i requisiti per i test soddisfino lo specifico caso del progetto;
    \item \textbf{Esecuzione dei test}: si eseguono i test tra cui stress test e test dei valori limite. Si verifica che utenti rappresentativi dell'utente medio siano in grado di operare il software;
    \item \textbf{validazione del software}: si verifica che il software funzioni in modo corretto.
\end{itemize}

\subsubsection{Test}
\label{ssub:test}
I test svolti al fine di validare il software sono le seguenti:
\begin{itemize}
	\item \textbf{Test di sistema}: viene verificato il corretto comportamento del sistema e la
    copertura di tutti i requisiti del prodotto;
	\item \textbf{Collaudo}: viene dimostrata la conformità del prodotto rispetto al contratto tra le
    parti verificando la correttezza del prodotto rispetto ad alcuni casi di prova.
\end{itemize}

I test di integrazione permettono l’esecuzione dei test di sistema svolgendo un controllo sulle funzionalità implementate, quindi si stabilisce con certezza che si sta rispettando quanto dichiarato. \\
Il processo di validazione prevede:

\paragraph{Strumenti}
\label{par:validazione:strumenti}

Di seguito vengono riportati gli strumenti utilizzati nelle diverse attività del processo di verifica:
\begin{itemize}
	\item \textbf{Selenium}: framework per il testing di applicazioni web;
\end{itemize}

\section{Processi primari}
\label{sec:processi_primari}

\subsection{Fornitura}
\label{sub:fornitura}

\subsubsection{Descrizione}
Il processo di fornitura aggrega le diverse attività che il gruppo \emph{SpaghettiCode} dovrà svolgere al fine di fornire il prodotto \emph{HD Viz} del \glossario{proponente} \emph{Zucchetti S.p.A.} e dei committenti \emph{prof. Tullio Vardanega} e \emph{prof. Riccardo Cardin}.\\

In questa sezione vengono descritte e normate tutte le attività ed i prodotti documentali che il gruppo \emph{SpaghettiCode} intende svolgere e produrre durante la fornitura del prodotto software \emph{HD Viz}.\\

Il processo di fornitura adottato dal gruppo \emph{SpaghettiCode} consiste nelle seguenti attività definite dallo standard di processo \emph{ISO/IEC 12207:1997}, opportunamente adattate alle necessità del progetto didattico:
\begin{itemize}
    \item \textbf{Implementazione};
    \item \textbf{Preparazione della risposta};
    \item \textbf{Pianificazione};
    \item \textbf{Esecuzione e controllo};
    \item \textbf{Revisione e valutazione};
    \item \textbf{Consegna e completamento}.
\end{itemize}

Il processo di fornitura produce i seguenti output documentali:
\begin{itemize}
    \item \textbf{Studio di fattibilità};
    \item \textbf{Lettera di presentazione};
	\item \textbf{Piano di Progetto};
	\item \textbf{Piano di Qualifica}.
\end{itemize}

\subsubsection{Attività}

\paragraph{Implementazione}
\label{par:Implementazione}
\subparagraph{Scopo}
\label{par:Implementazione:scopo}
In questa attività, il gruppo \emph{SpaghettiCode} si impegna ad eseguire uno studio di fattibilità per ogni capitolato presentato in data 2020-11-05, in modo da stabilire se lo svolgimento di un determinato capitolato è in grado di portare un ritorno sull'investimento di tempo e risorse.Questa attività ha come output il documento \textsc{Studio di fattibilità} da parte degli \emph{analisti}.\\
\subparagraph{Compiti}
\label{par:Implementazione:compiti}
Questa attività è composta dai seguenti compiti:
\begin{itemize}
    \item \textbf{Raccolta di informazioni generali}: si raccolgono tutte le informazioni basilari riguardanti il
    capitolato che comprendono nome, proponente e committente;
    \item \textbf{Comprensione delle caratteristiche}: si studiano e comprendono le caratteristiche del prodotto che
    deve essere sviluppato;
    \item \textbf{Comprensione dello scopo del progetto}: si studiano e comprendono i possibili fini del progetto;
    \item \textbf{Comprensione delle tecnologie interessate}: si studia e si determina quali siano le tecnologie
    interessate dal capitolato, se imposte o suggerite dal proponente;
    \item \textbf{Valutazione degli aspetti positivi}: vengono individuati gli aspetti positivi di ogni capitolato;
    \item \textbf{Valutazione dei rischi}: vengono individuati tutti i possibili rischi del capitolato proposto;
    \item \textbf{Conclusioni}: vengono ponderati gli aspetti positivi e gli aspetti negativi del capitolato, traendo
        quindi le conclusioni in merito alla fattibilità.
\end{itemize}

\paragraph{Preparazione della risposta}
\label{par:preparazione_della_risposta}
\subparagraph{Scopo}
\label{par:preparazione_della_risposta:scopo}
In questa attività, il gruppo \emph{SpaghettiCode} dichiara al proponente ed ai committenti \emph{prof. Tullio Vardanega} e \emph{prof. Riccardo Cardin}, di impegnarsi come fornitore per il capitolato scelto. Questa attività ha come output la stesura del documento \textsc{Lettera di presentazione}.\\
\subparagraph{Compiti}
\label{par:preparazione_della_risposta:compiti}
Questa attività è composta dai seguenti compiti:
\begin{itemize}
    \item \textbf{Scrittura della lettera di presentazione}: si prepara una lettera che formalizza la scelta di candidarsi come fornitori del capitolato scelto.
\end{itemize}

\paragraph{Pianificazione}
\label{par:pianificazione}
\subparagraph{Scopo}
\label{par:pianificazione:scopo}
In questa attività il gruppo \emph{SpaghettiCode} deve stabilire come pianificare il progetto, che \glossario{modello di sviluppo} software adottare, stabilire i requisiti e le strategie da seguire al fine di produrre un prodotto software di qualità. Questa attività ha come output la stesura del documento \textsc{\glossario{Piano di progetto}} da parte dagli \emph{amministratori}, sotto la supervisione del
\emph{\glossario{responsabile di progetto}}.\\
\subparagraph{Compiti}
\label{par:pianificazione:compiti}
Questa attività è scomposta nei seguenti compiti:
\begin{itemize}
    \item \textbf{Analisi dei rischi}: si analizzano i rischi che potrebbero presentarsi nel corso del progetto; vengono inoltre fornite le modalità con cui verranno risolti o ridimensionati. Un'analisi più esaustiva si troverà nel documento \textsc{Piano di progetto};
    \item \textbf{Scelta del modello di sviluppo}: viene scelto il \glossario{modello di sviluppo} che il gruppo decide di adottare in funzione dei requisiti imposti dal capitolato e le capacità lavorative dei componenti del gruppo;
    \item \textbf{Pianificazione}: si descrivono e pianificano le attività da eseguire nelle vari fasi del progetto, stabilendo le (\glossario{deadline}) per il loro completamento. Queste deadline non sono rigide perché, come accennato prima, la pianificazione sarà sempre soggetta a modifiche e/o aggiornamenti;
    \item \textbf{Stesura di preventivi e consuntivi}: viene stimata la quantità di lavoro necessaria per ogni obiettivo raggiunto durante lo svolgimento del progetto. Viene quindi esposto un \glossario{preventivo} e un successivo \glossario{consuntivo}, entrambi relativi ad un dato periodo.
\end{itemize}

\paragraph{Esecuzione e controllo}
\label{par:esecuzione_e_controllo}
\subparagraph{Scopo}
\label{par:esecuzione_e_controllo:scopo}
In questa attività si mette in pratica quanto definito nell'attività di pianificazione.
\subparagraph{Compiti}
\label{par:esecuzione_e_controllo:compiti}
Questa attività è scomposta nei seguenti compiti:
\begin{itemize}
    \item \textbf{Attuazione della pianificazione}: Sarà compito del \emph{responsabile di progetto} stabilire di volta in volta le scadenze e pianificare di conseguenza come pianificare le attività. Sarà invece l'\emph{amministratore di progetto} che dovrà occuparsi di evitare i conflitti e le sovrapposizioni di ruoli e responsabilità;
    \item \textbf{Monitoraggio dell'andamento del progetto}: Sarà compito del \emph{responsabile di progetto} tener conto dell'andamento delle attività, di far sì che non si accumuli ritardo e pianificare ed aggiornare le strategie risolutive più adatte per mitigare queste situazioni;
    \item \textbf{Applicazione delle modalità di risoluzione dei rischi riscontrati}: in caso si verifichino i rischi previsti o se ne incontrino di nuovi, è compito del \emph{responsabile di progetto} attuare le strategie risolutive più adeguate ed eventualmente aggiornarle.
\end{itemize}

\paragraph{Revisione e valutazione}
\label{par:revisione_e_valutazione}
\subparagraph{Scopo}
\label{par:revisione_e_valutazione:scopo}
Questa attività ha lo scopo di assicurare che tutti i compiti svolti dal gruppo \emph{SpaghettiCode}, in quanto fornitore, rispettino gli standard di qualità imposti dal progetto. Questa attività produce come output il documento \textsc{Piano di qualifica} redatto dai \emph{verificatori} \\
\subparagraph{Compiti}
\label{par:revisione_e_valutazione:Compiti}
Questa attività è scomposta nei seguenti compiti:
\begin{itemize}
    \item \textbf{Comunicazione con il proponente}: Si comunica in maniera più o meno formale e si tiene traccia delle decisioni prese con il proponente in modo da mantenere una comunicazione efficace e funzionale allo svolgimento del progetto;
    \item \textbf{Revisioni con i committenti}: Si effettuano periodicamente revisioni di avanzamento con i committenti \emph{prof. Tullio Vardanega} e \emph{prof. Riccardo Cardin}. Le revisioni verranno effettuate a sportello, quindi le scadenze verranno fissate man mano nel tempo, in base a quanto previsto dal \textsc{Piano di progetto};
    \item \textbf{Comunicazione interna}: Si comunica in maniera più o meno formale e si tiene traccia delle decisioni interne al gruppo prese in modo da mantenere una comunicazione efficace e funzionale allo svolgimento del progetto;
    \item \textbf{Verifica e validazione}: Si effettuano le attività di verifica e validazione su ogni prodotto software e documentale, in modo che soddisfi gli standard di qualità imposti dal \textsc{Piano di qualifica};
\end{itemize}
\paragraph{Piano di Qualifica}
\label{par:revisione_e_valutazione:Piano di Qualifica}
Il documento \textsc{Piano di qualifica}, output dell'attività di revisione e valutazione, è strutturato nel seguente modo:
\begin{itemize}
    \item \textbf{Qualità di processo}: sono individuati i processi dagli \glossario{standard di processo}, definiti
    degli obiettivi, strategie per attuarli e \glossario{metriche} per controllarli e misurarli;
    \item \textbf{Qualità di prodotto}: vengono individuate le caratteristiche più importanti del prodotto, gli
    obiettivi necessari per raggiungerle e le metriche per misurarle;
    \item \textbf{Specifiche dei test}: vengono definiti dei \glossario{test} che il prodotto deve superare per
    garantire il soddisfacimento dei requisiti;
    \item \textbf{Standard di qualità}: descritti gli \glossario{standard di qualità} selezionati;
    \item \textbf{Resoconto delle attività di verifica}: vengono esposti i risultati dei test eseguiti durante il
    periodo di revisione; le metriche usate per l'ottenimento di questi risultati sono redatte nel documento;
    \item \textbf{Lista di controllo}:  lista che contiene gli errori riscontrati (sezione in continua fase di
    aggiornamento per l'intera durata del progetto);
    \item \textbf{Valutazioni per il miglioramento}: vengono elencati i problemi riscontrati durante lo sviluppo del
    progetto e vengono proposte delle soluzioni che potrebbero portare alla risoluzione o alla mitigazione degli stessi.
\end{itemize}

In questo documento si fa uso di un codice identificativo dei rischi. Il codice avrà questa forma:
\begin{center}
	\textbf{[Tipologia][Codice]}
\end{center}
Dove:
\begin{itemize}

	\item \textbf{Tipologia}: indica la tipologia di rischio e può assumere tre valori:
	\begin{itemize}
		\item \textbf{RR}: indica un rischio legato ai requisiti;
		\item \textbf{RT}: indica un rischio legato alle tecnologie;
		\item \textbf{RO}: indica un rischio legato all'organizzazione.
	\end{itemize}
    \item \textbf{Codice}: Il codice è un numero progressivo univoco all'interno della tipologia, che permette di
    identificare univocamente il rischio.
\end{itemize}

Ogni rischio avrà, oltre al codice, due informazioni:
\begin{itemize}
	\item \textbf{Occorrenza}: indica a frequenza con cui occorre il rischio;
	\item \textbf{Gravità}: indica la gravità del rischio.
\end{itemize}

\paragraph{Consegna e completamento}
\label{par:consegna_e_completamento}
\subparagraph{Scopo}
\label{par:consegna_e_completamento:scopo}
Questo, essendo un progetto didattico, non prevede la parte più lunga ed impegnativa del ciclo di vita di un prodotto software; pertanto la consegna del prodotto software e della documentazione ad esso legata, è l'ultima attività del ciclo di vita del prodotto \emph{HD Viz}.\\
\subparagraph{Compiti}
\label{par:consegna_e_completamento:compiti}
Questa attività è composta dal seguente compito:
\begin{itemize}
	\item \textbf{Consegna del progetto}: Il gruppo \emph{SpaghettiCode} si assume la responsabilità di mettere a disposizione un riferimento tramite il quale mette a disposizione dei committenti e del proponente il prodotto software e tutti i prodotti documentali necessari;
\end{itemize}


\subsection{Sviluppo}
\label{sub:sviluppo}

\subsubsection{Descrizione}
\label{ssub:sviluppo:descrizione}
% TODO: aggiungere integrazione, test, installazione e accettazione
Il processo di sviluppo aggrega le diverse attività che il gruppo \emph{SpaghettiCode} dovrà svolgere al fine di fornire il prodotto \emph{HD Viz} del \glossario{proponente} \emph{Zucchetti S.p.A.} e dei committenti \emph{prof. Tullio Vardanega} e \emph{prof. Riccardo Cardin}.\\

In questa sezione vengono descritte e normate tutte le attività ed i prodotti documentali che il gruppo \emph{SpaghettiCode} intende svolgere per la produzione corretta ed efficace del prodotto software di cui si è fornitori\\

Il processo di sviluppo adottato dal gruppo \emph{SpaghettiCode} consiste nelle seguenti attività definite dallo standard di processo \emph{ISO/IEC 12207:1997}, opportunamente adattate alle necessità del progetto didattico:
\begin{itemize}
	\item \textbf{Analisi};
	\item \textbf{Progettazione};
    \item \textbf{Codifica};
    \item \textbf{Test};
\end{itemize}
Il processo di fornitura produce i seguenti output documentali:
\begin{itemize}
    \item \textbf{Analisi dei requisiti};
    \item \textbf{Manuale Sviluppatore};
    %TODO: prob c'è da spostare il manuale utente da qualche altra parte
	\item \textbf{Manuale Utente}.
\end{itemize}

\subsubsection{Attività}

\paragraph{Analisi}
\label{ssub:analisi}
\subparagraph{Scopo}
\label{par:analisi:scopo}
In questa attività, gli \emph{analisti} individuano e stabiliscono le caratteristiche del prodotto di comune accordo con il proponente. Questa attività ha come output documentale il documento \textsc{Analisi dei Requisiti}, che funge da vincolo contrattuale tra fornitore e proponente, all'interno del quale vengono definite le funzionalità che il prodotto deve offrire e vengono specificati i requisiti che deve soddisfare.\\
\subparagraph{compiti}
\label{par:analisi:compiti}
Questa attività è scomposta nei seguenti compiti:
\begin{itemize}
	\item \textbf{individuazione delle funzionalità}: Le funzionalità inoltre svolgono funzione di supporto per i requisiti, oltre ad essere
    di chiara comprensione anche per persone non esperte del settore informatico. Le funzionalità del prodotto vengono rappresentate mediante i casi d'uso e sono ricavate dalle seguenti fonti:
    \begin{itemize}
        \item Capitolato d'appalto: prima descrizione del prodotto messa a disposizione dal proponente;
        \item Dominio del problema: ambito nel quale rientra il problema;
        \item Incontri interni: discussioni interne tra i membri del gruppo;
        \item Incontri esterni: discussioni dei membri del gruppo con il proponente o con i committenti.
    \end{itemize}
	\item \textbf{definizione dei requisiti}: i requisiti fissano le caratteristiche del prodotto concordate con il cliente, forniscono indicazioni chiare i progettisti ed ai verificatori. Inoltre permettono di stimare quanto lavoro è stato fatto e quanto ce ne sia ancora da fare. I requisiti possono essere ricavati dalle seguenti fonti:
    \begin{itemize}
        \item Capitolato d'appalto: prima descrizione del prodotto messa a disposizione dal proponente;
        \item Casi d'uso: funzionalità del prodotto;
        \item Incontri interni: discussioni interne tra i membri del gruppo;
        \item Incontri esterni: discussioni dei membri del gruppo con il proponente o con i committenti.
    \end{itemize}
\end{itemize}
\paragraph{Analisi dei requisiti}
\label{par:analisi_dei_requisiti}
Il documento \textsc{Analisi dei requisiti}, output dell'omonima attività, struttura i casi d'uso e i requisiti nella modalità seguente:
\subparagraph{Classificazione dei casi d'uso}
\label{par:classificazione_casi_duso}
Ogni caso d'uso ha una struttura ben definita, riportata di seguito:
\begin{itemize}
    \item \textbf{Descrizione}: breve descrizione del caso d'uso;
    \item \textbf{Attore primario}: entità che interagisce direttamente con il prodotto;
    \item \textbf{Attori secondari}: entità che supportano l'attore primario nel portare a termine il caso d'uso in
    	esame (questo elemento non è necessariamente presente);
    \item \textbf{Precondizione}: condizione del sistema necessaria affinché possa compiersi il caso d'uso in esame;
    \item \textbf{Postcondizione}: condizione in cui si trova il sistema immediatamente dopo il compimento del caso
    d'uso in esame;
    \item \textbf{Scenario principale}: rappresentazione del flusso degli eventi previsti dal caso d'uso;
    \item \textbf{Scenari alternativi}: rappresentazioni alternative del flusso degli eventi del caso d'uso (questo
    	elemento non è necessariamente presente);
    \item \textbf{Estensioni}: elemento che indica casi d'uso eseguiti condizionatamente che determinano l'interruzione
    	dell'esecuzione del caso d'uso in esame (questo elemento non è necessariamente presente);
    \item \textbf{Inclusioni}: elemento che indica casi d'uso eseguiti incondizionatamente successivamente a quello in
    	esame (questo elemento non è necessariamente presente);
    \item \textbf{Generalizzazioni}: rappresentano delle possibili specializzazioni del caso d'uso (questo elemento non è necessariamente
    	presente).
\end{itemize}

Inoltre per chiarezza e manutenibilità a ciascun caso d'uso viene associato un codice immutabile che lo identifica univocamente
all'interno del progetto. Tale codice si presenta nella seguente forma:
\begin{center}
    \textbf{UC([Sistema])[CodiceBase](.[CodiceSottoCaso])*}
\end{center}
\emph{HD Viz}, altrimenti il campo può assumere i seguenti valori:
\begin{itemize}
	\item \textbf{S}: il caso d'uso ha luogo nel server di \emph{HD Viz}.
    \item \textbf{W}: il caso d'uso ha luogo nella Web app di \emph{HD Viz}.
\end{itemize}
\textbf{CodiceBase} è un intero positivo che distingue i casi d'uso, relativi allo stesso sistema, rispetto al primo livello di
granularità; \textbf{CodiceSottoCaso} invece è un intero positivo che distingue i casi d'uso relativi allo stesso sistema rispetto ai
livelli di granularità successivi nel caso siano presenti.

\subparagraph{Classificazione dei requisiti}
\label{par:classificazione_requisiti}
I requisiti sono composti dai seguenti elementi:
\begin{itemize}
	\item \textbf{Descrizione}: descrizione chiara del requisito in esame;
	\item \textbf{Importanza}: rilevanza di un requisito rispetto alle esigenze del prodotto;
	\item \textbf{Fonte}: riferimento univoco alla fonte dalla quale il requisito è stato individuato (capitolato d'appalto, caso d'uso,
		verbali interni o verbali esterni).
\end{itemize}

Inoltre ciascun requisito viene associato ad un codice immutabile che lo identifica univocamente che si presenta nella seguente forma:
\begin{center}
    \textbf{R[Tipologia][Importanza][Codice]}
\end{center}

Questa notazione, spiegata nel dettaglio di seguito, permette di identificare in modo univoco ogni requisito del prodotto.
\begin{itemize}
    \item \textbf{Tipologia}: permette di identificare la tipologia del requisito e i possibili valori che può assumere sono:
    \begin{itemize}
        \item \textbf{V}: indica che è un requisito di vincolo, ossia una condizione imposta dal proponente circa i servizi offerti dal
        	prodotto software all'utilizzatore finale;
        \item \textbf{F}: indica che è un requisito funzionale, cioè un vincolo riguardo le funzioni del prodotto;
        \item \textbf{P}: indica che è un requisito prestazionale, ossia specifica delle condizioni sulle prestazioni del prodotto software;
        \item \textbf{Q}: indica che è un requisito di qualità, quindi pone dei vincoli sulla qualità del prodotto.
    \end{itemize}
    \item \textbf{Importanza}: permette di individuare quale sia la rilevanza del requisito ed i suoi possibili valori sono:
    \begin{itemize}
        \item \textbf{O}: indica che il requisito è obbligatorio, quindi questo requisito dovrà essere necessariamente
        soddisfatto;
        \item \textbf{D}: indica che il requisito è desiderabile, quindi eventualmente negoziabile con il proponente.
        Il soddisfacimento del requisito verrebbe visto positivamente dal proponente, in quanto fornirebbe al prodotto
        una maggiore completezza, eppure non ne viene vincolata l'implementazione;
        \item \textbf{F}: indica che il requisito è facoltativo, dunque anche se porta un valore aggiunto al prodotto
        comporta una piccola miglioria a dispendio di molto tempo e lavoro.
    \end{itemize}

    \item \textbf{Codice}: permette di distinguere i requisiti appartenenti alla stessa tipologia e si presenta nella forma
    	\begin{center}
        	\textbf{[CodiceBase](.[CodiceSottoCaso])*}
    	\end{center}
    	dove \textbf{CodiceBase} e \textbf{CodiceSottoCaso} sono interi positivi ed essi permettono di correlare e distinguere i requisiti
    	specificando diversi livelli di granularità.
\end{itemize}

\paragraph{Progettazione}
\label{ssub:progettazione}

\subparagraph{Scopo}
\label{par:progettazione:scopo}
Lo scopo di quest'attività è quello di individuare le caratteristiche che il prodotto deve avere per soddisfare nel
modo migliore possibile le richieste del proponente in risposta ai requisiti individuati dall'analisi dei requisiti. \\
In quest'attività bisogna rispettare i seguenti vincoli:
\begin{itemize}
    \item Garantire la qualità del prodotto seguendo un principio di correttezza costruttivo;
    \item Organizzare e suddividere i compiti in modo da diminuire la complessità del problema, riducendolo via via in
    sottoproblemi sempre più elementari fino ad arrivare ai singoli componenti;
    \item Ottimizzare l'uso di risorse.
\end{itemize}

La progettazione è divisa in due parti fondamentali:
\begin{itemize}
    \item \glossario{Technology Baseline}: contiene le specifiche ad alto livello della progettazione del software,
    i relativi diagrammi \glossario{UML} e dei test;
    \item \glossario{Product Baseline}: arricchisce di dettagli quanto specificato nella Technology baseline e
    definisce i test necessari.
\end{itemize}

\subparagraph{Aspettative}

La progettazione è un'attività svolta dai \emph{Progettisti}, volta a produrre l'architettura logica del prodotto.
L'architettura deve essere formata da componenti chiari, riusabili e utilizzabili in modo che ci sia coesione tra
le parti.\\
L'architettura dovrà necessariamente rispettare i seguenti punti:
\begin{itemize}
    \item Soddisfare i requisiti individuati dall'analisi dei requisiti;
    \item Adattarsi in caso i requisiti evolvano;
    \item Riuscire a gestire situazioni erronee;
    \item Risultare affidabile anche in situazioni sfavorevoli come temporanee mancanze;
    \item Garantire un certo livello di sicurezza rispetto ai malfunzionamenti;
    \item Presentare solo il minimo intervallo possibile di indisponibilità durante i periodi di manutenzione;
    \item Impiegare efficientemente le risorse;
    \item Garantire la riusabilità delle sue parti anche in altri applicativi;
    \item Presentare componenti semplici e con basso livello di accoppiamento.
\end{itemize}

\subparagraph{Compiti}
\label{par:progettazione:compiti}
Questa attività è scomposta dai seguenti compiti:
\begin{itemize}
    \item \textbf{Trasformazione dei requisiti individuati in un'architettura che descriva ed identifichi ad alto livello le componenti del software}: sarà necessario prima di procedere con l'attività di codifica stabilire a priori quali componenti software devono essere presenti, in modo che vengano soddisfatte le aspettative;
    \item \textbf{Definizione ad alto livello delle interfacce tra le componenti del software}: per poter procedere in maniera organizzata con l'attività di codifica bisogna definire il comportamento delle varie componenti software;
    \item \textbf{Stendere la manualistica necessaria per descrivere le componenti software}: per poter garantire manutenibilità e semplificare il funzionamento del software sarà necessario redigere della manualistica specializzata per gli sviluppatori che illustri il funzionamento del software.
    \item \textbf{Stendere la manualistica necessaria per descrivere il funzionamento del software all'utente}: per poter garantire un semplice approccio dell'utente al software bisognerà redigere della manualistica che illustri all'utente il funzionamento del software.
    \item \textbf{definire e documentare i test necessari per l'integrazione del software}: stabilire un insieme di test che dovranno essere eseguiti al fine di verificare il funzionamento atteso del software.
\end{itemize}

In particolare, vengono normati i seguenti compiti, in modo da dare una linea guida da seguire per svolgerli.

%\subparagraph{Design pattern}
%La scelta dei \glossario{design pattern} da utilizzare è lasciata ai \emph{progettisti}, i quali dovranno
%assicurarsi che le loro scelte portino a una soluzione che sia flessibile e lasci una certa libertà ai
%\emph{programmatori}. Ogni design pattern utilizzato andrà spiegato e rappresentato in modo da poterne esporre
%significato e struttura.

\subparagraph{Diagrammi UML}
\label{par:diag-Uml}

Il gruppo ha scelto di utilizzare diagrammi UML allo scopo di rendere più chiare le scelte compiute in ambito di
progettazione. Questi verranno realizzati dai \emph{Progettisti}, in particolare verranno realizzati i seguenti diagrammi:
\begin{itemize}
	\item \textbf{Diagrammi di classe:} definiscono le caratteristiche statiche di un sistema, in particolare classi, tipi, metodi, attributi e le relazioni che vi intercorrono.
	\item \textbf{Diagrammi di package:} permettono la definizione dall'alto dell'architettura, infatti rappresentano raggruppamenti di classi e unità.
	%\item \textbf{Diagrammi di sequenza:} illustrano le azioni che avvengono tra oggetti/classi, ad esempio quanto vengono invocati metodi.
	\item \textbf{Diagrammi dei casi d'uso:} rappresentano le funzionalità del sistema attraverso una visione esterna al sistema; è un insieme di scenari e sequenze di azioni che hanno in comune il medesimo obiettivo per l'utente. 
\end{itemize}
Buona norma sarà:
\begin{itemize}
	\item evitare package vuoti, classi e parametri non utilizzati e parametri senza tipo;
	\item evitare la presenza di dipendenze circolari;
	\item utilizzare ove possibile interfacce e classi astratte;
	\item utilizzare nomi significativi per le classi in modo da facilitarne la comprensione;
\end{itemize}

\subparagraph{Test}

Come specificato precedentemente, la definizione dei test è parte integrante dell'attività di progettazione in quanto ne garantisce le caratteristiche qualitative, quindi ogni \emph{progettista} dovrà affiancare alla progettazione del sistema la definizione dei relativi test.


\paragraph{Codifica}
\label{ssub:codifica}
\subparagraph{Scopo}
\label{par:codifica:scopo}
L'attività di codifica viene svolta da parte dei \emph{Programmatori} ed ha il compito di tradurre l'architettura logica, mantenendo gli
standard qualitativi imposti e descritti all'interno del \textsc{Piano di Qualifica}, nel prodotto software. Di seguito vengono presentate
le regole e le norme che il gruppo pone sull'attività al fine di garantire la leggibilità del codice e di agevolarne manutenzione,
verifica e validazione.

\subparagraph{Compiti}
\label{par:codifica:Compiti}
Questa attività è scomposta dai seguenti compiti:
\begin{itemize}
    \item \textbf{Sviluppo delle unità software}: le unità definite dalla progettazione del software dovranno essere tradotte in codice;
    \item \textbf{Testing delle unità del software}: Le unità tradotte in codice dovranno essere testate per verificarne il comportamento.
    \item \textbf{Aggiornare i requisiti e la documentazione relativa}: i requisiti potrebbero cambiare durante lo sviluppo. Se la modifica dei requisiti dovesse portare a modifiche dell'architettura o del funzionamento dell'applicativo, queste modifiche dovranno essere estese anche alla relativa documentazione.
    \item \textbf{tenere traccia delle metriche di qualità previste}: durante lo sviluppo bisognerà tenere traccia e considerare criteri come:
        \begin{itemize}
            \item tracciabilità dei requisiti;
            \item copertura dei requisiti;
            \item conformità ai risultati aspettati;
            \item conformità della copertura dei test.
        \end{itemize}
    \item \textbf{Integrare il software con le necessarie modifiche effettuate}: lo sviluppo verrà effettuato mediante successive iterazioni. Sarà necessario quindi, che ogni iterazione sia integrata con il software preesistente.
\end{itemize}

\subparagraph{Regole generali di codifica}
\label{par:stile_codifica}

Al fine di garantire uniformità nel codice prodotto, si è deciso di stabilire delle regole nella scrittura del codice; in generale, a
prescindere dal tipo di file sorgente, valgono le seguenti norme:
\begin{itemize}
	\item Encoding: tutti i file sorgenti devono essere in codifica \glossario{UTF-8};
    \item Indentazioni: i blocchi di codice annidati, commenti esclusi, devono presentare 4 spazi di rientro rispetto al livello precedente;
    \item Righe: ciascuna riga può contenere al massimo 140 caratteri;
    \item Wrapping: nel caso una riga contenga più di 140 caratteri essa deve essere suddivisa su più linee di modo che rispettino tutte
    	la lunghezza massima e le righe aggiunte devono essere indentate di due livelli rispetto alla riga originale;
    \item Lingua: la lingua utilizzata per i nomi delle variabili e dei metodi deve essere l'inglese.
\end{itemize}



\subparagraph{Linee guida JavaScript}
\label{par:convenzioni_javascript}

Di seguito vengono specificate le linee guida per la codifica di codice JavaScript:
\begin{itemize}
	\item \textbf{Intestazione:} \\
		ciascun file sorgente JavaScript deve contenere la seguente intestazione specificando gli opportuni valori
		\begin{lstlisting}[style=htmlcssjs]
			/**
	 	 	 * @file <Descrizione del file>
	 	 	 *
	 	 	 * @file           <Descrizione del file>
	 	 	 *
	 	 	 * Data Creazione: <Data di creazione>
	 	 	 *
	 	 	 * @version        <Versione prodotto>
	 	 	 * @author         SpaghettiCode
	 	 	 * @author         <Nome Cognome>
	 	 	 * [@author        <Nome Cognome>]
	 	 	 */
		\end{lstlisting}
		seguendo le norme in \refSec{par:date} e \refSec{par:codice_versione} rispettivamente per il formato della data e il codice di versione;
	\item \textbf{Intestazione funzioni:} \\
		ciascuna funzione e/o metodo dev'essere preceduta dalla seguente intestazione specificando opportunamente i valori
		\begin{lstlisting}[style=htmlcssjs]
			/**
			 * <Descrizione della funzione>
			 *
			 * [@param {<tipo parametro>} <Nome parametro> - <Descrizione parametro>]
			 */
		\end{lstlisting}
\end{itemize}

\subparagraph{Convenzioni sui nomi}

Ciascun elemento deve avere un nome espressivo che esprima chiaramente che cosa rappresenta, in modo da migliorare la leggibilità e
semplificare la manutenzione.

Di seguito vengono elencate le regole per l'assegnazione dei nomi di diversi componenti.
\begin{itemize}
	\item \textbf{Classi}:\\
		i nomi di classi, interfacce e record devono seguire la convenzione \glossario{PascalCase} \\
        \begin{lstlisting}[style=htmlcssjs]
			class Foo {
			...
			}
		\end{lstlisting}

    \item \textbf{Metodi}:\\
    	i nomi dei metodi devono seguire la convenzione \glossario{CamelCase}, inoltre i metodi che ritornano valori booleani devono avere il
    	nome che comincia per \texttt{is} e deve essere indicativo del loro comportamento\\
		\begin{lstlisting}[style=htmlcssjs]
			class Foo {
				fooMethod() {
				...
				}

				isTrue() {
					return true;
				}
			}
		\end{lstlisting}

	\item \textbf{Costanti}:\\
		i nomi delle costanti, ossia degli elementi che non possono cambiare stato, seguono la convenzione
		\glossario{CONSTANT\_CASE} \\
		\begin{lstlisting}[style=htmlcssjs]
			const MEANING_OF_LIFE = 42;
		\end{lstlisting}

    \item \textbf{Variabili}:\\
		i nomi delle variabili devono seguire la convenzione CamelCase \\
		\begin{lstlisting}[style=htmlcssjs]
			let correctVariableName = true;
			let correct = true;
		\end{lstlisting}

	\item \textbf{Funzioni}:\\
		i nomi delle funzioni devono seguire la convenzione CamelCase \\

		\begin{lstlisting}[style=htmlcssjs]
			function fooBar() {
			...
			}
		\end{lstlisting}
	\item \textbf{Parametri}:\\
		i nomi dei parametri devono seguire la convenzione CamelCase \\

		\begin{lstlisting}[style=htmlcssjs]
			function sum(firstNumber, secondNumber) {

			}
		\end{lstlisting}

	\end{itemize}

\subparagraph{Formattazione}

La formattazione del codice deve rendere immediata l'identificazione degli elementi e di conseguenza dev'essere incentrata sull'espressività
rispetto alla sinteticità.
Di seguito vengono elencate le regole per quanto riguarda la formattazione del codice:
\begin{itemize}
	\item \textbf{Parentesi}:\\
		tutte le strutture di controllo (\texttt{if}, \texttt{else}, \texttt{for}, \texttt{do}, \texttt{while}), anche se contenenti
		un'unica istruzione, devono essere provviste delle parentesi graffe.\\

		L'unica eccezione si ha con il costrutto \texttt{if} definito \emph{inline}\\

		\begin{lstlisting}[style=htmlcssjs]
			if (isTrue()) {
				foo();
				bar();
			} else {
				fooBar();
			}

			if (isTrue()) fooBar();
		\end{lstlisting}

	\item \textbf{Blocchi}:\\
		per quanto riguarda i blocchi di codice l'apertura di un blocco non dev'essere preceduta da un carattere di interruzione di linea
		mentre si deve andare a capo dopo l'apertura del blocco e prima della chiusura di un blocco;
		\begin{lstlisting}[style=htmlcssjs]
			function isRightBlock() {
				return true;
			}
		\end{lstlisting}

	\item \textbf{Istruzioni}:\\
		ciascuna istruzione dev'essere terminata dal punto e virgola e deve essere seguita da un'interruzione di linea.

	\item \textbf{Spazi bianchi}:\\
		nelle assegnazioni e all'interno delle espressioni si devono inserire spazi tra gli elementi e gli operatori.
		\begin{lstlisting}[style=htmlcssjs]
			let expression = firstNumber + secondNumber * secondNumber;
		\end{lstlisting}
\end{itemize}

\subparagraph{Funzionalità del linguaggio}

Alcune funzionalità peculiari del linguaggio sono preferibili ad altre nella codifica di software di qualità, pertanto di seguito vengono
specificati i costrutti del linguaggio consentiti:
\begin{itemize}
	\item \textbf{Variabili locali}:\\
		per la definizione delle variabili vengono utilizzati i costrutti \texttt{const} e \texttt{let}; il costrutto \texttt{var} va
		evitato in modo da poter gestire in maniera migliore lo \emph{scope} delle variabili;

	\item \textbf{Classi}:\\
		per comodità di verifica viene preferito l'uso del costrutto \texttt{class} rispetto alla definizione di classi anonime \\
		\begin{lstlisting}[style=htmlcssjs]
			class MyClass {
			...
			}
		\end{lstlisting}

	\item \textbf{Funzioni}:\\
		per comodità di verifica viene preferita la definizione di funzioni esplicite rispetto alla definizione di funzioni anonime \\
		\begin{lstlisting}[style=htmlcssjs]
			function myFunction(foo, bar) {
			...
			}
		\end{lstlisting}

	\item \textbf{Funzioni anonime}:\\
		qualora fosse necessario implementare funzioni anonime viene preferito l'utilizzo dell'operatore \emph{freccia}
		\begin{lstlisting}[style=htmlcssjs]
			(foo, bar) => {
			...
			}
		\end{lstlisting}


	\item \textbf{Letterali}:\\
		nell'instanziazione di \texttt{Object} e/o \texttt{Array} preferire l'utilizzo dei letterali rispetto ai costruttori
		\begin{lstlisting}[style=htmlcssjs]
			let myArray = [];
			let myObject = {};
		\end{lstlisting}

	\item \textbf{Operatore di espansione}:\\
		per riferirsi agli elementi di un array preferire l'utilizzo dell'operatore di espansione rispetto a costrutti più onerosi;

	\item \textbf{Confronti}:\\
		nel caso di confronti utilizzare gli operatori di identità (\texttt{===} e \texttt{!==}) anziché quelli di confronto.
\end{itemize}

\subparagraph{Linee guida HTML}
\label{par:convenzioni_html}

Di seguito vengono elencate le regole che dovranno essere adottate nello sviluppo di file HTML:
\begin{itemize}
	\item \textbf{Dichiarazione DOCTYPE}:\\
		la prima linea di ciascun file deve contenere la dichiarazione del tipo di documento come segue\\
		\begin{lstlisting}[style=html]
			<!DOCTYPE html>
		\end{lstlisting}
	\item \textbf{Dichiarazione encoding}:\\
		all'interno della sezione head dev'essere presente un tag meta che specifichi l'encoding del file
		\begin{lstlisting}[style=html]
			<meta charset="utf-8">
		\end{lstlisting}
	\item \textbf{Tag}:\\
		i tag ed i nomi degli attributi devono essere esclusivamente in minuscolo;
	\item \textbf{Attributi}:\\
		i valori degli attributi vanno racchiusi all'interno dei doppi apici
		\begin{lstlisting}[style=html]
			<a class="class-value"></a>
		\end{lstlisting}
	\item \textbf{Id e classi}:\\
		i valori degli id e delle classi devono privilegiare l'espressività e seguire la convenzione \glossario{kebab-case}
		\begin{lstlisting}[style=html]
			<a id="id-value" class="class-value">Foo</a>
		\end{lstlisting}
\end{itemize}

\subparagraph{Strumenti}

Sono riportati di seguito gli strumenti utilizzati nel processo di sviluppo:
\begin{itemize}
	\item \textbf{Draw.io}: applicazione web molto versatile per disegnare grafici UML che offre la possibilità di lavorare
		contemporaneamente;
	\item \textbf{HTML}: linguaggio utilizzato per lo sviluppo di pagine web utilizzato per specificare la struttura della pagina;
	\item \textbf{CSS}: linguaggio utilizzato nello sviluppo di pagine web per gestire e personalizzare la presentazione;
	\item \textbf{JavaScript}: linguaggio di scripting per web application richiesto dal proponente che fornisce la possibilità di
		utilizzare la libreria D3.js;
	\item \textbf{D3.js}: libreria open source scritta in \emph{JavaScript} con lo scopo di facilitare la visualizzazione dei dati in
		grafici, è lo strumento principale per la realizzazione del prodotto HD Viz.
\end{itemize}

\section{Introduzione}
\label{sec:introduzione}

\subsection{Scopo del documento}
% il \glossario{way of working} 
Lo scopo di questo documento è quello di fissare le convenzioni e regole che ciascun membro del gruppo 
\emph{SpaghettiCode} dovrà rispettare al fine di sviluppare efficientemente ed efficacemente il prodotto richiesto. % per tutto lo svolgimento del \glossario{progetto}. 
L'aggregazione delle varie \glossario{attività} descritte all'interno del documento genera diversi 
\glossario{processi}, ispirati allo standard \textsc{ISO/IEC 12207:1995}, che permetteranno ai membri del team di effettuare transizioni di stato del progetto il più economicamente possibile. 

Il documento segue uno sviluppo di tipo incrementale e perciò sarà sottoposto continuamente a modifiche (rimozioni o aggiunte). Al completamento di 
una qualsiasi di queste azioni, tutti i membri del gruppo dovranno essere notificati, in modo da poter adeguarsi alle norme il più aggiornate 
possibili. Il documento è, infatti, ancora incompleto e verrà aggiornato riguardo alle regole specifiche per ogni processo del \glossario{progetto}.

\subsection{Scopo del prodotto}
Il \glossario{capitolato} \emph{HD Viz} si pone come obiettivo la realizzazione di una \glossario{web application} che 
permetta la visualizzazione di dati pluridimensionali mediante diverse tipologie di grafici, i quali permettono 
all'utente di individuare strutture e/o modelli intrinseci dei dati stessi, nella fase di esplorazione dei dati 
(\glossario{EDA}).

\subsection{Glossario}
Al fine di eliminare qualsiasi equivocità nei termini presenti all'interno del documento e quindi dell'insorgere di 
incomprensioni, viene fornito il documento \textsc{Glossario v1.0.0}, nel quale vengono riportate le definizioni 
specifiche dei termini che presentano una "G" a pedice.

\subsection{Riferimenti}
\subsubsection{Riferimenti normativi}
\begin{itemize}
    \item \textbf{Capitolato d'appalto \textsc{C4 - HD Viz}}: \\
    \url{https://www.math.unipd.it/~tullio/IS-1/2020/Progetto/C4.pdf};
    \item \textbf{Ulteriori informazioni sul capitolato C4}: \\
    \url{https://www.dropbox.com/s/nslvtrq2wcycoqw/HD\%20Viz.mp4?dl=0}.
\end{itemize}

\subsubsection{Riferimenti informativi}
\begin{itemize}
    \item \textbf{Standard ISO/IEC 12207:1995}: \\
    \url{https://www.math.unipd.it/~tullio/IS-1/2009/Approfondimenti/ISO_12207-1995.pdf};
    \item \textbf{Documentazione \glossario{Git}}: \\
    \url{https://git-scm.com/docs};
    \item \textbf{Documentazione \glossario{GitHub}}: \\
    \url{https://docs.github.com/en/free-pro-team@latest};
    \item \textbf{Documentazione \glossario{\LaTeX}}: \\
    \url{https://www.latex-project.org/help/documentation/};
    \item \textbf{Documentazione \glossario{JavaScript}}: \\
    \url{https://developer.mozilla.org/en-US/docs/Web/JavaScript};
    \item \textbf{Documentazione libreria \glossario{3D.js}}: \\
    \url{https://github.com/d3/d3/wiki}.

\end{itemize}
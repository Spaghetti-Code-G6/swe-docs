\section{Processi organizzativi}
\label{sec:processi_organizativi}

\subsection{Processi di coordinamento}

\subsubsection{Scopo}
In questo paragrafo vengono raccolte le modalità di coordinamento adottate dal gruppo per le comunicazioni e gli 
incontri con i vari soggetti coinvolti durante il ciclo di vita del progetto. Si svolge questa attività per avere 
dei canali d'interazione comuni e non escludere nessuno o rischiare di perdere informazioni importanti. Rispetto 
agli anni passati, questo processo assume un ruolo di assoluta importanza, date le circostanze di questo periodo 
nel quale è impossibile incontrarsi di persona.

\subsubsection{Comunicazione}
Durante lo svolgimento di questo progetto verranno usati più strumenti per la comunicazione a seconda del soggetto 
interessato. I soggetti divisi per ruolo sono i seguenti:
\begin{itemize}
    \item Proponente: \emph{Piccoli Gregorio} rappresentante dell'azienda \emph{Zucchetti S.p.A.};
    \item Committenti: prof. \emph{Vardanega Tullio} e prof. \emph{Cardin Riccardo};
    \item \emph{SpaghettiCode}: tutti i membri del gruppo;
    \item Altri gruppi: i gruppi candidati allo stesso capitolato sono Gruppo 5, 10 e 15.
\end{itemize}
Tutte le comunicazioni saranno svolte per via telematica, data l'impossibilità di incontrarsi di persona.

\paragraph{Interna}

I membri del gruppo si sono accordati per usare Discord come canale principale per le comunicazioni e Telegram 
come canale secondario. Su Discord sono stati creati dei canali appositi divisi a seconda delle tematiche che 
vengono trattate (es. generale, note-risorse, incontri, gestione-gruppo), inoltre ogni ruolo ha a disposizione 
un canale apposito nel quale poter discutere di argomenti inerenti alle attività in corso. Il gruppo di 
Telegram, invece, viene usato per comunicazioni informali e più rapide.

\paragraph{Esterna}

Le comunicazioni con i committenti avverranno tramite video-chiamate con Zoom e messaggi di posta elettronica; 
quelle con il proponente avverranno tramite video-chiamate su Zoom, Skype e messaggi di posta elettronica; infine 
le comunicazioni con gli altri gruppi dello stesso capitolato avverranno tramite un gruppo Telegram appositamente 
creato.

\paragraph{Riunioni}

Tutte le riunioni saranno svolte tramite i canali di comunicazione scelti per il soggetto interessato. 

\paragraph{Strumenti}

\begin{itemize}
    \item Discord: applicazione VoIP e di messaggistica istantanea, facile da usare e versatile;
    \item Telegram: servizio di messaggistica istantanea;
    \item Zoom: servizio di teleconferenze;
    \item Skype: software di messaggistica istantanea e VoIP.
\end{itemize}

\subsection{Pianificazione}

\subsubsection{Scopo}

Nei prossimi paragrafi verrà descritto come il gruppo \emph{SpaghettiCode} intende pianificare il lavoro; verranno 
elencati i ruoli con i loro relativi compiti e l'assegnazione di essi tra i membri del gruppo. Prendendo come esempio 
lo standard ISO/IEC 12207, il processo di pianificazione sarà strutturato nel seguente modo:
\begin{itemize}
    \item Ruoli con relativi compiti;
    \item Assegnazione ruoli;
    \item Divisione del lavoro.
\end{itemize}

\subsubsection{Ruoli di progetto}

\paragraph{Responsabile}

Il \textbf{responsabile} rappresenta il progetto ed è il punto di riferimento per le comunicazioni con il committente. 
Per poter pianificare ed anticipare l'evoluzione del progetto deve possedere capacità tecniche e delle competenze 
pregresse, deve essere in grado di gestire le risorse e tracciare i progressi. Ha la responsabilità di scelta e 
approvazione su gran parte del progetto e partecipa per tutta la durata di esso.\\
In particolare, ha il compito di:
\begin{itemize}
    \item Redigere l'organigramma;
    \item Redigere il \textsc{Piano di Progetto};
    \item Coordinare i membri del gruppo, le attività e le risorse a disposizione;
    \item Gestire le criticità;
    \item Approvare i documenti;
    \item Approvare l'offerta del committente.
\end{itemize}

\paragraph{Amministratore}

L'\textbf{\glossario{amministratore}} è il responsabile dell'efficienza e dell'operatività dell'ambiente di lavoro, 
deve assicurarsi che le risorse siano sempre presenti e operanti.\\
In particolare, ha i seguenti compiti:
\begin{itemize}
    \item Gestire il controllo della configurazione del prodotto;
    \item Gestire il versionamento;
    \item Gestire la documentazione del progetto;
    \item Redigere le \textsc{Norme di Progetto};
    \item Collaborare alla redazione del \textsc{Piano di Progetto};
    \item Redigere ed attuare i piani e le procedure di \emph{Gestione della Qualità};
    \item Risolvere i problemi legati alla gestione dei processi.
\end{itemize}
    
\paragraph{Analista}

L'\textbf{\glossario{analista}} ha una notevole esperienza professionale e vasta conoscenza del dominio del problema. 
Si occupa di esporre il problema in maniera chiara con un linguaggio simile a quello usato dal proponente. Il lavoro 
dell'analista ha un grande impatto sulla riuscita del progetto, pertanto è preferibile che il ruolo venga svolto 
contemporaneamente da più di uno. 
Generalmente sono pochi e non seguono il progetto fino al suo completamento.\\
In particolare, ha i seguenti compiti:
\begin{itemize} 
    \item Redigere lo \textsc{Studio di Fattibilità};
    \item Redigere l'\textsc{Analisi dei Requisiti}.
\end{itemize}

\paragraph{Progettista}

Il \textbf{\glossario{progettista}} è una persona con competenze tecniche e tecnologiche avanzate e con un'ampia 
esperienza professionale. Si occupa dello sviluppo della soluzione al problema presentato tramite l'attività di 
progettazione, spesso assumendo anche responsabilità di scelta e gestione. Possono esserci contemporaneamente più 
progettisti, pur sempre in numero contenuto, ed essi seguono il progetto fino alla manutenzione.\\
In particolare, ha i seguenti compiti:
\begin{itemize}
    % TODO: Technology Baseline e Product Baseline?
    \item Redigere la Specifica Tecnica;
    \item Redigere la Definizione di Prodotto;
    \item Redigere la parte programmatica del \textsc{Piano di Qualifica}.
\end{itemize}

\paragraph{Programmatore}

Il \textbf{\glossario{programmatore}} ha competenze tecniche specifiche, ma responsabilità limitate, si occupa di 
implementare la soluzione trovata dal \textbf{progettista} tramite attività di codifica del prodotto e dei test di 
ausilio alla verifica. Partecipa a lungo all'interno del progetto, contribuendo anche alla manutenzione.\\
In particolare, ha i seguenti compiti:
\begin{itemize}
    % TODO: Technology Baseline?
    \item Implementare la Specifica Tecnica tramite codifica;
    \item Implementare i test d'ausilio necessari per l'esecuzione delle prove di verifica e validazione.
\end{itemize}

\paragraph{Verificatore}

Il \textbf{\glossario{verificatore}} ha competenze tecniche, esperienze di progetto e conoscenza delle norme, oltre che 
capacità di giudizio e relazione. Si occupa di attività di verifica e validazione, partecipa all'intero ciclo di vita 
del progetto assicurandosi che quanto fatto sia conforme alle attese. Illustra nel \textsc{Piano di Qualifica} l'esito 
e la completezza delle verifiche e delle prove effettuate.\\
In particolare, ha i seguenti compiti:
\begin{itemize}
    \item Esaminare i prodotti in fase di revisione tramite le tecniche e gli strumenti descritti nelle \textsc{Norme di Progetto};
    \item Segnalare eventuali errori o modifiche necessari ai diretti interessati, in modo che possano correggerli.
\end{itemize}

\subsubsection{Assegnazione dei ruoli}

I diversi ruoli verranno assegnati mediante rotazione a ciascun raggiungimento di una scadenza terminale. Ogni membro 
dovrà ricoprire più ruoli durante l'intero ciclo di vita del progetto, per un periodo significativo, abbastanza lungo 
da non interrompere la continuità delle attività in corso. Inizialmente i ruoli sono stati assegnati casualmente, 
perché nessuno dei membri del gruppo aveva conoscenze pregresse. Si prevede di fare scelte più mirate nella prossima 
rotazione, tenendo in considerazione le conoscenze acquisite nell'ultimo periodo e gli interessi sviluppati da parte 
dei membri verso le tecnologie usate.

\subsubsection{Assegnazione dei compiti}

Ogni membro dovrà svolgere i suoi compiti in base al ruolo assegnatogli. Per tenere traccia di quanto è stato fatto e 
di quanto resta ancora da fare si è deciso di usare il sistema di \emph{Issue Tracking} offerto da \emph{GitHub}. Ogni 
membro potrà creare delle \emph{Issue} relative al compito da svolgere, dovranno essere piccoli task che potranno 
essere svolti anche da una sola persona. Se il lavoro da svolgere non è strettamente legato ad un ruolo preciso qualsiasi membro potrà prendersene carico previa prenotazione.

\subsubsection{Metriche}

Durante i meeting tra i membri del gruppo ci si dovrà accordare con delle scadenze e verranno fissate delle 
\glossario{Milestone} su \emph{GitHub}. Tramite le \emph{Milestone} si potrà conoscere l'andamento dei lavori e la 
percentuale di completamento, infatti, ogni \emph{Issue} creata dovrà essere assegnata ad una Milestone. 

\subsubsection{Gestione dei rischi}

Per gestire i rischi, il gruppo si è accordato nell'usare un sistema di tag tramite \emph{Label} delle \emph{Issue}. 
Alla creazione di una nuova \emph{Issue} sarà possibile assegnare un livello di priorità, che andrà da minore, normale, 
importante e critica a seconda dell'urgenza con la quale deve essere svolto il relativo compito. In questo modo ogni 
membro del gruppo potrà vedere se ci sono attività che necessitano di maggior attenzione e/o risorse rispetto ad altre 
e potrà intervenire, assegnandosi la \emph{Issue} oppure sollecitando i diretti interessati.

\subsection{Formazione}

\subsubsection{Scopo}

Lo scopo della formazione è quello di uniformare le capacità tecniche e le conoscenze tra i vari membri del gruppo, in 
modo da poter lavorare e comunicare in sintonia.

\subsubsection{Descrizione}

Per ogni membro di \emph{SpaghettiCode} è prevista la formazione tramite studio autonomo delle varie tecnologie che 
vengono adoperate o che sono state richieste da \emph{Zucchetti S.p.A.} durante la presentazione del capitolato e 
durante gli incontri successivi. In caso di difficoltà il gruppo è disponibile a fare formazione tramite incontri nei 
canali di comunicazione ufficiali.

\subsubsection{Aspettative}

Ci si aspetta che tutti i membri del gruppo acquisiscano familiarità con le seguenti tecnologie:
\begin{itemize}
    \item \LaTeX;
    \item Git e GitHub;
    \item JavaScript;
    \item Libreria D3.js.
\end{itemize}
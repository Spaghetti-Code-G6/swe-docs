\section{Processi organizzativi}
\label{sec:processi_organizativi}

\subsection{Processo di coordinamento}

\subsubsection{Descrizione}
Il processo di coordinamento aggrega tutte le diverse attività che il gruppo \emph{SpaghettiCode} dovrà svolgere al fine di gestire e controllare l'andamento del prodotto e dei processi attivi.\\

Il processo di coordinamento adottato dal gruppo \emph{SpaghettiCode} consiste nelle seguenti attività definite dallo standard di processo \emph{ISO/IEC 12207:1997}, opportunamente adattate alle necessità del progetto didattico:
\begin{itemize}
    \item \textbf{Implementazione};
    \item \textbf{Pianificazione};
    \item \textbf{Esecuzione e controllo};
    \item \textbf{Revisione e valutazione};
    \item \textbf{Consegna e completamento}.
\end{itemize}

\subsubsection{Attività}


\paragraph{Implementazione}
\label{par:Implementazione}
\subparagraph{Scopo}
\label{par:Implementazione:scopo}
In questa attività, si stabiliscono i requisiti necessari allo svolgimento dei processi ed il \emph{manager} si incarica di far sì che i processi possano essere svolti in maniera adeguata, appropriata e rispettosa delle scadenze.\\
\subparagraph{Compiti}
\label{par:Implementazione:compiti}
Questa attività è composta dai seguenti compiti:
\begin{itemize}
    \item \textbf{Definizione dei requisiti}: si definiscono i requisiti necessari per svolgere i diversi processi;
    \item \textbf{Controllo delle risorse}: il \emph{manager} controlla che le risorse (personale, materiali, ambiente e tecnologie) siano adeguate allo scopo;
    \item \textbf{Aggiustamento dei criteri}: se alcuni requisiti individuati non dovessero risultare adeguati, si modificano in comune accordo tra le varie parti coinvolte.
\end{itemize}

\paragraph{Pianificazione}
\label{par:pianificazione}
\subparagraph{Scopo}
\label{par:pianificazione:scopo}
In questa attività il \emph{manager} stabilisce come pianificare l'esecuzione dei processi.\\
\subparagraph{Compiti}
\label{par:pianificazione:compiti}
Questa attività è composta dai seguenti compiti:
\begin{itemize}
    \item \textbf{Pianificazione}: si descrive e pianifica l'esecuzione del processo, tenendo conto di:
    \begin{itemize}
        \item Stima temporale;
        \item Complessità;
        \item Assegnazione delle responsabilità;
        \item Quantificazione dei rischi.
    \end{itemize}
\end{itemize}

\paragraph{Esecuzione e controllo}
\label{par:esecuzione_e_controllo}
\subparagraph{Scopo}
\label{par:esecuzione_e_controllo:scopo}
In questa attività il \emph{manager} mette in pratica quanto definito nell'attività di pianificazione.
\subparagraph{Compiti}
\label{par:esecuzione_e_controllo:compiti}
Questa attività è scomposta nei seguenti compiti:
\begin{itemize}
    \item \textbf{Attuazione della pianificazione}: il \emph{manager} si occupa di applicare la pianificazione;
    \item \textbf{Monitoraggio dell'andamento del progetto}: sarà compito del \emph{manager} tener conto dell'andamento dei processi;
    \item \textbf{Applicazione delle modalità di risoluzione dei rischi riscontrati}: in caso si verifichino i rischi previsti o se ne incontrino di nuovi, è compito del \emph{manager} attuare le strategie risolutive più adeguate ed eventualmente aggiornarle.
\end{itemize}

\paragraph{Revisione e valutazione}
\label{par:revisione_e_valutazione}
\subparagraph{Scopo}
\label{par:revisione_e_valutazione:scopo}
Questa attività ha lo scopo di assicurare che tutti i prodotti ed i risultati dei compiti e delle attività rispettino i requisiti di qualità e di completezza.\\
\subparagraph{Compiti}
\label{par:revisione_e_valutazione:Compiti}
Questa attività è scomposta nei seguenti compiti:
\begin{itemize}
    \item \textbf{Verifica dei prodotti}: il \emph{manager} valuta che i prodotti software ed i documenti rispettino i requisiti richiesti;
    \item \textbf{Verifica dei risultati}: il \emph{manager} valuta i risultati ottenuti al completamento dell'esecuzione di un'attività o di un compito.
\end{itemize}

\paragraph{Completamento}
\label{par:completamento}
\subparagraph{Scopo}
\label{par:completamento:scopo}
Questa attività ha lo scopo di determinare se e quando un processo, un compito, o un'attività è da considerarsi terminata.\\
\subparagraph{Compiti}
\label{par:completamento:compiti}
Questa attività è composta dal seguente compito:
\begin{itemize}
	\item \textbf{Verifica del completamento}: al termine di un processo, di un compito, o una attività, il \emph{manager} si assume la responsabilità di determinare se quel processo, compito o attività è concluso rispetto ai criteri stabiliti dal gruppo \emph{SpaghettiCode};
\end{itemize}


In questo paragrafo vengono raccolte le modalità di coordinamento adottate dal gruppo per le comunicazioni e gli
incontri con i vari soggetti coinvolti durante il ciclo di vita del progetto. Si svolge questa attività per avere
dei canali d'interazione comuni e non escludere nessuno o rischiare di perdere informazioni importanti. Rispetto
agli anni passati, questo processo assume un ruolo di assoluta importanza, date le circostanze di questo periodo
nel quale è impossibile incontrarsi di persona.

\subsubsection{Comunicazione}
Durante lo svolgimento di questo progetto verranno usati più strumenti per la comunicazione a seconda del soggetto
interessato. I soggetti divisi per ruolo sono i seguenti:
\begin{itemize}
    \item Proponente: \emph{Piccoli Gregorio} rappresentante dell'azienda \emph{Zucchetti S.p.A.};
    \item Committenti: prof. \emph{Vardanega Tullio} e prof. \emph{Cardin Riccardo};
    \item \emph{SpaghettiCode}: tutti i membri del gruppo;
    \item Altri gruppi: i gruppi candidati allo stesso capitolato sono Gruppo 5, 10 e 15.
\end{itemize}
Tutte le comunicazioni saranno svolte per via telematica, data l'impossibilità di incontrarsi di persona.

\paragraph{Interna}

I membri del gruppo si sono accordati per usare Discord come canale principale per le comunicazioni e Telegram
come canale secondario. Su Discord sono stati creati dei canali appositi divisi a seconda delle tematiche che
vengono trattate (es. generale, note-risorse, incontri, gestione-gruppo), inoltre ogni ruolo ha a disposizione
un canale apposito nel quale poter discutere di argomenti inerenti alle attività in corso. Il gruppo di
Telegram, invece, viene usato per comunicazioni informali e più rapide.

\paragraph{Esterna}

Le comunicazioni con i committenti avverranno tramite video-chiamate con Zoom e messaggi di posta elettronica;
quelle con il proponente avverranno tramite video-chiamate su Zoom, Skype e messaggi di posta elettronica; infine
le comunicazioni con gli altri gruppi dello stesso capitolato avverranno tramite un gruppo Telegram appositamente
creato.

\paragraph{Strumenti}

\begin{itemize}
    \item Discord: applicazione VoIP e di messaggistica istantanea, facile da usare e versatile;
    \item Telegram: servizio di messaggistica istantanea;
    \item Zoom: servizio di teleconferenze;
    \item Skype: software di messaggistica istantanea e VoIP.
\end{itemize}

\subsubsection{Ruoli di progetto}

\paragraph{Responsabile}

Il \emph{responsabile} rappresenta il progetto ed è il punto di riferimento per le comunicazioni con il committente.
Per poter pianificare ed anticipare l'evoluzione del progetto deve possedere capacità tecniche e competenze
pregresse, inoltre deve essere in grado di gestire le risorse e tracciare i progressi. Ha la responsabilità di scelta e
approvazione su gran parte del progetto e partecipa per tutta la durata di esso.\\
In particolare, ha il compito di:
\begin{itemize}
    \item Redigere l'organigramma;
    \item Redigere il \textsc{Piano di Progetto};
    \item Coordinare i membri del gruppo, le attività e le risorse a disposizione;
    \item Gestire le criticità;
    \item Approvare i documenti;
    \item Approvare l'offerta del committente.
\end{itemize}

\paragraph{Amministratore}

L'\emph{\glossario{amministratore}} è il responsabile dell'efficienza e dell'operatività dell'ambiente di lavoro e
deve assicurarsi che le risorse siano sempre presenti e operanti.\\
In particolare, ha i seguenti compiti:
\begin{itemize}
    \item Gestire il controllo della configurazione del prodotto;
    \item Gestire il versionamento;
    \item Gestire la documentazione del progetto;
    \item Redigere le \textsc{Norme di Progetto};
    \item Collaborare alla redazione del \textsc{Piano di Progetto};
    \item Redigere ed attuare i piani e le procedure di \emph{Gestione della Qualità};
    \item Risolvere i problemi legati alla gestione dei processi.
\end{itemize}

\paragraph{Analista}

L'\emph{\glossario{analista}} ha una notevole esperienza professionale e una vasta conoscenza del dominio del problema.
Si occupa di esporre il problema in maniera chiara con un linguaggio simile a quello usato dal proponente. Il lavoro
dell'analista ha un grande impatto sulla riuscita del progetto, pertanto è preferibile che il ruolo venga svolto
contemporaneamente da più di una persona.
Gli \emph{analisti} non seguono il progetto fino al suo completamento.\\
In particolare, ha i seguenti compiti:
\begin{itemize}
    \item Redigere lo \textsc{Studio di Fattibilità};
    \item Redigere l'\textsc{Analisi dei Requisiti}.
\end{itemize}

\paragraph{Progettista}

Il \emph{\glossario{progettista}} è una persona con competenze tecniche e tecnologiche avanzate e con un'ampia
esperienza professionale. Si occupa dello sviluppo della soluzione al problema presentato tramite l'attività di
progettazione, spesso assumendo anche responsabilità di scelta e gestione. Possono esserci contemporaneamente più
progettisti, pur sempre in numero contenuto, ed essi seguono il progetto fino alla manutenzione.\\
In particolare, ha i seguenti compiti:
\begin{itemize}
    \item Redigere la Specifica Tecnica;
    \item Redigere la Definizione di Prodotto;
    \item Redigere la parte programmatica del \textsc{Piano di Qualifica}.
\end{itemize}

\paragraph{Programmatore}

Il \emph{\glossario{programmatore}} ha competenze tecniche specifiche ma responsabilità limitate e si occupa di
implementare la soluzione trovata dal \emph{progettista} tramite attività di codifica del prodotto e dei test di
ausilio alla verifica. Partecipa a lungo all'interno del progetto, contribuendo anche alla manutenzione.\\
In particolare, ha i seguenti compiti:
\begin{itemize}
    \item Implementare la Specifica Tecnica tramite codifica;
    \item Implementare i test d'ausilio necessari per l'esecuzione delle prove di verifica e validazione.
\end{itemize}

\paragraph{Verificatore}

Il \emph{\glossario{verificatore}} ha competenze tecniche, esperienze di progetto e conoscenza delle norme, oltre che
capacità di giudizio e relazione. Si occupa di attività di verifica e validazione, partecipa all'intero ciclo di vita
del progetto assicurandosi che quanto fatto sia conforme alle attese. Illustra nel \textsc{Piano di Qualifica} l'esito
e la completezza delle verifiche e delle prove effettuate.\\
In particolare, ha i seguenti compiti:
\begin{itemize}
    \item Esaminare i prodotti in fase di revisione tramite le tecniche e gli strumenti descritti nelle \textsc{Norme di Progetto};
    \item Segnalare eventuali errori o modifiche necessarie ai diretti interessati, in modo che possano correggere.
\end{itemize}

\subsubsection{Assegnazione dei ruoli}

I diversi ruoli verranno assegnati mediante rotazione all'incirca ogni due settimane. Ogni membro
dovrà ricoprire più ruoli durante l'intero ciclo di vita del progetto, per un periodo significativo, abbastanza lungo
da non interrompere la continuità delle attività in corso. Inizialmente i ruoli sono stati assegnati casualmente,
perché nessuno dei membri del gruppo aveva conoscenze pregresse.

Dal momento in cui i membri del gruppo cominciano a conoscere i rispettivi punti di forza e debolezza, l'assegnazione dei ruoli risulta
più efficace e porta valore aggiunto al prodotto.

\subsubsection{Assegnazione dei compiti}

Ogni membro dovrà svolgere diversi compiti in base al ruolo assegnatogli. Per tenere traccia di quanto è stato fatto e
di quanto resta ancora da fare si è deciso di usare il sistema di \emph{Issue Tracking} offerto da \emph{GitHub}. Inizialmente sarà compito
degli \emph{amministratori} aggiungere i ticket nel sistema, marcandoli con una \emph{label} che indichi quale ruolo
dovrà prendersene carico e assegnare tali ticket alla persona incaricata. Successivamente ogni membro potrà creare delle \emph{Issue}
relative al compito da svolgere marcandole sempre opportunamente. I ticket dovranno rappresentare piccoli task che potranno essere svolti
da una sola persona in un meno di una settimana. Nel caso un compito non fosse strettamente legato ad un ruolo preciso, qualsiasi membro
potrà prendersene carico previa prenotazione.

\subsubsection{Assegnazione risorse}

Durante gli incontri tra i membri del gruppo in seguito al termine di un incremento si deve meglio definire l'incremento successivo,
assegnando le scadenze dei task da cui è composto e i membri che lo svolgeranno. Per rappresentare le scadenze temporali verranno fissate
delle \glossario{Milestone} su \emph{GitHub} e create delle \emph{Issue} per i task assegnate alle \emph{Milestone} precedentemente definite.

\subsubsection{Tracciamento dei progressi e delle risorse}
\label{ssub:pianificazione:tracciamento}
Per tenere traccia dei progressi del lavoro e dei compiti assegnati, viene utilizzato l'\emph{Issue Tracking System} e i
\emph{projects} di \emph{GitHub}. Le \emph{Issue} relative allo stesso incremento vengono poste nello stesso \emph{projects} organizzato a
modi \glossario{kanban board}; in questo modo ogni membro del gruppo, in base allo stato della singola \emph{Issue} o del \emph{project},
può conoscere rispettivamente lo stato di avanzamento del singolo task oppure dell'incremento. Ogni membro del gruppo si impegna quindi ad
aggiornare lo stato di avanzamento della issue in base al lavoro svolto e all'effettivo stato della stessa.

\subsubsection{Gestione dei rischi}

Per gestire i rischi, il gruppo si è accordato nell'usare un sistema di tag tramite \emph{label} delle \emph{Issue}.
Alla creazione di una nuova \emph{Issue} sarà possibile assegnare un livello di priorità, che andrà da minore, normale,
importante e critica a seconda dell'urgenza con la quale deve essere svolto il relativo compito. In questo modo ogni
membro del gruppo potrà vedere se ci sono attività che necessitano di maggior attenzione e/o risorse rispetto ad altre
e potrà intervenire, assegnandosi la \emph{Issue} oppure sollecitando i diretti interessati.

\subsection{Formazione}

\subsubsection{Scopo}

Lo scopo della formazione è quello di uniformare le capacità tecniche e le conoscenze tra i vari membri del gruppo, in modo da poter
lavorare e comunicare in sintonia. Per ogni membro di \emph{SpaghettiCode} è prevista la formazione tramite studio autonomo delle varie
tecnologie che vengono adoperate o che sono state richieste da \emph{Zucchetti S.p.A.} durante la presentazione del capitolato e gli
incontri successivi. In caso di difficoltà il gruppo è disponibile a fare formazione tramite incontri nei canali di comunicazione ufficiali.

\subsubsection{Aspettative}

Ci si aspetta che tutti i membri del gruppo acquisiscano familiarità con le seguenti tecnologie:
\begin{itemize}
    \item \LaTeX;
    \item Git e GitHub;
    \item JavaScript;
    \item Libreria D3.js.
\end{itemize}

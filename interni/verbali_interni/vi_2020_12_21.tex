\documentclass{article}

\usepackage[italian]{babel}
\usepackage[margin=20mm, footskip=20pt]{geometry}
\usepackage{graphicx}
\usepackage{subfiles}
\usepackage{hyperref}
\usepackage{nameref}
\usepackage{titlesec}
\usepackage{longtable}
\usepackage[table]{xcolor}
\usepackage{titling}
\usepackage{lastpage}
\usepackage{ifthen}
\usepackage{calc}
\usepackage{soulutf8}
\usepackage{contour}
\usepackage{float}
\usepackage{fancyhdr}
\usepackage{multirow}
\usepackage{pgfgantt}
\usepackage{lscape}
\usepackage{background}
\usepackage{lmodern}
\usepackage{textcomp}
\usepackage{lastpage}
\usepackage[utf8]{inputenc}
\usepackage{makecell}
\usepackage{listings}
\usepackage{parcolumns}

% \definecolor{lightgray}{rgb}{0.95, 0.95, 0.95}
\definecolor{darkgray}{rgb}{0.4, 0.4, 0.4}
%\definecolor{purple}{rgb}{0.65, 0.12, 0.82}
\definecolor{editorGray}{rgb}{0.95, 0.95, 0.95}
\definecolor{editorOcher}{rgb}{1, 0.5, 0} % #FF7F00 -> rgb(239, 169, 0)
\definecolor{editorGreen}{rgb}{0, 0.5, 0} % #007C00 -> rgb(0, 124, 0)
\definecolor{orange}{rgb}{1,0.45,0.13}		
\definecolor{olive}{rgb}{0.17,0.59,0.20}
\definecolor{brown}{rgb}{0.69,0.31,0.31}
\definecolor{purple}{rgb}{0.38,0.18,0.81}
\definecolor{lightblue}{rgb}{0.1,0.57,0.7}
\definecolor{lightred}{rgb}{1,0.4,0.5}

% definizione dei percorsi in cui cercare immagini
\graphicspath{ {./}
    {./src/img/}
}

% setup della sottolineatura
\setuldepth{Flat}
\contourlength{0.8pt}

\newcommand{\uline}[1]{%
  \ul{{\phantom{#1}}}%
  \llap{\contour{white}{#1}}%
}


% setup dei link
\hypersetup{
  % set true if you want colored links (instead of boxes)
  colorlinks=true,
  % set to all if you want both sections and subsections linked
  linktoc=all,
  % set color for file links
  filecolor=blue,
  % set color for internal links
  linkcolor=black,
  % set url color
  urlcolor=blue,
  % set characters encoding in the bookmarks tab
  pdfencoding=unicode,
}

% setup forma \paragraph e \subparagraph
\titleformat{\paragraph}[hang]{\normalfont\normalsize\bfseries}{\theparagraph}{1em}{}
\titleformat{\subparagraph}[hang]{\normalfont\normalsize\bfseries}{\thesubparagraph}{1em}{}

% setup profondità indice di default
\setcounter{secnumdepth}{5}
\setcounter{tocdepth}{5}

\makeatletter %% non togliere, i comandi che definiscono i placeholder vanno qui
% esempio di utilizzo: \appendToGraphicspath{./img/} (un comando diverso per ogni path da includere)
% N.B.: ci DEVE essere un forward slash alla fine del path, a indicare che è una cartella.
\newcommand\appendToGraphicspath[1]{%
  \g@addto@macro\Ginput@path{{#1}}%
}

\newcommand{\setTitle}[1]{%
  \newcommand{\@placeholderTitle}{#1}%
}
\newcommand{\placeholderTitle}{\@placeholderTitle}

\newcommand{\setUso}[1]{%
  \newcommand{\@uso}{#1}%
}
\newcommand{\uso}{\@uso}

\newcommand{\setVersione}[1]{%
  \newcommand{\@versione}{#1}%
}
\newcommand{\versione}{\@versione}

\newcommand{\disabilitaVersione}{%
  \renewcommand{\setVersione}[1]{}%
  \renewcommand{\versione}{DISABILITATA}
}

\newcommand{\setResponsabile}[1]{%
  \newcommand{\@responsabile}{#1}%
}
\newcommand{\responsabile}{\@responsabile}

\newcommand{\setRedattori}[1]{%
  \newcommand{\@redattori}{#1}%
}
\newcommand{\redattori}{\@redattori}

\newcommand{\setVerificatori}[1]{%
  \newcommand{\@verificatori}{#1}%
}
\newcommand{\verificatori}{\@verificatori}

\newcommand{\setDestinatari}[1]{%
  \newcommand{\@destinatari}{#1}%
}
\newcommand{\destinatari}{\@destinatari}

\newcommand{\setDescrizione}[1]{%
  \newcommand{\@descrizione}{#1}%
}
\newcommand{\descrizione}{\@descrizione}

\newcommand{\setModifiche}[1]{%
  \newcommand{\@modifiche}{#1}%
}

\newcommand{\modifiche}{\@modifiche}
\makeatother %% non togliere, i comandi che definiscono i placeholder vanno qui

% hook per lo script che genera il glossario
\newcommand{\glossario}[1]{#1\textsubscript{G}}

% comandi per rendere opzionali gli elenchi di figure
\newcommand{\elencoFigure}{%
  \renewcommand{\listfigurename}{Elenco delle figure}%
  \listoffigures%
}

\newcommand{\disabilitaElencoFigure}{%
  \renewcommand{\elencoFigure}{}%
}

% comandi per rendere opzionali le tabelle
\newcommand{\elencoTabelle}{%
  \renewcommand{\listtablename}{Elenco delle tabelle}%
  \listoftables%
}

\newcommand{\disabilitaElencoTabelle}{%
  \renewcommand{\elencoTabelle}{}%
}

% comando per riferirsi ad una sezione
\newcommand{\refSec}[1]{%
  \hyperref[#1]{\S\ref*{#1}}%
}

% CSS
\lstdefinelanguage{CSS}{
	keywords={
		color,
		background-image:,
		margin,
		padding,
		font,
		weight,
		display,
		position,
		top,
		left,
		right,
		bottom,
		list,
		style,
		border,
		size,
		white,
		space,
		min,
		width, 
		transition:, 
		transform:, 
		transition-property, 
		transition-duration, 
		transition-timing-function
	},	
	sensitive=true,
	morecomment=[l]{//},
	morecomment=[s]{/*}{*/},
	morestring=[b]',
	morestring=[b]",
	alsoletter={:},
	alsodigit={-}
}

% JavaScript
\lstdefinelanguage{JavaScript}{
	morekeywords={
		let,
		const,
		class,
		typeof, 
		new, 
		true, 
		false, 
		catch, 
		function, 
		return, 
		null, 
		catch, 
		switch, 
		var, 
		if, 
		in, 
		while, 
		do, 
		else, 
		case, 
		break
	},
	morecomment=[s]{/*}{*/},
	morecomment=[l]//,
	morestring=[b]",
	morestring=[b]'
}

\lstdefinelanguage{HTML5}{
	language=html,
	sensitive=true,	
	alsoletter={<>=-},	
	morecomment=[s]{<!-}{-->},
	tag=[s],
	otherkeywords={
		% General
		>,
		% Standard tags
		<!DOCTYPE,
		</html, <html, <head, <title, </title, <style, </style, <link, </head, <meta, />,
		% body
		</body, <body,
		% Divs
		</div, <div, </div>, 
		% Paragraphs
		</p, <p, </p>,
		% scripts
		</script, <script,
		% More tags...
		<canvas, /canvas>, <svg, <rect, <animateTransform, </rect>, </svg>, <video, <source, <iframe, </iframe>, </video>, <image, 
		</image>, <header, </header, <article, </article
	},
	ndkeywords={
		% General
		=,
		% HTML attributes
		charset=, src=, id=, width=, height=, style=, type=, rel=, href=,
		% SVG attributes
		fill=, attributeName=, begin=, dur=, from=, to=, poster=, controls=, x=, y=, repeatCount=, xlink:href=,
		% properties
		margin:, padding:, background-image:, border:, top:, left:, position:, width:, height:, margin-top:, margin-bottom:, font-size:, 
		line-height:,
		% CSS3 properties
		transform:, -moz-transform:, -webkit-transform:,
		animation:, -webkit-animation:,
		transition:,  transition-duration:, transition-property:, transition-timing-function:,
	}
}

\lstdefinestyle{htmlcssjs} {%
	% General design
	%  backgroundcolor=\color{editorGray},
	basicstyle={\small\ttfamily},   
	frame=single,
	% line-numbers
	xleftmargin={0.75cm},
	stepnumber=1,
	firstnumber=1,
	numberfirstline=true,	
	% Code design
	identifierstyle=\color{black},
	keywordstyle=\color{blue}\bfseries,
	ndkeywordstyle=\color{editorGreen}\bfseries,
	stringstyle=\color{editorOcher}\ttfamily,
	commentstyle=\color{brown}\ttfamily,
	% Code
	language=JavaScript,
	% alsolanguage=JavaScript,
	alsodigit={.:;},	
	tabsize=2,
	showtabs=false,
	showspaces=false,
	showstringspaces=false,
	extendedchars=true,
	breaklines=true,
}

\lstdefinestyle{html} {%
	% General design
	%  backgroundcolor=\color{editorGray},
	basicstyle={\small\ttfamily},   
	frame=single,
	% line-numbers
	xleftmargin={0.75cm},
	stepnumber=1,
	firstnumber=1,
	numberfirstline=true,	
	% Code design
	identifierstyle=\color{black},
	keywordstyle=\color{blue}\bfseries,
	ndkeywordstyle=\color{editorGreen}\bfseries,
	stringstyle=\color{editorOcher}\ttfamily,
	commentstyle=\color{brown}\ttfamily,
	% Code
	language=HTML5,
	alsodigit={.:;},	
	tabsize=2,
	showtabs=false,
	showspaces=false,
	showstringspaces=false,
	extendedchars=true,
	breaklines=true,
}


\appendToGraphicspath{../../config/src/img/}

\setTitle{Verbale Interno 2020-12-21}

\setVersione{v1.0.0}

\setResponsabile{
	Giorgia Paparazzo
}

\setRedattori{
	Contro Daniel Eduardo
}

\setVerificatori{
	Pagotto Manuel
}

\setUso{Interno}

\setDestinatari{
	prof. Vardanega Tullio \\ &
	prof. Cardin Riccardo \\ &
	SpaghettiCode
}

\setDescrizione{Riassunto dell'incontro realizzato dal gruppo \emph{SpaghettiCode} tenutosi il 21 Dicembre 2020 in forma di meeting online.}

\setModifiche{
	v1.0.0 & Giorgia Paparazzo & Responsabile & 2020-12-27 & Approvazione del documento \\
	v0.1.0 & Manuel Pagotto & Verificatore & 2020-12-23 & Verifica del documento \\
	v0.0.3 & Daniel Eduardo Contro & Analista & 2020-12-23 & Redatte \hyperref[sec:info_generali]{\S 1}, \hyperref[sub:sub:organizzazione_gruppo]{\S 3.1}, \hyperref[sub:convenzioni]{\S 3.4} e \hyperref[sec:conclusione]{\S 4} e riformattata \hyperref[sub:registro_modifiche]{\S 3.3} \\
	v0.0.2 & Daniel Eduardo Contro & Analista & 2020-12-22 & Redatte \hyperref[sub:git_latex]{\S 3.2} e \hyperref[sub:registro_modifiche]{\S 3.3} \\
	v0.0.1 & Daniel Eduardo Contro & Analista & 2020-12-21 & Creazione del documento e della suddivisione
}

\disabilitaElencoFigure
\disabilitaElencoTabelle

\begin{document}
	
\pagenumbering{gobble}


% variabile prima pagina (serve nell'istruzione dopo per stampare il bg solo nella prima pagina)
\newif\iffirstpage
\firstpagetrue

% setta l'immagina di bg
\backgroundsetup{
	scale=1,
	opacity=0.2,
	placement=top,
	contents={%
			\iffirstpage
				\includegraphics[width=\paperwidth]{datascience_og_colori.png}%
				\global\firstpagefalse
			\fi
		}%
}

\begin{titlepage}% per non stampare il numero della pagina

	\centering % allinea al centro la pagina
	\hspace{0.05\textwidth}% spazio tra linea e testo
	% lasciare questa riga per il corretto funzionamento di \parbox
	\parbox[b]{0.4\textwidth}{% cambiando la larghezza del testo il paragrafo si muove a destra o a sinistra 
	{\hspace{0.05\textwidth}\includegraphics[width=4cm,height=4cm]{logo_colori.png}}\\[3\baselineskip] % logo
	{\Huge\bfseries SpaghettiCode}\\ [\baselineskip] %titolo
	{\texttt{spaghetti.code.g6@gmail.com}}\\[\baselineskip]\\[4\baselineskip] % 
	{\Large\textsc\mbox{\placeholderTitle{}}}\\[4\baselineskip] % nome del documento
	{\begin{tabular}{r|l}
		\hline                                  \\
		% testo in grassetto
		\textbf{Versione}     & \versione{}     \\
		\rule{0pt}{3ex}%  EXTRA vertical height 
		\textbf{Approvazione} & \responsabile{} \\
		\rule{0pt}{3ex}%  EXTRA vertical height 
		\textbf{Redazione}    & \redattori{}    \\
		\rule{0pt}{3ex}%  EXTRA vertical height 
		\textbf{Verifica}     & \verificatori{} \\
		\rule{0pt}{3ex}%  EXTRA vertical height 
		\textbf{Uso}          & \uso{}          \\
		\rule{0pt}{3ex}%  EXTRA vertical height 
		\textbf{Destinato a}  & \destinatari{}
		\ifthenelse{\equal{\uso}{Esterno}}{
		\\ & Zucchetti S.p.A.
		}{}
	\end{tabular}}\\[4\baselineskip]

	}

	{\bfseries Descrizione}\\
	{\descrizione{}}\\[1\baselineskip]



\end{titlepage}

\newgeometry{textheight=660pt, lmargin=2cm, tmargin=2cm, rmargin=2cm}

% setup di header e footer nelle pagine senza numero
\fancypagestyle{nopage}{%
	\fancyhf{}%
	\fancyhead[R]{\includegraphics[width=1.3cm]{logo_colori.png}}%
	\fancyhead[L]{\emph{SpaghettiCode}\\\placeholderTitle{}}%
}
% setup di header e footer nelle pagine col numero
\fancypagestyle{usual}{%
	\fancyhf{}%
	\fancyhead[R]{\includegraphics[width=1.3cm]{logo_colori.png}}%
	\fancyhead[L]{\emph{SpaghettiCode}\\\placeholderTitle{}}%
	\fancyfoot[R]{\thepage\ di~\pageref{LastPage}}%
}
\setlength{\headheight}{1.8cm}

\newpage
\pagestyle{nopage}

\setcounter{table}{-1}

%REGISTRO DELLE MODIFICHE

\section*{Registro delle modifiche}
\label{sec:registro_delle_modifiche}

\rowcolors{2}{white!80!lightgray!90}{white}
\renewcommand{\arraystretch}{2} % allarga le righe con dello spazio sotto e sopra

\begin{longtable}
	[H]{|>{\centering\bfseries}m{2cm}|>{\centering}m{3.5cm}|>{\centering}m{2.5cm}|>{\centering}m{3cm}|>{\centering\arraybackslash}m{5cm}|}
	
	\hline
	\rowcolor{lightgray}
	{\textbf{Versione}} & {\textbf{Nominativo}} & {\textbf{Ruolo}} & {\textbf{Data}} & {\textbf{Descrizione}} \\
	\hline
	\endfirsthead
	
	\hline
	\rowcolor{lightgray}
	{\textbf{Versione}} & {\textbf{Nominativo}} & {\textbf{Ruolo}} & {\textbf{Data}} & {\textbf{Descrizione}} \\
	\hline
	\endhead
	
	\hline
 	\multicolumn{5}{|c|}{\emph{Continua alla pagina successiva......}}\\
	
	\endfoot
	\hline
	\endlastfoot
	
	\modifiche{}

\end{longtable}
% section registro_delle_modifiche (end)

\newpage
\thispagestyle{nopage}
\pagenumbering{roman}
\tableofcontents

\elencoFigure{}%
\elencoTabelle{}%

\newpage
\pagestyle{usual}
\pagenumbering{arabic}


\section{Informazioni generali}
\label{sec:info_generali}

\subsection{Informazioni incontro}
\label{sub:info_incontro}

\begin{itemize}
	\item \textbf{Luogo}: Applicazione desktop \glossario{Discord};
	\item \textbf{Data}: 2020-12-21;
	\item \textbf{Ora}: 17:30-19:30
	\item \textbf{Partecipanti}:
	\begin{itemize}
		\item Contro Daniel Eduardo
		\item Fichera Jacopo
		\item Kostadinov Samuel
		\item Masevski Martin
		\item Pagotto Manuel
		\item Paparazzo Giorgia
		\item Rizzo Stefano
	\end{itemize}
\end{itemize}
	

\section{Ordine del giorno}
\label{sec:ordine_del_giorno}
	Vengono riportati gli elementi che sono stati discussi nel corso del meeting:
	\begin{itemize}
		\item \nameref{sub:organizzazione_gruppo};
		\item \nameref{sub:registro_modifiche};
		\item \hyperref[sub:git_latex]{Versionamento dei file su \glossario{GitHub} e utilizzo di \glossario{LaTeX}};
		\item \nameref{sub:convenzioni}.
	\end{itemize}

\section{Resoconto}
\label{sec:resoconto}

	\subsection{Organizzazione del gruppo}
	\label{sub:organizzazione_gruppo}
	L'incontro si é aperto con una discussione riguardante la pianificazione dei ruoli nell'arco di tutta la durata del progetto, i cui risultati sono stati riportati nel 
	\textit{Piano di Progetto}. Inoltre si é discusso di alcune possibili problematiche identificate da alcuni membri del gruppo durante la stesura del documento 
	\textit{Studio di Fattibilità}.
	
	\subsection{Versionamento dei file su GitHub e utilizzo di \LaTeX}
	\label{sub:git_latex}
	A seguito della necessità di possedere una storia sequenziale e precisa dei documenti, emersa durante la stesura dello studio di fattibilità, si é svolta una discussione in merito 
	all'utilizzo della repository su GitHub e dell'utilizzo di \LaTeX\ per la redazione dei documenti. In un primo momento a questi strumenti era stato preferito 
	l'utilizzo di \glossario{Google Docs}, ma ora che il template per i documenti in \LaTeX\ é giunto ad una maturità ritenuta adeguata dal gruppo, é stato deciso di 
	proseguire con la redazione dei documenti utilizzando quest'ultimo, mantenendo la storia delle diverse versioni nel repository \textit{Documenti} dell'organizzazione 
	\emph{SpaghettiCode} su GitHub.

	\subsection{Problematiche del registro delle modifiche}
	\label{sub:registro_modifiche}
	Il gruppo si é confrontato, in seguito alla stesura del documento interno \textit{Studio di Fattibilità} e all'incontro (del 17 Dicembre 2020) svoltosi con il prof. Tullio 
	Vardanega, sul come compilare il registro delle modifiche di modo che non risultino conflitti di ruolo. Si é quindi deciso di procedere effettuando una verifica del documento ogni 
	qualvolta venga fatta una modifica sostanziale, in modo da assicurare la conformità dello stesso; inoltre si é deciso di incrementare la versione \textit{minor} ad ogni verifica.

	\subsection{Convenzioni da adottare nei documenti}
	\label{sub:convenzioni}
	Si é discusso in merito alle convenzioni da adottare nella nomenclatura dei documenti del repository ufficiale, e al termine si é optato per l'utilizzo dello stile \glossario
	{snake\_case}. Si é deciso anche di evitare l'utilizzo di caratteri speciali e spazi per non incorrere in possibili problematiche di compatibilità tra sistemi operativi diversi.

\section{Conclusione dell'incontro}
\label{sec:conclusione}
L'incontro si é concluso con una pianificazione dettagliata e distribuzione dei task da eseguire entro il prossimo incontro.

\end{document}

\documentclass{article}

\usepackage[italian]{babel}
\usepackage[margin=20mm, footskip=20pt]{geometry}
\usepackage{graphicx}
\usepackage{subfiles}
\usepackage{hyperref}
\usepackage{nameref}
\usepackage{titlesec}
\usepackage{longtable}
\usepackage[table]{xcolor}
\usepackage{titling}
\usepackage{lastpage}
\usepackage{ifthen}
\usepackage{calc}
\usepackage{soulutf8}
\usepackage{contour}
\usepackage{float}
\usepackage{fancyhdr}
\usepackage{multirow}
\usepackage{pgfgantt}
\usepackage{lscape}
\usepackage{background}
\usepackage{lmodern}
\usepackage{textcomp}
\usepackage{lastpage}
\usepackage[utf8]{inputenc}
\usepackage{makecell}
\usepackage{listings}
\usepackage{parcolumns}

% \definecolor{lightgray}{rgb}{0.95, 0.95, 0.95}
\definecolor{darkgray}{rgb}{0.4, 0.4, 0.4}
%\definecolor{purple}{rgb}{0.65, 0.12, 0.82}
\definecolor{editorGray}{rgb}{0.95, 0.95, 0.95}
\definecolor{editorOcher}{rgb}{1, 0.5, 0} % #FF7F00 -> rgb(239, 169, 0)
\definecolor{editorGreen}{rgb}{0, 0.5, 0} % #007C00 -> rgb(0, 124, 0)
\definecolor{orange}{rgb}{1,0.45,0.13}		
\definecolor{olive}{rgb}{0.17,0.59,0.20}
\definecolor{brown}{rgb}{0.69,0.31,0.31}
\definecolor{purple}{rgb}{0.38,0.18,0.81}
\definecolor{lightblue}{rgb}{0.1,0.57,0.7}
\definecolor{lightred}{rgb}{1,0.4,0.5}

% definizione dei percorsi in cui cercare immagini
\graphicspath{ {./}
    {./src/img/}
}

% setup della sottolineatura
\setuldepth{Flat}
\contourlength{0.8pt}

\newcommand{\uline}[1]{%
  \ul{{\phantom{#1}}}%
  \llap{\contour{white}{#1}}%
}


% setup dei link
\hypersetup{
  % set true if you want colored links (instead of boxes)
  colorlinks=true,
  % set to all if you want both sections and subsections linked
  linktoc=all,
  % set color for file links
  filecolor=blue,
  % set color for internal links
  linkcolor=black,
  % set url color
  urlcolor=blue,
  % set characters encoding in the bookmarks tab
  pdfencoding=unicode,
}

% setup forma \paragraph e \subparagraph
\titleformat{\paragraph}[hang]{\normalfont\normalsize\bfseries}{\theparagraph}{1em}{}
\titleformat{\subparagraph}[hang]{\normalfont\normalsize\bfseries}{\thesubparagraph}{1em}{}

% setup profondità indice di default
\setcounter{secnumdepth}{5}
\setcounter{tocdepth}{5}

\makeatletter %% non togliere, i comandi che definiscono i placeholder vanno qui
% esempio di utilizzo: \appendToGraphicspath{./img/} (un comando diverso per ogni path da includere)
% N.B.: ci DEVE essere un forward slash alla fine del path, a indicare che è una cartella.
\newcommand\appendToGraphicspath[1]{%
  \g@addto@macro\Ginput@path{{#1}}%
}

\newcommand{\setTitle}[1]{%
  \newcommand{\@placeholderTitle}{#1}%
}
\newcommand{\placeholderTitle}{\@placeholderTitle}

\newcommand{\setUso}[1]{%
  \newcommand{\@uso}{#1}%
}
\newcommand{\uso}{\@uso}

\newcommand{\setVersione}[1]{%
  \newcommand{\@versione}{#1}%
}
\newcommand{\versione}{\@versione}

\newcommand{\disabilitaVersione}{%
  \renewcommand{\setVersione}[1]{}%
  \renewcommand{\versione}{DISABILITATA}
}

\newcommand{\setResponsabile}[1]{%
  \newcommand{\@responsabile}{#1}%
}
\newcommand{\responsabile}{\@responsabile}

\newcommand{\setRedattori}[1]{%
  \newcommand{\@redattori}{#1}%
}
\newcommand{\redattori}{\@redattori}

\newcommand{\setVerificatori}[1]{%
  \newcommand{\@verificatori}{#1}%
}
\newcommand{\verificatori}{\@verificatori}

\newcommand{\setDestinatari}[1]{%
  \newcommand{\@destinatari}{#1}%
}
\newcommand{\destinatari}{\@destinatari}

\newcommand{\setDescrizione}[1]{%
  \newcommand{\@descrizione}{#1}%
}
\newcommand{\descrizione}{\@descrizione}

\newcommand{\setModifiche}[1]{%
  \newcommand{\@modifiche}{#1}%
}

\newcommand{\modifiche}{\@modifiche}
\makeatother %% non togliere, i comandi che definiscono i placeholder vanno qui

% hook per lo script che genera il glossario
\newcommand{\glossario}[1]{#1\textsubscript{G}}

% comandi per rendere opzionali gli elenchi di figure
\newcommand{\elencoFigure}{%
  \renewcommand{\listfigurename}{Elenco delle figure}%
  \listoffigures%
}

\newcommand{\disabilitaElencoFigure}{%
  \renewcommand{\elencoFigure}{}%
}

% comandi per rendere opzionali le tabelle
\newcommand{\elencoTabelle}{%
  \renewcommand{\listtablename}{Elenco delle tabelle}%
  \listoftables%
}

\newcommand{\disabilitaElencoTabelle}{%
  \renewcommand{\elencoTabelle}{}%
}

% comando per riferirsi ad una sezione
\newcommand{\refSec}[1]{%
  \hyperref[#1]{\S\ref*{#1}}%
}

% CSS
\lstdefinelanguage{CSS}{
	keywords={
		color,
		background-image:,
		margin,
		padding,
		font,
		weight,
		display,
		position,
		top,
		left,
		right,
		bottom,
		list,
		style,
		border,
		size,
		white,
		space,
		min,
		width, 
		transition:, 
		transform:, 
		transition-property, 
		transition-duration, 
		transition-timing-function
	},	
	sensitive=true,
	morecomment=[l]{//},
	morecomment=[s]{/*}{*/},
	morestring=[b]',
	morestring=[b]",
	alsoletter={:},
	alsodigit={-}
}

% JavaScript
\lstdefinelanguage{JavaScript}{
	morekeywords={
		let,
		const,
		class,
		typeof, 
		new, 
		true, 
		false, 
		catch, 
		function, 
		return, 
		null, 
		catch, 
		switch, 
		var, 
		if, 
		in, 
		while, 
		do, 
		else, 
		case, 
		break
	},
	morecomment=[s]{/*}{*/},
	morecomment=[l]//,
	morestring=[b]",
	morestring=[b]'
}

\lstdefinelanguage{HTML5}{
	language=html,
	sensitive=true,	
	alsoletter={<>=-},	
	morecomment=[s]{<!-}{-->},
	tag=[s],
	otherkeywords={
		% General
		>,
		% Standard tags
		<!DOCTYPE,
		</html, <html, <head, <title, </title, <style, </style, <link, </head, <meta, />,
		% body
		</body, <body,
		% Divs
		</div, <div, </div>, 
		% Paragraphs
		</p, <p, </p>,
		% scripts
		</script, <script,
		% More tags...
		<canvas, /canvas>, <svg, <rect, <animateTransform, </rect>, </svg>, <video, <source, <iframe, </iframe>, </video>, <image, 
		</image>, <header, </header, <article, </article
	},
	ndkeywords={
		% General
		=,
		% HTML attributes
		charset=, src=, id=, width=, height=, style=, type=, rel=, href=,
		% SVG attributes
		fill=, attributeName=, begin=, dur=, from=, to=, poster=, controls=, x=, y=, repeatCount=, xlink:href=,
		% properties
		margin:, padding:, background-image:, border:, top:, left:, position:, width:, height:, margin-top:, margin-bottom:, font-size:, 
		line-height:,
		% CSS3 properties
		transform:, -moz-transform:, -webkit-transform:,
		animation:, -webkit-animation:,
		transition:,  transition-duration:, transition-property:, transition-timing-function:,
	}
}

\lstdefinestyle{htmlcssjs} {%
	% General design
	%  backgroundcolor=\color{editorGray},
	basicstyle={\small\ttfamily},   
	frame=single,
	% line-numbers
	xleftmargin={0.75cm},
	stepnumber=1,
	firstnumber=1,
	numberfirstline=true,	
	% Code design
	identifierstyle=\color{black},
	keywordstyle=\color{blue}\bfseries,
	ndkeywordstyle=\color{editorGreen}\bfseries,
	stringstyle=\color{editorOcher}\ttfamily,
	commentstyle=\color{brown}\ttfamily,
	% Code
	language=JavaScript,
	% alsolanguage=JavaScript,
	alsodigit={.:;},	
	tabsize=2,
	showtabs=false,
	showspaces=false,
	showstringspaces=false,
	extendedchars=true,
	breaklines=true,
}

\lstdefinestyle{html} {%
	% General design
	%  backgroundcolor=\color{editorGray},
	basicstyle={\small\ttfamily},   
	frame=single,
	% line-numbers
	xleftmargin={0.75cm},
	stepnumber=1,
	firstnumber=1,
	numberfirstline=true,	
	% Code design
	identifierstyle=\color{black},
	keywordstyle=\color{blue}\bfseries,
	ndkeywordstyle=\color{editorGreen}\bfseries,
	stringstyle=\color{editorOcher}\ttfamily,
	commentstyle=\color{brown}\ttfamily,
	% Code
	language=HTML5,
	alsodigit={.:;},	
	tabsize=2,
	showtabs=false,
	showspaces=false,
	showstringspaces=false,
	extendedchars=true,
	breaklines=true,
}

%aggiungere percorsi dai quali il documento prende le immagini
\appendToGraphicspath{../../config/src/img/}

\setTitle{Verbale Interno 2020-11-25}

\setVersione{v1.0.0}

\setResponsabile{
    Paparazzo Giorgia
}

\setRedattori{
    Paparazzo Giorgia
}

\setVerificatori{
    Rizzo Stefano
}

\setUso{Interno}

\setDestinatari{
    prof. Vardanega Tullio \\ &
    prof. Cardin Riccardo \\ &
    SpaghettiCode
}

\setDescrizione{Verbale della chiamata del gruppo \emph{SpaghettiCode} del 2020-11-25.}

\disabilitaElencoFigure{}
\disabilitaElencoTabelle{}

\setModifiche{
  v1.0.0 & Giorgia Paparazzo & Responsabile & 2020-12-17 & Approvazione del documento & // \\
  v0.1.0 & Stefano Rizzo & Verificatore & 2020-12-14 & Verifica del documento & // \\
  v0.0.1 & Giorgia Paparazzo & Redattore & 2020-12-10 & Redazione del documento & //
}

\begin{document}

\pagenumbering{gobble}


% variabile prima pagina (serve nell'istruzione dopo per stampare il bg solo nella prima pagina)
\newif\iffirstpage
\firstpagetrue

% setta l'immagina di bg
\backgroundsetup{
	scale=1,
	opacity=0.2,
	placement=top,
	contents={%
			\iffirstpage
				\includegraphics[width=\paperwidth]{datascience_og_colori.png}%
				\global\firstpagefalse
			\fi
		}%
}

\begin{titlepage}% per non stampare il numero della pagina

	\centering % allinea al centro la pagina
	\hspace{0.05\textwidth}% spazio tra linea e testo
	% lasciare questa riga per il corretto funzionamento di \parbox
	\parbox[b]{0.4\textwidth}{% cambiando la larghezza del testo il paragrafo si muove a destra o a sinistra 
	{\hspace{0.05\textwidth}\includegraphics[width=4cm,height=4cm]{logo_colori.png}}\\[3\baselineskip] % logo
	{\Huge\bfseries SpaghettiCode}\\ [\baselineskip] %titolo
	{\texttt{spaghetti.code.g6@gmail.com}}\\[\baselineskip]\\[4\baselineskip] % 
	{\Large\textsc\mbox{\placeholderTitle{}}}\\[4\baselineskip] % nome del documento
	{\begin{tabular}{r|l}
		\hline                                  \\
		% testo in grassetto
		\textbf{Versione}     & \versione{}     \\
		\rule{0pt}{3ex}%  EXTRA vertical height 
		\textbf{Approvazione} & \responsabile{} \\
		\rule{0pt}{3ex}%  EXTRA vertical height 
		\textbf{Redazione}    & \redattori{}    \\
		\rule{0pt}{3ex}%  EXTRA vertical height 
		\textbf{Verifica}     & \verificatori{} \\
		\rule{0pt}{3ex}%  EXTRA vertical height 
		\textbf{Uso}          & \uso{}          \\
		\rule{0pt}{3ex}%  EXTRA vertical height 
		\textbf{Destinato a}  & \destinatari{}
		\ifthenelse{\equal{\uso}{Esterno}}{
		\\ & Zucchetti S.p.A.
		}{}
	\end{tabular}}\\[4\baselineskip]

	}

	{\bfseries Descrizione}\\
	{\descrizione{}}\\[1\baselineskip]



\end{titlepage}

\newgeometry{textheight=660pt, lmargin=2cm, tmargin=2cm, rmargin=2cm}

% setup di header e footer nelle pagine senza numero
\fancypagestyle{nopage}{%
	\fancyhf{}%
	\fancyhead[R]{\includegraphics[width=1.3cm]{logo_colori.png}}%
	\fancyhead[L]{\emph{SpaghettiCode}\\\placeholderTitle{}}%
}
% setup di header e footer nelle pagine col numero
\fancypagestyle{usual}{%
	\fancyhf{}%
	\fancyhead[R]{\includegraphics[width=1.3cm]{logo_colori.png}}%
	\fancyhead[L]{\emph{SpaghettiCode}\\\placeholderTitle{}}%
	\fancyfoot[R]{\thepage\ di~\pageref{LastPage}}%
}
\setlength{\headheight}{1.8cm}

\newpage
\pagestyle{nopage}

\setcounter{table}{-1}

%REGISTRO DELLE MODIFICHE

\section*{Registro delle modifiche}
\label{sec:registro_delle_modifiche}

\rowcolors{2}{white!80!lightgray!90}{white}
\renewcommand{\arraystretch}{2} % allarga le righe con dello spazio sotto e sopra

\begin{longtable}
	[H]{|>{\centering\bfseries}m{2cm}|>{\centering}m{3.5cm}|>{\centering}m{2.5cm}|>{\centering}m{3cm}|>{\centering\arraybackslash}m{5cm}|}
	
	\hline
	\rowcolor{lightgray}
	{\textbf{Versione}} & {\textbf{Nominativo}} & {\textbf{Ruolo}} & {\textbf{Data}} & {\textbf{Descrizione}} \\
	\hline
	\endfirsthead
	
	\hline
	\rowcolor{lightgray}
	{\textbf{Versione}} & {\textbf{Nominativo}} & {\textbf{Ruolo}} & {\textbf{Data}} & {\textbf{Descrizione}} \\
	\hline
	\endhead
	
	\hline
 	\multicolumn{5}{|c|}{\emph{Continua alla pagina successiva......}}\\
	
	\endfoot
	\hline
	\endlastfoot
	
	\modifiche{}

\end{longtable}
% section registro_delle_modifiche (end)

\newpage
\thispagestyle{nopage}
\pagenumbering{roman}
\tableofcontents

\elencoFigure{}%
\elencoTabelle{}%

\newpage
\pagestyle{usual}
\pagenumbering{arabic}


\section{Informazioni generali}%
\label{sec:informazioni_generali}

\subsection{Informazioni incontro}%
\label{sub:informazioni_incontro}
\begin{itemize}
    \item \textbf{Luogo}: Applicazione desktop \glossario{Discord};
    \item \textbf{Data}: 2020-11-25;
    \item \textbf{Ora}: 17:15 - 19:15;
    \begin{itemize}
		\item Contro Daniel Eduardo
		\item Fichera Jacopo
		\item Kostadinov Samuel
		\item Masevski Martin
		\item Pagotto Manuel
		\item Paparazzo Giorgia
		\item Rizzo Stefano
	\end{itemize}
\end{itemize}

\section{Ordine del giorno}%
\label{sec:ordine_del_giorno}

Di seguito vengono riportati i punti dell'ordine del giorno discussi:
\begin{itemize}
    \item Utilizzo della piattaforma \glossario{Trello};
    \item Scelta del mezzo di comunicazione da adottare;
    \item Utilizzo di \LaTeX per la documentazione;
    \item Mantenimento di documenti formali degli incontri;
    \item Confronto degli studi di fattibilità sui capitolati di interesse.
\end{itemize}

\section{Resoconto}%
\label{sec:resoconto}

\subsection{Utilizzo della piattaforma Trello}%
\label{sub:piattaforma_trello}
Il gruppo concorda di provare Trello per verificare il suo utilizzo, mantenendolo se il suo uso si rivela positivo per l’aspetto organizzativo. Si discute della possibilità di integrare i vari software con Git. E’ possibile allegare i documenti su Trello e viene ritenuto utile al fine di avere notifiche su eventuali cose importanti da tenere d’occhio.


\subsection{Scelta del mezzo di comunicazione da adottare}%
\label{sub:mezzo_comunicazione}
Si propone di usare \emph{Discord} per gestire e suddividere le informazioni importanti di cui bisogna discutere, e Telegram per interagire all’interno del gruppo. Si propone di creare su Discord una serie di canali per organizzare meeting tra di noi e fissare date di discussioni.

\subsection{Utilizzo di \LaTeX per la documentazione}%
\label{sub:latex}
Viene sottolineata la necessità di scrivere la documentazione in maniera condivisa. \\
\emph{Visual Studio Code} e https://www.overleaf.com/, sono presentati come soluzioni in grado di adempiere allo scopo. \LaTeX viene considerato utile perché permette di produrre documenti più rapidi e completi. Viene fatto notare che i documenti Google sono più comodi nella condivisione e visione da parte di più utenti. È stato proposto di creare un progetto su Overleaf condiviso con tutti, dove è possibile modificare il progetto tramite link d’invito. Inoltre è stato appurato si possa condividere su Dropbox, GitHub e Google Drive.


\subsection{Mantenimento di documenti formali degli incontri}%
\label{sub:documenti_formali}%
Il gruppo decide di iniziare a creare dei verbali per ogni incontro formale.


\subsection{Confronto degli studi di fattibilità sui capitolati di interesse}%
\label{sub:studi_fattibilita}
Vengono esposti i risultati dell’analisi individuale dei capitolati di interesse:
\begin{itemize}
    \item Il C2 risulta vantaggioso per il fatto che lo sviluppo di un’applicazione del genere richieda conoscenze richieste nel mondo del lavoro. Importante il fatto che l’azienda sappia precisamente cosa vuole e come debba essere fatto. Pericoloso è lo sviluppo del back-end che risulta complesso. Viene evidenziato che ci sia molto lavoro da svolgere;
    \item Il C3 ha già la richiesta di gruppi soddisfatta. Non è chiaro cosa vogliano e la simulazione dei dati solleva perplessità. La parte di video riconoscimento è opzionale, ma è fattibile perché ci sono molti esempi online. Il gruppo conviene sul fatto che si rischierebbe di passare più tempo a simulare dei dati piuttosto che a costruire modelli, con la paura che il lavoro possa non venire valorizzato;
    \item Il C4 è fortemente improntato alla programmazione web; Non vi sono limiti particolari riguardo le tecnologie da usare. Viene consigliato di usare la libreria D3.js perché fornisce strumenti orientati alla visualizzazione dei dati. Sembra essere vantaggioso il fatto che ci sia una base da cui partire: ci sono tutorial da cui partire, viene presentata l’ipotesi che sia possibile usare il software che è stato mostrato durante la presentazione del capitolato;
    \item Il C5 richiede principalmente sviluppo back-end. Sembra che i committenti richiedano qualcosa di centralizzato. Per quanto riguarda le tecnologie da usare Java potrebbe sembrerebbe essere una buona soluzione per il back-end. Vengono presentate alcune possibili idee relative alla progettazione. Viene posto un punto di domanda relativamente al come effettuare testing. Vengono anche evidenziate alcune possibili problematiche relative al design del progetto proposto dall’azienda.
\end{itemize}

\setlength{\parindent}{0pt}Successivamente il gruppo esprime le proprie preferenze sui capitolati adottando una scala da 1 a 3 (3 il massimo interesse ed 1 il minimo):

\rowcolors{2}{white!80!lightgray!90}{white}
\renewcommand{\arraystretch}{2} % allarga le righe con dello spazio sotto e sopra
\begin{longtable}[H]{>{\centering\bfseries}m{2cm} >{\centering}m{3.5cm} >{\centering}m{2.5cm} >{\centering}m{3cm} >{\centering\arraybackslash}m{5cm}}
    \rowcolor{lightgray}
    {} & {\textbf{C2}} & {\textbf{C3}} & {\textbf{C4}} & {\textbf{C5}}  \\
    \endfirsthead%
    \rowcolor{lightgray}
    {} & {\textbf{C2}} & {\textbf{C3}} & {\textbf{C4}} & {\textbf{C5}}  \\
    \endhead%
    Daniel & & 2 & 1 & 3 \\
    Jacopo & & 1 & 2 & 3 \\
    Samuel & 1 & & 2 & 3 \\
    Martin & 1 & & 3 & 2 \\
    Manuel & & 2 & 3 & 1 \\
    Giorgia & & 2 & 3 & 1 \\
    Stefano & 2 & 1 & 3 &  \\
    \textbf{TOTALE} & 6 & 6 & 17 & 13
\end{longtable}

\section{Conclusione dell’incontro}%
\label{sec:conclusione_incontro}
È stato scelto l’ordinamento dei capitolati. Il gruppo sceglie di partecipare al capitolato C4 come prima scelta, C5 come seconda e C2 come terza scelta.

\end{document}
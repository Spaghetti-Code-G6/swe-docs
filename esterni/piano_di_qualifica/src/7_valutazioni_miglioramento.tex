\documentclass[../piano_di_qualifica.tex]{subfiles}
\begin{document}

Qui di seguito viene presentata la valutazione fatta dai membri del gruppo SpaghettiCode circa il lavoro svolto durante l'attività conclusa.\\
I problemi analizzati riguardano:
\begin{itemize}
	\item l'organizzazione: ovvero problemi relativi alla comunicazione e organizzazione;
	\item ruoli: problemi relativi al corretto svolgimento dei ruoli;
	\item strumenti di lavoro: problemi relativi all'impiego di strumenti scelti.
\end{itemize}
\emph{Questa sezione è attualmente incompleta, verrà integrata col proseguimento del progetto}.

\subsection{Valutazioni sull'organizzazione}
\label{sub:valut_org}
Questo anno particolare ha richiesto molti cambiamenti e flessibilità da parte degli studenti. Sebbene ad oggi la maggior parte dei compiti e
degli incontri è possibile svolgerla online, non sono mancate alcune difficoltà.

\begin{center}
	\begin{longtable}{|l|p{6cm}|p{6cm}|}
		\hline
		\rowcolor{lightgray}
		{\textbf{Problema}} & {\textbf{Descrizione}} & {\textbf{Soluzione}} \\
		\hline
		Incontri di gruppo & Non sempre tutti sono disponibili per un certo momento concordato &
		Si è deciso di fare una riunione con almeno 5 membri del gruppo \\
		\hline
		\rowcolor{white}
		\caption{Difficoltà sorte nell'organizzazione}
	\end{longtable}
\end{center}


\subsection{Valutazioni degli strumenti}
\label{sub:valut_strumenti}

\begin{center}
	\begin{longtable}{|p{2.5cm}|p{4.5cm}|p{4.5cm}|p{4.5cm}|}
		\hline
		\rowcolor{lightgray}
		\textbf{Strumento} & \textbf{Problema rilevato} & \textbf{Contromisura} & \textbf{Effetti} \\
		Version Control System & Non tutti i componenti avevano confidenza con strumenti di versionamento & Alcuni colleghi hanno fatto una sorta di lezione per aiutare i meno esperti ad allinearsi al gruppo & Grazie alle celeri spiegazioni tutti hanno potuto lavorare subito ai documenti, in conformità con l'organizzazione amministrativa del momento \\
		\LaTeX & Quasi nessuno aveva esperienza con questo strumento di scrittura & Un membro che aveva già avuto esperienze precedenti ha preparato e fornito al gruppo un template pronto, e ha tracciato una guida di utilizzo & Grazie a questo supporto tutti hanno potuto produrre documenti conformi al nuovo standard stilistico stabilito, nonchè velocizzare la produzione di documenti\\
		\hline
		\rowcolor{white}
		\caption{Valutazioni degli strumenti}
	\end{longtable}
\end{center}


\subsection{Valutazioni dei ruoli}
\label{sub:valut_ruoli}

\subsubsection{Fase di Revisione dei Requisiti}

\begin{center}
	\begin{longtable}{|p{2.5cm}|p{4.5cm}|p{4.5cm}|p{4.5cm}|}
		\hline
		\rowcolor{lightgray}
		{\textbf{Ruolo}} & {\textbf{Problema rilevato}} & {\textbf{Contromisura}} & {\textbf{Effetti}} \\
		\hline
		\endfirsthead
		\hline
		\rowcolor{lightgray}
		{\textbf{Ruolo}} & {\textbf{Problema rilevato}} & {\textbf{Contromisura}} & {\textbf{Effetti}} \\
		\hline
		\endhead

		\hline
		\rowcolor{white}
		\multicolumn{3}{|c|}{\emph{Continua alla pagina successiva...}} \\
		\hline
		\endfoot
		\endlastfoot

		Responsabile & Per mancata esperienza precedente in questo ambito è stato necessario un supporto & Tutto il gruppo è stato in grado di autogestirsi; ognuno ha scelto attivamente che ruolo assumere e di conseguenza che documenti redigere; il ruolo di responsabile è stato affidato a più persone & Grazie al supporto reciproco è stato possibile tracciare le prime linee guida del gruppo, grazie al costante controllo è stato possibile tenere saldo il gruppo nonché ridurre quanto possibile i ritardi \\
		Analista & Il ruolo di analista è fondamentale e allo stesso tempo delicato. Delineare attualmente i requisiti richiesti è un lavoro difficile in quanto si inizia ad avere consapevolezza della portata del progetto & Si è deciso di affidare il ruolo di Analista a più componenti in quanto lavorare in team favorisce la ricerca e la comprensione dei vari aspetti del progetto. D'altra parte si è sofferto nell'organizzazione di un singolo documento tra più soggetti & Affidare lo stesso documento a più persone ha fatto soffrire un po’ l’orchestrazione dei vari componenti dello stesso, causa anche di una amministrazione acerba, in fase di costruzione secondo le necessità che si stavano vivendo \\
		Verificatore & Partire con il piede giusto è fondamentale, quindi si vede necessario l'impiego di più verificatori & Si è scelto di affidare il ruolo di Verificatore a tutti i componenti del gruppo in quanto ciascuno riesce a notare aspetti che qualcun altro non percepisce, inoltre impegnarsi tutti nella verifica ha permesso di accelerare i tempi di approvazione dei documenti & L’attività di verifica si è dimostrata attenta riducendo il più possibile il rischio di errori sfuggiti nei documenti \\
		\hline
		\rowcolor{white}
		\caption{Valutazioni dei ruoli in fase RR}
	\end{longtable}
\end{center}

\subsubsection{Fase di Revisione di Progettazione}

\begin{center}
	\begin{longtable}{|p{2.5cm}|p{4.5cm}|p{4.5cm}|p{4.5cm}|}
		\hline
		\rowcolor{lightgray}
		{\textbf{Ruolo}} & {\textbf{Problema rilevato}} & {\textbf{Contromisura}} & {\textbf{Effetti}}  \\
		\hline
		\endfirsthead
		\hline
		\rowcolor{lightgray}
		{\textbf{Ruolo}} & {\textbf{Problema rilevato}} & {\textbf{Contromisura}} & {\textbf{Effetti}} \\
		\hline
		\endhead

		\hline
		\rowcolor{white}
		\multicolumn{3}{|c|}{\emph{Continua alla pagina successiva...}} \\
		\hline
		\endfoot
		\endlastfoot

		Responsabile & Vista l’attitudine del gruppo a perdersi nel tempo, il responsabile ha cercato di mantenere il gruppo compatto per garantire l’avanzamento del progetto & Il responsabile ha chiesto periodicamente lo stato dei lavori, garantendo comunicazione e percezione del ritardo presente e della necessità di darsi da fare & Grazie a queste accortezze il ritardo maturato è stato tutto sommato minimo rispetto, perfettamente in linea con quello maturato in precedenza \\
		Amministratore & La fase precedente ha evidenziato i punti deboli nell’amministrazione della repository, mettendo il nuovo amministratore di fronte alla necessità di ridefinire le risorse e gli spazi & Grazie al supporto reciproco gli amministratori hanno ridefinito le regole di utilizzo della repository e di versionamento, creando uno standard comune a tutti & Il gruppo ha potuto lavorare più agilmente, anche nella divisione di uno stesso documento tra più persone, riducendo così il rischio di conflitti  \\
		Verificatore & Nella fase precedente i verificatori agivano liberamente in base ad esigenze esplicitate via chat oppure in base alla eventuale mancanza di lavoro da fare & Grazie alla nuova amministrazione si è potuto mantenere un ritmo di verifica più costante. Ogni modifica prevedeva una pull request e quindi la relativa verifica da qualcuno indicato dal richiedente & La verifica è stata costante, ad ogni modifica, aumentando notevolmente la bontà dei documenti, nonché migliorando il versionamento stesso, in quanto ora in tabella vengono inserite solo le modifiche che hanno avuto esito positivo \\
		Progettista & La figura del progettista e programmatore è stata fortemente coincidente, con una netta prevalenza di programmazione in stile western. Nella produzione del PoC si è stabilito i requisiti da soddisfare e si è subito programmato senza stabilire prima delle linee guida & Per la fase successiva si ha dato più spazio alla progettazione, stabilendo le linee guida del software, per poi costruirlo & Grazie ad una attenta valutazione delle necessità software si è potuto sveltire la programmazione in quanto vi erano sufficienti linee guida a cui attenersi \\

		\hline
		\rowcolor{white}
		\caption{Valutazioni dei ruoli in fase RP}
	\end{longtable}
\end{center}

\subsubsection{Fase di Revisione di Qualifica}

\begin{center}
	\begin{longtable}{|p{2.5cm}|p{4.5cm}|p{4.5cm}|p{4.5cm}|}
		\hline
		\rowcolor{lightgray}
		{\textbf{Ruolo}} & {\textbf{Problema rilevato}} & {\textbf{Contromisura}} & {\textbf{Effetti}}  \\
		\hline
		\endfirsthead
		\hline
		\rowcolor{lightgray}
		{\textbf{Ruolo}} & {\textbf{Problema rilevato}} & {\textbf{Contromisura}} & {\textbf{Effetti}} \\
		\hline
		\endhead

		\hline
		\rowcolor{white}
		\multicolumn{3}{|c|}{\emph{Continua alla pagina successiva...}} \\
		\hline
		\endfoot
		\endlastfoot

		Progettista e programmatore & Si è visto necessario creare una architettura che facesse da base all’applicativo, affinchè tutti i programmatori sapessero precisamente cosa fare, come farlo, ma anche sapessero come sarebbe stato realizzato il compito degli altri colleghi & Tutti i progettisti hanno collaborato a tracciare l’architettura dell’applicativo, permettendo quindi uno sviluppo più rapido & Grazie all’architettura tracciata tutti i programmatori sapevano cosa avrebbero fatto i colleghi, e  come sarebbe stato fatto, permettendo poi una rapida integrazione delle singoli componenti in una unica unità  \\

		\hline
		\rowcolor{white}
		\caption{Valutazioni dei ruoli in fase RQ}
	\end{longtable}
\end{center}

\end{document}

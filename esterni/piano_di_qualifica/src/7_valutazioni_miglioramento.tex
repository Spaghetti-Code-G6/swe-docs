\documentclass[../piano_di_qualifica.tex]{subfiles}
\begin{document}

Qui di seguito viene presentata la valutazione fatta dai membri del gruppo SpaghettiCode circa il lavoro svolto durante l'attività conclusa.\\
I problemi analizzati riguardano:
\begin{itemize}
	\item l'organizzazione: ovvero problemi relativi alla comunicazione e organizzazione;
	\item ruoli: problemi relativi al corretto svolgimento dei ruoli;
	\item strumenti di lavoro: problemi relativi all'impiego di strumenti scelti.
\end{itemize}
\emph{Questa sezione è attualmente incompleta, verrà integrata col proseguimento del progetto}.

\subsection{Valutazioni sull'organizzazione}
\label{sub:valut_org}
Questo anno particolare ha richiesto molti cambiamenti e flessibilità da parte degli studenti. Sebbene ad oggi la maggior parte dei compiti e
degli incontri è possibile svolgerla online, non sono mancate alcune difficoltà.

\begin{center}
	\begin{longtable}{|l|p{6cm}|p{6cm}|}
		\hline
		\rowcolor{lightgray}
		{\textbf{Problema}} & {\textbf{Descrizione}} & {\textbf{Soluzione}} \\
		\hline
		Incontri di gruppo & Non sempre tutti sono disponibili per un certo momento concordato &
		Si è deciso di fare una riunione con almeno 5 membri del gruppo \\
		\hline
		\rowcolor{white}
		\caption{Difficoltà sorte nell'organizzazione}
	\end{longtable}
\end{center}

\subsection{Valutazioni dei ruoli}
\label{sub:valut_ruoli}
Tutti i ruoli sono e saranno sempre assunti su base volontaria. Essendo la prima volta per ciascun membro, sono emerse alcune difficoltà che tuttavia sono state superate.\par

\begin{center}
	\begin{longtable}{|p{2.5cm}|p{7cm}|p{7cm}|}
		\hline
		\rowcolor{lightgray}
		{\textbf{Ruolo}} & {\textbf{Problema rilevato}} & {\textbf{Contromisura}} \\
		\hline
		\endfirsthead
		\hline
		\rowcolor{lightgray}
		{\textbf{Ruolo}} & {\textbf{Problema rilevato}} & {\textbf{Contromisura}} \\
		\hline
		\endhead

		\hline
		\rowcolor{white}
		\multicolumn{3}{|c|}{\emph{Continua alla pagina successiva...}} \\
		\hline
		\endfoot
		\endlastfoot

		Responsabile & Per mancata esperienza precedente in questo ambito è stato necessario un supporto & Tutto il gruppo è stato in grado di autogestirsi; ognuno ha scelto attivamente che ruolo assumere e di conseguenza che documenti redigere; il ruolo di responsabile è stato affidato a più persone \\
		Analista & Il ruolo di analista è fondamentale e allo stesso tempo delicato. Delineare attualmente i requisiti richiesti è un lavoro difficile in quanto si inizia ad avere consapevolezza della portata del progetto & Si è deciso di affidare il ruolo di Analista a più componenti in quanto lavorare in team favorisce la ricerca e la comprensione dei vari aspetti del progetto. D'altra parte si è sofferto nell'organizzazione di un singolo documento tra più soggetti \\
		Verificatore & Partire con il piede giusto è fondamentale, quindi si vede necessario l'impiego di più verificatori & Si è scelto di affidare il ruolo di Verificatore a tutti i componenti del gruppo in quanto ciascuno riesce a notare aspetti che qualcun altro non percepisce, inoltre impegnarsi tutti nella verifica ha permesso di accelerare i tempi di approvazione dei documenti \\
		Amministratore & Nella ricerca degli strumenti più adeguati per svolgere il lavoro non è stato sempre facile stabilire delle norme che mettessero d'accordo tutti & Grazie alla comunicazione interna è stato possibile delineare gli standard e le convenzioni da adottare per svolgere i vari ruoli, soprattutto per quanto riguarda la documentazione \\
		Progettista & La scelta del modello architetturale da adottare e l'identificazione dei design pattern all' interno del nostro progetto ha occupato parte del tempo di lavoro, inoltre è stato dedicato del tempo allo studio degli stessi & In principio, per evitare ritardi sul lavoro, chi ha svolto il ruolo di Analista nella fase precedente ha svolto anche il ruolo di Progettista nella fase di RP \\

		\hline
		\rowcolor{white}
		\caption{Valutazioni dei ruoli}
	\end{longtable}
\end{center}

\subsection{Valutazioni degli strumenti}
\label{sub:valut_strumenti}

\begin{center}
	\begin{longtable}{|p{3cm}|p{4.5cm}|p{4.5cm}|}
		\hline
		\rowcolor{lightgray}
		\textbf{Strumento} & \textbf{Problema rilevato} & \textbf{Contromisura} \\
		Version Control System & Non tutti i componenti avevano confidenza con strumenti di versionamento & Alcuni colleghi hanno fatto una sorta di lezione per aiutare i meno esperti ad allinearsi al gruppo \\
		\LaTeX & Quasi nessuno aveva esperienza con questo strumento di scrittura & Un membro che aveva già avuto esperienze precedenti ha preparato e fornito al gruppo un template pronto, e ha tracciato una guida di utilizzo \\
		\hline
		\rowcolor{white}
		\caption{Valutazioni degli strumenti}
	\end{longtable}
\end{center}

\subsection{Retrospettiva sulle fasi concluse}
\label{sub:retrospettiva}

\subsubsection{Retrospettiva fase di Analisi (RR)}
\label{par:retrospettiva-RR}
Alla fine della fase di Analisi il gruppo ha ritenuto opportuno incontrarsi per discutere del risultato conseguito, delle segnalazioni ricevute nella valutazione, e delle varie soluzioni da adottare per correggere gli errori.

Per gli errori compresi si è proceduto rapidamente alla opportuna correzione, per altri è stato necessario chiedere un incontro
con i professori.

Qui di seguito vengono elencate alcune delle osservazioni fatte e le relative contromisure prese.

\begin{center}
	\begin{longtable}{|p{5cm}|p{10cm}|}
		\hline
		\rowcolor{lightgray}
		\textbf{Osservazioni} & \textbf{Soluzioni adottate} \\
		\hline

		Scatto di versionamento & Si è scelto di emanciparsi dai gruppi degli anni precedenti scegliendo modificare	lo scatto relativo agli indici X.Y.Z. dei documenti. Nello specifico la modifica o correzione di una sezione/paragrafo già esistente corrisponde ad uno scatto dell'indice Z, l'aggiunta di nuove sezioni/paragrafi comporta lo scatto dell'indice Y, l'aggiunta di nuove sezioni/paragrafi tali che essi rendano incompatibile il documento con le sue versioni precedenti corrisponde ad uno scatto dell'indice X. Per ulteriori dettagli vedere il documento \textsc{Norme di Progetto} \\
		La pianificazione presentata nel PdP non concorda con l’adozione del modello di sviluppo incrementale dichiarato & Si è scelto di modificare la pianificazione e di renderla più fine e maggiormente conforme al modello di sviluppo incrementale. \\
		Il PaF per ora è un mero esercizio contabile & Si è ragionato sul concetto di PaF al fine di trasformarlo in un'attività utile alla comprensione dell'andamento del gruppo e dei costi. \\
		Differenza tra verbali e resoconti & A seguito di un incontro col prof. Vardanega si è capito meglio la differenza tra i verbali e i resoconti. Abbiamo quindi cambiato approccio avvicinandoci più al concetto stesso di verbale. \\
		Attualizzazione dei rischi & Si è scelto di creare una nuova sezione nel \textsc{Piano di Progetto} che trattasse questa sezione mancante. \\
		Vari errori circa il documento \textsc{Analisi dei Requisiti v.1.0.0} & A seguito di un incontro col prof. Cardin sono state effettuate le opportune correzioni. \\

		\hline
		\rowcolor{white}
		\caption{Retrospettiva in fase di analisi}
	\end{longtable}
\end{center}

Per quanto riguarda i ruoli il gruppo ha maturato una maggiore comprensione circa i compiti degli stessi e come applicarsi. Nello specifico:
\begin{itemize}
	\item \textbf{Responsabile:} conoscendo meglio i membri del gruppo e le loro abilità il Responsabile riesce ad assegnare compiti più in linea con le abilità di tutti.
	\item \textbf{Amministratore:} una volta capite le necessità del gruppo l'Amministratore riesce a rettificare le norme e organizzare la repository.
	\item \textbf{Analista:} a seguito della valutazione e di ulteriori incontri con l'azienda l'Analista riesce a rettificare il documento \textsc{Analisi dei Requisiti} e perseguire meglio gli obiettivi richiesti dal proponente.
	\item \textbf{Verificatore:} grazie alla nuova amministrazione della repository il Verificatore riesce a eseguire meglio il suo lavoro.
\end{itemize}

Anche riguardo gli strumenti il gruppo ha acquisito maggiore confidenza con gli stessi ed è riuscito a lavorare in maniera più spedita.

\subsubsection{Retrospettiva fase di progettazione e codifica TB (RP)}
\label{par:retrospettiva-RP}

Il gruppo si è riunito e ha discusso delle problematiche segnalate nella valutazione dei documenti consegnati alla RP.
Gli errori più gravi hanno riguardato segnalazioni ricevute in precedenza, infatti nonostante le soluzioni adottate, queste non sono state risolutive dei problemi emersi.

Dopo aver discusso delle possibili soluzioni e averle trovate, si è deciso di chiedere direttamente ai professori una conferma della loro correttezza.

Per gli errori meno gravi, invece, si sono applicate prontamente le soluzioni discusse solamente tra i membri del gruppo.

\begin{center}
	\begin{longtable}{|p{4cm}|p{11cm}|}
		\hline
		\rowcolor{lightgray}
		\textbf{Osservazioni} & \textbf{Soluzioni adottate} \\
		\hline

		Versionamento & Si associa ad ogni azione assegnata a un individuo (creazione, modifica, rimozione) la corrispondente verifica (anch'essa associata a un altro individuo), se la verifica approva, allora lo stato del CI cambia e con esso il numero di versione; altrimenti l'azione fallisce e il CI non cambia. \\
		Modello di sviluppo & Si fa affidamento su una pianificazione basata sugli obiettivi. \\
		Consuntivo di periodo & Si ragiona sulla pianificazione oraria futura in base alle esigenze che si pensa possano essere più utili per il futuro sulla base dell'esperienza attuale e sullo stato del prodotto. \\
		Cruscotto di valutazione & Si aggiungono informazioni sugli andamenti globali delle varie fasi e si mettono a grafico i dati in modo da renderli facilmente fruibili. \\
		Struttura verbali & Adottare un sistema di tracciamento delle decisioni prese in una tabella a fine di ogni verbale. \\
		Quantità verbali & Si fa almeno un incontro con il proponente in modo da coinvolgerlo e aggiornarlo sullo stato dello sviluppo. \\
		Processi nelle Norme & Si approfondisce lo standard \glossario{ISO/IEC 12207} per poter comprendere meglio la natura dei processi e le loro relazioni. \\
		Attuazione dei rischi & Si descrive più in dettaglio il processo di mitigazione del problema che si è verificato in modo che venga risolto nell'immediato, che guardi al futuro e, quindi, che non si presenti più grazie alla soluzione adottata nel presente. \\
		Glossario & Corretti i bookmarks mancanti. \\
		Errori in AR & Risolti gli errori segnalati. \\

		\hline
		\rowcolor{white}
		\caption{Retrospettiva in fase di progettazione e codifica TB}
	\end{longtable}
\end{center}

% TODO: altre cose da segnalare, ad esempio la TB non passata

\end{document}

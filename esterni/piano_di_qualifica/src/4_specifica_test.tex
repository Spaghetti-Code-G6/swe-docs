\documentclass[../piano_di_qualifica.tex]{subfiles}

\begin{document}

Per garantire la qualità di prodotto è necessario stabilire delle metriche per l’esecuzione e il soddisfacimento dei test. Tuttavia in questa fase di del progetto è prematuro stabilire delle metriche precise e complete.


\begin{table}[!ht]
	\centering
	\begin{tabular}{|c|c|c|c|}
		\hline
		\rowcolor{lightgray}
		\textbf{Metrica} & \textbf{Nome}                          & \textbf{Accettabile} & \textbf{Ottimale} \\
		\hline
		MTS1             & Test eseguiti in rapporto ai requisiti & 100\%                & 100\%             \\
		MTS2             & Percentuale test passati               & 85\%                 & 100\%             \\
		\hline
	\end{tabular}
	\caption{Metriche dei test}
\end{table}

\subsection{Test di accettazione}%
\label{sub:test_accett}
Sono test che dimostrano che il prodotto realizzato soddisfa i requisiti concordati con il proponente.
Questi test si dividono in test di sistema, test di integrazione e test di unità.

\subsection{Test di sistema}%
\label{sub:test_sist}
Per assicurare che il progetto rispetti i requisiti identificati nel documento \textsc{Analisi dei Requisiti} verranno eseguiti i seguenti test.

\begin{center}
	\begin{longtable}{|c|c|p{8cm}|c|}
		\hline
		\rowcolor{lightgray}
		{\textbf{Codice}} & {\textbf{Riferimento}} & {\textbf{Descrizione}}                                                                                                    & {\textbf{Implementazione}} \\
		\hline
		\endfirsthead
		\hline
		\rowcolor{lightgray}
		{\textbf{Codice}} & {\textbf{Riferimento}} & {\textbf{Descrizione}}                                                                                                    & {\textbf{Implementazione}} \\
		\hline
		\endhead

		\hline
		\rowcolor{white}
		\multicolumn{4}{|c|}{\emph{Continua alla pagina successiva...}}                                                                                                                                     \\
		\hline
		\endfoot
		\endlastfoot

		TS0               & RFO1                   & Si verifichi che l'utente possa creare un ambiente di lavoro                                                              & Sì                         \\
		TS1               & RFO1.1                 & Si verifichi che l'utente possa inserire dati nel sistema                                                                 & Sì                         \\
		TS2               & RFO1.2                 & Si verifichi che l'utente possa importare dati tramite file CSV                                                           & Sì                         \\
		TS3               & RFO1.3                 & Si verifichi che l'utente possa importare dati tramite un database esterno                                                & Sì                         \\
		TS4               & RFO1.3.1               & Si verifichi che l'utente possa aprire un collegamento con un database                                                    & Sì                         \\
		TS5               & RFO1.3.2               & Si verifichi che l'utente possa importare i dati nel sistema mediante la ricerca su un database tra quelli disponibili    & Sì                         \\
		TS6               & RFO2                   & Si verifichi che l'utente possa creare  un grafico                                                                        & Sì                         \\
		TS7               & RFO2.1                 & Si verifichi che l'utente possa selezionare la costruzione di un grafico di sua scelta                                    & Sì                         \\
		TS8               & RFO2.2                 & Si verifichi che l'utente possa selezionare l'opzione di costruzione di un grafico di tipo Scatter Plot Matrix            & Sì                         \\
		TS9               & RFO2.3                 & Si verifichi che l'utente possa selezionare l'opzione di costruzione di un grafico di tipo Force Field                    & Sì                         \\
		TS10              & RFO2.4                 & Si verifichi che l'utente possa selezionare l'opzione di costruzione di un grafico di tipo Heat Map                       & Sì                         \\
		TS11              & RFO2.5                 & Si verifichi che l'utente possa selezionare l'opzione di costruzione di un grafico di tipo PLMA                           & Sì                         \\
		TS12              & RFO2.6                 & Si verifichi che l'utente possa selezionare l'opzione di costruzione di un grafico di tipo Distance Map                   & Sì                         \\
		TS13              & RFO3                   & Si verifichi che l'utente possa modificare i metadati associati al dataset                                                & Sì                         \\
		TS14              & RFO3.1                 & Si verifichi che l'utente possa selezionare di quale dimensione desidera modificare i metadati                            & Sì                         \\
		TS15              & RFO3.2                 & Si verifichi che l'utente possa modificare i metadati di tipo associati al dataset                                        & Sì                         \\
		TS16              & RFO3.3                 & Si verifichi che l'utente possa modificare i metadati di visibilità delle dimensioni del dataset                          & Sì                         \\
		TS17              & RFO4.5.2               & Si verifichi che l'utente possa ordinare la Distance Map                                                                  & Sì                         \\
		TS18              & RFO4.5.3.1             & Si verifichi che l'utente possa ordinare la Distance Map mediante clustering gerarchico                                   & Sì                         \\
		TS19              & RFO4.5.3.2             & Si verifichi che l'utente possa associare un dendrogramma al clustering gerarchico nella Distance Map                     & Sì                         \\
		TS20              & RFO5                   & Si verifichi che l'utente possa essere informato in caso di errori durante l'inserimento di un file                       & Sì                         \\
		TS21              & RFO6                   & Si verifichi che l'utente possa essere informato in caso di errore durante l'acceso al database                           & Sì                         \\
		TS22              & RFO7                   & Si verifichi che l'utente possa essere informato in caso la query usata per recuperare i dati non ritorni nessun dato     & Sì                         \\
		TS23              & RFO8                   & Si verifichi che l'utente possa essere informato in caso provi a modificare la visibilità di un dato quando non può farlo & Sì                         \\
		TS24              & RVO4                   & Si verifichi che l'applicativo possa visualizzare dati ad almeno 15 dimensioni                                            & Sì                         \\
		TS25              & RVO5                   & Si verifichi che l'applicativo sia supportato dal browser Google Chrome dalla versione 87.0.0                             & Sì                         \\
		TS26              & RVO6                   & Si verifichi che l'applicativo sia supportato dal browser Mozilla Firefox dalla versione 85.0.0                           & Sì                         \\
		TS27              & RVO7                   & Si verifichi che l'applicativo sia supportato dal browser Opera dalla versione 74.0.0                                     & Sì                         \\
		TS28              & RVO8                   & Si verifichi che l'applicativo sia supportato dal browser Safari, nella sua versione per Mac, dalla versione 14.0.0       & Sì                         \\
		\hline
		\rowcolor{white}
		\caption{Riepilogo dei test di sistema}
	\end{longtable}
\end{center}

\subsection{Test di integrazione}%
\label{sub:test_int}

\begin{center}
	\begin{longtable}{|c|p{10cm}|c|}
		\hline
		\rowcolor{lightgray}
		{\textbf{Codice}} & {\textbf{Descrizione}}                                                                                                                   & {\textbf{Implementazione}} \\
		\hline
		\endfirsthead
		\hline
		\rowcolor{lightgray}
		{\textbf{Codice}} & {\textbf{Descrizione}}                                                                                                                   & {\textbf{Implementazione}} \\
		\hline
		\endhead

		\hline
		\rowcolor{white}
		\multicolumn{3}{|c|}{\emph{Continua alla pagina successiva...}}                                                                                                                           \\
		\hline
		\endfoot
		\endlastfoot


		TI1               & Viene verificata la corretta integrazione tra il modulo di caricamento dei file e il server                                              & Sì                         \\
		TI2               & Viene verificata la corretta integrazione tra il database e il server                                                                    & Sì                         \\
		TI3               & Viene verificata la corretta integrazione tra i moduli responsabili della gestione dei grafici e la libreria D3.js                       & Sì                         \\
		TI4               & Viene verificata la corretta integrazione tra i dati letti da file e la loro manipolazione con la libreria D3.js                         & Sì                         \\
		TI5               & Viene verificata la corretta integrazione tra i dati letti da database e la loro manipolazione con la libreria D3.js                     & Sì                         \\
		TI6               & Viene verificata la corretta integrazione tra le componenti GUI e la logica del modello                                                  & Sì                         \\
		TI7               & Viene verificata la corretta integrazione tra la persistenza della sessione nel browser e la gestione della sessione da parte del server & Sì                         \\
		TI8               & Viene verificata la corretta integrazione dei moduli presenter e observer dei vari grafici                                               & Sì                         \\
		TI9               & Viene verificata la corretta integrazione della GUI con i vari grafici                                                                   & Sì                         \\
		TI10              & Viene verificata la corretta integrazione con il server e il cambio di file da GUI                                                       & Sì                         \\

		\hline
		\rowcolor{white}
		\caption{Riepilogo dei test di sistema}
	\end{longtable}
\end{center}

\subsection{Test di unità}%
\label{sub:test_unit}

\begin{center}
	\begin{longtable}{|c|p{10cm}|c|}
		\hline
		\rowcolor{lightgray}
		{\textbf{Codice}} & {\textbf{Descrizione}}                                                                  & {\textbf{Implementazione}} \\
		\hline
		\endfirsthead
		\hline
		\rowcolor{lightgray}
		{\textbf{Codice}} & {\textbf{Descrizione}}                                                                  & {\textbf{Implementazione}} \\
		\hline
		\endhead

		\hline
		\rowcolor{white}
		\multicolumn{3}{|c|}{\emph{Continua alla pagina successiva...}}                                                                          \\
		\hline
		\endfoot
		\endlastfoot

		TU1               & Viene verificato che il pulsante di caricamento chiama la funzione corretta             & Sì                         \\
		TU2               & Viene verificato il corretto funzionamento del bottone di invio del file                & Sì                         \\
		TU3               & Viene verificato che sono i file corretti possono essere caricati                       & Sì                         \\
		TU4               & Viene verificato che il caricamento del file avviene correttamente                      & Sì                         \\
		TU5               & Viene verificato il corretto funzionamento dei bottoni di selezione del grafico         & Sì                         \\
		TU6               & Viene verificata la possibilità di cambiare tipologia al grafico                        & Sì                         \\
		TU7               & Viene verificato il numero dei dati letti dal file                                      & Sì                         \\
		TU8               & Viene verificato che il menù delle dimensioni contiene solo le dimensioni corrette      & Sì                         \\
		TU9               & Viene verificata la possibilità di deselezionare le dimensioni                          & Sì                         \\
		TU10              & Viene verificata la possibilità di selezionare al massimo 5 dimensioni (Scatter Plot)   & Sì                         \\
		TU11              & Viene verificato che le dimensioni plottate siano effettivamente quelle selezionate     & Sì                         \\
		TU12              & Viene verificato che cambiando le dimensioni da plottare cambino anche nei grafici      & Sì                         \\
		TU13              & Viene verificata la possibilità di selezionare le dimensioni da colorare                & Sì                         \\
		TU14              & Viene verificato che la selezione dei colori funziona correttamente                     & Sì                         \\
		TU15              & Viene verificato che funzioni correttamente la selezione dei gradienti                  & Sì                         \\
		TU16              & Viene verificato che la selezione della brillanza funzioni correttamente                & Sì                         \\
		TU17              & Viene verificato il corretto funzionamento della selezione della brillanza del grafico  & Sì                         \\
		TU18              & Viene verificato che gli algoritmi di distanza vengano visualizzati correttamente       & Sì                         \\
		TU19              & Viene verificato che gli algoritmi di distanza modifichino correttamente il grafico     & Sì                         \\
		TU20              & Viene verificato che la selezione degli algoritmi di ordinamento funzioni correttamente & Sì                         \\
		TU21              & Viene verificata la persistenza della sessione                                          & Sì                         \\
		TU22              & Viene verificato il corretto collegamento al database                                   & Sì                         \\
		TU23              & Viene verificato che solo i file con una certa dimensione possano essere caricati       & Sì                         \\
		TU24              & Viene verificato il corretto funzionamento degli alert                                  & Sì                         \\
		%TU25              & Longdescsomuchlongdescr  & Sì\\
		%TU26              & Longdescsomuchlongdescr& Sì\\
		%TU27              & Longdescsomuchlongdescr& Sì\\
		%TU28              & Longdescsomuchlongdescr& Sì\\
		%TU29              & Longdescsomuchlongdescr& Sì\\
		%TU30              & Longdescsomuchlongdescr& Sì\\
		\hline
		\rowcolor{white}
		\caption{Riepilogo dei test di sistema}
	\end{longtable}
\end{center}

\end{document}
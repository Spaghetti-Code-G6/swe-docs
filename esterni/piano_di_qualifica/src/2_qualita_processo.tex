\documentclass[../piano_di_qualifica.tex]{subfiles}

\begin{document}

\subsection{Scopo}
In questa sezione saranno presentati gli obiettivi di qualità che si vogliono raggiungere relativi ai processi implementati e ai loro prodotti, così da rendere efficace la valutazione della qualità e di selezionare le migliori metriche.\\
La qualità di un prodotto è fortemente influenzata dai processi utilizzati, per questo motivo si è deciso di usare come riferimento lo standard ISO/IEC 15504 (SPICE)  e il metodo PDCA (o \glossario{ciclo di Deming}) per il miglioramento continuo della qualità durante tutto l'arco del progetto.\\
Per ogni processo sono elencate le sue funzioni principali, gli obiettivi prefissati e le metriche adottate per raggiungere quell’obiettivo.

\subsubsection{PROC01 Gestione progetto}
Lo scopo di questo macro-processo è quello di pianificare ed organizzare il lavoro da svolgere per adempiere ai requisiti trovati. L'esito dell'intero progetto è dipendente da questo processo, che deve strutturare il \glossario{way of working} del gruppo.\\

\setlength{\parindent}{0pt}\textbf{Funzioni:}
\smallbreak
\begin{itemize}
	\item \textbf{Scelta standard}: trovare gli standard che si adattano meglio allo sviluppo di questo progetto;
	\item \textbf{Sviluppo sotto-processi}: i processi devo poter essere associati ad azioni mirate;
	\item \textbf{Suddivisione lavoro}: assegnazione di compiti ai membri del gruppo;
	\item \textbf{Programmare scadenze}: stabilire delle baseline;
	\item \textbf{Programmare formazione}: prevedere dei periodi di autoformazione;
	\item \textbf{Rispettare budget}: rimanere nei limiti di budget stimati durante il preventivo.
\end{itemize}

\textbf{Obiettivi:}
\smallbreak
\begin{itemize}
	\item \textbf{OPR01 Scadenze}: rispettare il più possibile le date di scadenza presenti all'interno del \textsc{Piano di Progetto};
	\item \textbf{OPR02 Budget parziale}: mantenere le risorse messe a disposizione ad inizio progetto per tutta la sua durata; riuscire a contabilizzare dei consuntivi coerenti con i preventivi permetterà di ridurre al minimo l'oscillazione del prezzo finale;
	\item \textbf{OPR03 Budget totale}: mantenere le risorse messe a disposizione ad inizio progetto per tutta la sua durata al fine di non avere oscillazioni sul prezzo finale;
	\item \textbf{OPR04 Versionamento}: i prodotti devono essere versionati per poterne conoscere lo storico delle modifiche;
\end{itemize}

\textbf{Metriche:}
\smallbreak
\begin{itemize}
	\item \textbf{MPR01 Varianza della pianificazione}: i documenti non devono essere completati oltre i 4 giorni di ritardo preventivati;
	\item \textbf{MPR02 Varianza dei costi parziali}: il budget di ogni step di progetto deve mantenersi entro il $\pm$5\% del budget prefissato;
	\item \textbf{MPR03 Varianza dei costi totali}: il budget totale non deve essere superato;
	\item \textbf{MPR04 Versionamento}: i commit di una repository permettono di tenere una miglior traccia delle modifiche e di accedere facilmente all’ultima versione del prodotto. In base a sondaggi presenti in Internet, abbiamo constatato che un giusto numero di commit da effettuare in media durante una settimana produttiva è almeno 25;
\end{itemize}

\subsubsection{PROC02 Analisi}
Questo macro-processo si riferisce ad ogni tipo di analisi presente in questo progetto, non solo l'analisi dei requisiti.\\

\textbf{Funzioni:}
\smallbreak
\begin{itemize}
	\item \textbf{Individuare requisiti}: individuare i requisiti espliciti e impliciti e classificarli;
	\item \textbf{Individuare rischi}: individuare i rischi e classificarli per poterli evitare o per mitigare i loro effetti;
	\item \textbf{Individuare obiettivi}: individuare gli obiettivi per poter perseguire la qualità nel progetto;
	\item \textbf{Individuare norme}: trovare delle norme e poi adattarle per migliorare il lavoro del gruppo.
\end{itemize}

\textbf{Obiettivi:}
\smallbreak
\begin{itemize}
	\item \textbf{OPR05 Adempimento requisiti obbligatori}: tutti i requisiti obbligatori devono essere soddisfatti;
	\item \textbf{OPR06 Adempimento requisiti non obbligatori}: i requisiti non obbligatori, cioè desiderabili e opzionali, potranno essere soddisfatti solo se tutti i requisiti obbligatori sono stati completati prima della fine del progetto;
	\item \textbf{OPR07 Rischi non previsti}: il verificarsi di imprevisti può accadere nel corso del progetto, ma devono poter essere rilevabili prima che succeda;
	\item \textbf{OPR08 Rispetto obiettivi}: tutti gli obiettivi dovranno essere rispettati per avere un prodotto di qualità;
	\item \textbf{OPR09 Rispetto delle norme}: tutte le norme dovranno essere seguite; se ci fosse necessità di fare diversamente, dovranno essere riadattate.
\end{itemize}

\textbf{Metriche:}
\smallbreak
\begin{itemize}
	\item \textbf{MPR05 Requisiti obbligatori}: sono i requisiti minimi;
	\item \textbf{MPR06 Requisiti non obbligatori desiderabili}: sono i requisiti opzionali;
	\item \textbf{MPR07 Rischi non previsti}: sono i rischi che si sono verificati e che non erano stati previsti in fase di pianificazione;
	\item \textbf{MPR08 Obiettivi soddisfatti}: ovvero se sono stati soddisfatti tutti gli obiettivi pianificati entro la consegna;
	\item \textbf{MPR09 Norme rispettate}: ovvero se tutte le norme stabilite nel documento \textsc{Norme di Progetto} sono state rispettate;
\end{itemize}

Nello specifico il calcolo usato per verificare la percentuale di requisiti e obiettivi soddisfatti sarà il seguente: \par
\begin{center}
	$C = \frac{N_{fni}}{N_{fi}} * 100$
\end{center}

dove
\begin{enumerate}
	\item N\textsubscript{FnI} è il numero di funzionalità non implementate;
	\item N\textsubscript{FI} è il numero di funzionalità implementate.
\end{enumerate}

%\subsubsection{PROC03 Documentazione}
%Questo processo ha lo scopo di garantire la produzione di documenti di qualità. Di seguito, si riportano le scelte effetturate, gli strumenti utilizzati e le modifiche attuate durante l'intero progetto.\\
%
%\textbf{Funzioni:}
%\smallbreak
%\begin{itemize}
%	\item \textbf{Verifica documenti}: controllo dei documenti da parte dei verificatori;
%	\item \textbf{Approvazione documenti}: controllo e approvazione finale da parte del responsabile.
%\end{itemize}
%
%\textbf{Obiettivi:}
%\smallbreak
%\begin{itemize}
%	\item \textbf{QPR10 Ciclo di vita dei documenti}: i documenti devono rispettare ogni fase del loro ciclo di vita e le scadenze prefissate;
%	\item \textbf{QPR11 Leggibilità}: i documenti devono poter essere letti e capiti da chiunque abbia almeno una licenza media.
%\end{itemize}

\subsubsection{PROC03 Verifica}
Questo processo ha lo scopo di valutare i prodotti per stabilire se presentano errori, se rispettano gli obiettivi di qualità e se sono corretti nella loro forma e contenuto. \\

\textbf{Funzioni:}
\smallbreak
\begin{itemize}
	\item \textbf{Verificare funzionalità}:  i prodotti devono rispettare i requisiti prefissati e quindi bisogna controllare che il loro output sia quello atteso;
	\item \textbf{Verifica processi}:  controllare che i processi rispettino il loro ciclo di vita;
	\item \textbf{Verifica qualità}:  bisogna verificare che le metriche degli obiettivi di qualità siano entro limiti accettabili;
	\item \textbf{Verifica rispetto norme}: bisogna verificare costantemente che le norme prefissate siano sempre rispettate.
\end{itemize}

\textbf{Obiettivi:}
\smallbreak
\begin{itemize}
	\item \textbf{OPR10 Verifica costante}: tutti i prodotti devono essere costantemente sotto verifica e le verifiche devono essere sempre le medesime, per avere consistenza con i risultati e poter apportare miglioramenti di qualità.
\end{itemize}

\textbf{Metriche:}
\smallbreak
\begin{itemize}
	\item \textbf{MPR10 Frequenza di controllo}: è la frequenza con cui viene effettuata la verifica dei documenti o Issue portate a termine.
\end{itemize}

\subsection{Tabella qualità di processo}
La seguente tabella indica gli obiettivi di qualità che i processi devono possedere.\\
Ogni obiettivo di qualità è indicato con:
\smallbreak
\begin{itemize}
	\item \textbf{Obiettivo}: contrassegnato dal suo codice identificativo;
	\item \textbf{Metrica}: contrassegnato dal suo codice identificativo, indica come sarà garantita la qualità dell'obiettivo;
	\item \textbf{Accettabile}: indica il valore sotto il quale non sarà garantita la qualità;
	\item \textbf{Preferibile}: indica il valore a cui si punta per avere la massima qualità;
\end{itemize}

\begin{table}[!ht]
	\centering
	\begin{tabular}{|p{5.5cm}|p{4.5cm}|c|c|}
		\hline
		\rowcolor{lightgray}
		\textbf{Obiettivo}                          & \textbf{Metrica}                    & \textbf{Accettabile} & \textbf{Preferibile} \\
		\hline
		OPR01 Scadenze                              & MPR01 Varianza della pianificazione 	& 4 giorni             	& 0 giorni             	\\
		OPR02 Budget parziale                       & MPR02 Varianza dei costi parziali   	& $\pm$5\%             	& 0\%                  	\\
		OPR03 Budget totale							& MPR03 Varianza dei costi totali	  	& 0						& 0						\\
		OPR04 Versionamento							& MPR04 Versionamento					& 25					& $\sim$25				\\
		OPR05 Adempimento requisiti obbligatori     & MPR05 Requisiti obbligatori         	& 100\%                	& 100\%                	\\
		OPR06 Adempimento requisiti non obbligatori & MPR06 Requisiti non obbligatori     	& da definire          	& 100\%                	\\
		OPR07 Rischi non previsti 					& MPR07 Rischi inattesi					& 2						& 0						\\
		OPR08 Rispetto obiettivi                    & MPR08 Obiettivi soddisfatti         	& 100\%                	& 100\%                	\\
		OPR09 Rispetto delle norme                  & MPR09 Norme rispettate              	& 100\%                	& 100\%                	\\
		OPR10 Verifica costante                     & MPR10 Frequenza di controllo        	& Ad ogni Milestone    	& Ad ogni modifica     	\\
		\hline
	\end{tabular}
	\caption{Metriche qualità di processo}
\end{table}

\clearpage

\end{document}

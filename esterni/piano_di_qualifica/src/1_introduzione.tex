\documentclass[../piano_di_qualifica.tex]{subfiles}
\begin{document}

\subsection{Scopo del documento}
Il documento ha come scopo quello di presentare i metodi di \glossario{verifica} e \glossario{validazione} adottati per garantire la qualità del prodotto e del processo durante tutta la durata del progetto. 
Per poter garantire ciò verrà usato un sistema di verifica continua, permettendo l'individuazione di eventuali problematiche in modo automatico e facile. Inoltre tramite essa sarà possibile risolvere tali problematiche rapidamente, ottimizzando al massimo le risorse tempo.

Attraverso l’impiego di metriche empiriche riproducibili sarà possibile ottenere risultati quantificabili, oggettivi e misurabili. Infatti, questo documento potrà essere utilizzato dal committente per verificare il lavoro svolto dal gruppo.

Questo documento verrà redatto seguendo una \glossario{filosofia incrementale}: i contenuti iniziali sono da considerarsi incompleti, verranno sottoposti a significative modifiche \glossario{just in time} durante lo svolgimento del progetto.

\subsection{Scopo del prodotto}
Il capitolato C4 - \emph{HD Viz} nasce dalla necessità di trasformare grosse moli di dati multidimensionali in grafici che diano la possibilità di interpretare le informazioni o apprenderne di nuove. Il gruppo \emph{SpaghettiCode} si offre quindi di sviluppare la web-application commissionata dall’azienda \emph{Zucchetti S.p.A.} seguendo le tecnologie richieste dal proponente.

\subsection{Glossario}
All'interno del documento ci sono termini che potrebbero presentare significati ambigui o incongruenti a seconda del contesto. Per evitare incomprensioni viene fornito un documento \textsc{Glossario} contenente la spiegazione dei termini. Queste parole sono contrassegnate con una "G" a pedice ad ogni prima occorrenza del termine.

\subsection{Riferimenti}
\label{sub:riferimenti}

\subsubsection{Riferimenti normativi}
\begin{itemize}
	\item Norme di Progetto v2.3.0;
	\item Regolamento organigramma e specifica tecnico-economica: \url{https://www.math.unipd.it/~tullio/IS-1/2020/Progetto/RO.html};
	\item Capitolato d’appalto C4: \url{https://www.math.unipd.it/~tullio/IS-1/2020/Progetto/C4.pdf};
	\item Valutazione esito RR: \url{https://www.math.unipd.it/~tullio/IS-1/2020/Progetto/RR/SpaghettiCode.pdf};
	\item Valutazione esito RP: \url{https://www.math.unipd.it/~tullio/IS-1/2020/Progetto/RP/Spaghetti_Code.pdf};
	\item Verbale Interno 2021-02-12.
\end{itemize}

\subsubsection{Riferimenti informativi}

\begin{itemize}
	\item ISO/IEC 9126: \url{https://en.wikipedia.org/wiki/ISO/IEC_9126}
	\item Slide del corso di Ingegneria del Software, gestione di progetto: \url{https://www.math.unipd.it/~tullio/IS-1/2020/Dispense/L06.pdf};
	\item Slide del corso di Ingegneria del Software, qualità software: \url{https://www.math.unipd.it/~tullio/IS-1/2020/Dispense/L12.pdf};
	\item Slide del corso di Ingegneria del Software, qualità di processo: \url{https://www.math.unipd.it/~tullio/IS-1/2020/Dispense/L13.pdf};
	\item Slide del corso di Ingegneria del Software, verifica e validazione \url{https://www.math.unipd.it/~tullio/IS-1/2020/Dispense/L14.pdf};
	\item Indice Gulpease: \url{https://it.wikipedia.org/wiki/Indice_Gulpease}.
\end{itemize}

\end{document}
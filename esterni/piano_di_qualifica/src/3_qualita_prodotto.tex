\documentclass[../piano_di_qualifica.tex]{subfiles}

\begin{document}

Per garantire la qualità del prodotto il gruppo ha deciso di fare riferimento allo \glossario{standard ISO/IEC 9126}, il quale regolamenta le modalità con cui produrre un software di buona qualità. Di questo standard il gruppo sceglie di seguire alcune delle metriche esposte, e sceglie di stabilirne altre da seguire per rendere il prodotto qualitativamente valido; qui di seguito vengono elencate.

\subsection{Modello di qualità}
Il gruppo intende seguire a pieno le caratteristiche descritte nello standard adottato circa il modello di qualità.\par
Il prodotto che \emph{SpaghettiCode} intende rilasciare deve essere:\par
\begin{itemize}
	\item \textbf{Funzionale}: è la capacità di un prodotto software di fornire funzioni che soddisfino esigenze stabilite, necessarie per operare sotto condizioni specifiche.
	      \begin{itemize}
		      \item Il prodotto software sarà in grado di fornire un appropriato insieme di funzioni per i compiti prefissati dall’utente; sarà in grado di fornire i risultati concordati con il proponente; sarà in grado di interagire ed operare con più sistemi specificati dal proponente; se richiesto dal proponente, sarà in grado di proteggere informazioni negando accessi non autorizzati.
	      \end{itemize}
	\item \textbf{Affidabile}: è la capacità del prodotto software di mantenere uno specificato livello di prestazioni quando usato in date condizioni e per un dato periodo.
	      \begin{itemize}
		      \item Il prodotto software sarà in grado di evitare errori e malfunzionamenti; in caso questi dovessero presentarsi sarà possibile recuperare i dati su cui si stava lavorando; sarà in grado di aderire agli standard definiti con il proponente.
	      \end{itemize}
	\item \textbf{Efficiente}: è la capacità di fornire appropriate prestazioni relative alla quantità di risorse usate.
	      \begin{itemize}
		      \item Il prodotto software sarà in grado di fornire adeguati tempi di risposta, elaborazione e velocità di attraversamento; sarà in grado di sfruttare adeguatamente le risorse e sarà efficiente.
	      \end{itemize}
	\item \textbf{Usabile}: è la capacità del prodotto software di essere capito, appreso e usato dall'utente.
	      \begin{itemize}
		      \item Il prodotto software sarà di facile comprensione per l’utente; assieme ad esso verrà consegnato il manuale d’utente per consentire a chiunque il suo utilizzo.
	      \end{itemize}
	\item \textbf{Mantenibile}: è la capacità del software di essere modificato, includendo correzioni, miglioramenti o adattamenti.
	      \begin{itemize}
		      \item Il prodotto software sarà in grado di essere analizzato lato codice; sarà in grado di essere modificato e soggetto ad evoluzioni; sarà facilmente testabile per verificare le modifiche apportate.
	      \end{itemize}
	\item \textbf{Portabile}: è la capacità del software di essere trasportato da un ambiente di lavoro ad un altro.
	      \begin{itemize}
		      \item il prodotto software sarà in grado di essere adattato a diversi ambienti sulla base degli accordi presi con il proponente.
	      \end{itemize}
\end{itemize}

\subsection{Prodotti}
Per prodotto si intende ciò che è concretamente utilizzabile, consultabile o eseguibile. In questo caso si tratta dei documenti e del software.

\subsubsection{Documenti}
Come trattato nel precedente capitolo i documenti devono essere letti e capiti da chiunque abbia almeno la licenza media.

Gli obiettivi per un documento sono quindi:

\setlength{\parindent}{0pt}\textbf{Obiettivi:}
\smallbreak
\begin{itemize}
	\item \textbf{OPD01 Leggibilità del testo}: i documenti devono essere letti in modo fluido evitando periodi troppo lunghi;
	\item \textbf{OPD02 Correttezza ortografica}: i documenti non devono presentare errori.
\end{itemize}

\textbf{Metriche:}
\smallbreak
\begin{itemize}
	\item \textbf{MPD01 Indice di Gulpease};
	\item \textbf{MPD02 Errori ortografici}.
\end{itemize}

Per verificare questi requisiti si farà ricorso all’ \glossario{indice di Gulpease}. L'Indice Gulpease è un indice di leggibilità di un testo, è tarato sulla lingua italiana, e considera due variabili linguistiche: la lunghezza della parola e la lunghezza della frase rispetto al numero delle lettere.\\
La formula è la seguente:\par
\begin{center}
	$G = 89\ +\ \frac{300 * (numero\ di\ frasi) - 10 * (numero\ di\ lettere)}{numero\ di\ parole} $
\end{center}
Come descritto nella seguente tabella, ci aspettiamo che i documenti siano entro uno specifico range di valori e che siano corretti, quindi che la quantità di errori presenti sia 0. \par

\begin{table}[!ht]
	\centering
	\begin{tabular}{|c|c|c|}
		\hline
		\rowcolor{lightgray}
		\textbf{ID metrica}      & \textbf{Accettabile} & \textbf{ Ottimale} \\
		\hline
		MPD01 Indice di Gulpease & \(\ge 60\)           & \(\ge 80\)         \\
		MPD02 Errori ortografici & \(= 0\)              & \(= 0\)            \\
		\hline
	\end{tabular}
	\caption{Indici di qualità per le metriche di comprensione del prodotto}
\end{table}


\subsubsection{Software}
Il software rilasciato deve essere di qualità, e per renderlo tale si farà riferimento al \emph{Modello di qualità} descritto sopra.

Di seguito presentiamo alcune delle metriche che intendiamo adottare.

\textbf{Obiettivi:}
\smallbreak
\begin{itemize}
	\item \textbf{OPS01 Assenza di bug}: il prodotto non deve presentare bug;
	\item \textbf{OPS02 Assenza di errori}: il prodotto deve presentare meno errori possibile;
	\item \textbf{OPS03 Assenza di file senza intestazione}: non ci devono essere file senza intestazione;
	\item \textbf{OPS04 Usabilità del prodotto}: rappresentata tramite il numero di click necessari per arrivare al contenuto desiderato dall'utente;
	\item \textbf{OPS05 Superamento test}: i test ai quali il codice viene sottoposto;
	\item \textbf{OPS06 Copertura test}: il codice deve essere completamente verificato dai test;
	\item \textbf{OPS07 Manutenibilità codice}: il codice deve essere agevolmente manutenibile;
	\item \textbf{OPS08 Leggibilità}: il codice deve essere comprensibile.
\end{itemize}

\textbf{Metriche:}
\smallbreak
\begin{itemize}
	\item \textbf{MPS01 Presenza di bug};
	\item \textbf{MPS02 Densità di errori}: l'abilità del prodotto di resistere a malfunzionamenti. Si calcola così:
	      \begin{center} $M = \frac{N_{er}}{N_{te}} * 100$ \end{center}
	      dove
	      \subitem N\textsubscript{ER} indica il numero di errori rilevati durante il testing;
	      \subitem N\textsubscript{TE} indica il numero di test effettuati;
	\item \textbf{MPS03 File senza intestazione};
	\item \textbf{MPS04 Usabilità del prodotto};
	\item \textbf{MPS05 Test superati};
	\item \textbf{MPS06 Line coverage};
	\item \textbf{MPS07 Densità duplicazione codice};
	\item \textbf{MPS08 Rapporto tra righe di commento e righe di codice}.
\end{itemize}

\subsection{Tabella riassuntiva degli obiettivi di qualità di prodotto}
La seguente tabella indica gli obiettivi di qualità che i prodotti dovranno possedere e le metriche ad essi associate con i relativi valori accettabili e ottimali.

\begin{center}
	\begin{longtable}{|p{5cm}|p{6cm}|c|c|}
		\hline
		\rowcolor{lightgray}
		\textbf{Obiettivo qualità prodotto}   & \textbf{Metrica associata}                             & \textbf{Accettabile} & \textbf{Ottimale} \\
		\hline
		\endfirsthead
		\hline
		\rowcolor{lightgray}
		\textbf{Obiettivo qualità prodotto}   & \textbf{Metrica associata}                             & \textbf{Accettabile} & \textbf{Ottimale} \\
		\hline
		\endhead

		\hline
		\rowcolor{white}
		\multicolumn{4}{|c|}{\emph{Continua alla pagina successiva...}}                                                                           \\
		\hline
		\endfoot
		\endlastfoot
		\rowcolor{lightgray} \multicolumn{4}{|c|}{Documenti}                                                                                      \\
		\hline
		OPD01 Leggibilità del testo           & MPD01 Indice di Gulpease                               & \(\ge 60\)           & \(\ge 80\)        \\
		OPD02 Correttezza ortografica         & MPD02 Errori ortografici                               & 0                    & 0                 \\
		\hline \rowcolor{lightgray} \multicolumn{4}{|c|}{Software}                                                                                \\
		\hline
		OPS01 Assenza di bug                  & MPS01 Presenza di bug                                  & 0                    & 0                 \\
		OPS02 Assenza di errori               & MPS02 Densità di errori                                & 10\%                 & 0\%               \\
		OPS03 Assenza file senza intestazione & MPS03 File senza intestazione                          & 0                    & 0                 \\
		OPS04 Usabilità del prodotto          & MPS04 Usabilità del prodotto                           & \(\leq10\) click     & \(\leq6\) click   \\
		OPS05 Superamento test                & MPS05 Test superati                                    & \(\ge 80\%\)         & 100\%             \\
		OPS06 Copertura test                  & MPS06 Line coverage                                    & \(\ge 80\%\)         & 100\%             \\
		OPS07 Manutenibilità codice           & MPS07 Densità duplicazione codice                      & \(\leq 10\%\)        & 0\%               \\
		OPS08 Leggibilità                     & MPS08 Rapporto tra righe di commento e righe di codice & \(\leq 10\%\)        & \(\leq 20\%\)     \\
		\hline
		\rowcolor{white}
		\caption{Obiettivi di qualità di prodotto}
	\end{longtable}
\end{center}

\end{document}
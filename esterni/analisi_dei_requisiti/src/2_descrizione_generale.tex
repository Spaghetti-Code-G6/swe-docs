\documentclass[../analisi-dei-requisiti.tex]{subfiles}

\begin{document}

\subsection{Caratteristiche del prodotto}
\label{sub:caratteristiche_del_prodotto}
Il progetto \emph{HD Viz} mette a disposizione diverse funzionalità per la visualizzazione e l'esplorazione dei
dati. L'applicativo offre una serie di servizi tra cui la possibilità di importare dati mediante file CSV o attraverso query da database esterno; la possibilità di scegliere e visualizzare diverse tipologie di grafici e strumenti per la manipolazione delle visualizzazioni.

In seguito al caricamento dei dati e alla scelta della visualizzazione, l'utente può esplorare e manipolare il grafico
al fine di analizzare i dati e scovare schemi intrinseci nella loro natura.

La tipologia della visualizzazione inoltre può essere cambiata in corso d'opera per mettere in risalto aspetti diversi
dei dati caricati, così da poter evidenziare situazioni interessanti che altrimenti non si sarebbero presentate.

\subsection{Obiettivi del prodotto}
\label{sub:obiettivo_del_prodotto}
L'obiettivo del progetto presentato è la realizzazione di una web application che permetta la visualizzazione di dati a
molte dimensioni mediante diverse tipologie di grafici.
I dati potranno essere caricati tramite file CSV o query da database esterno, mentre le visualizzazioni messe a
disposizione saranno l'elemento cardine dell'applicativo; esse andranno infatti a fornire supporto all'utente
nella fase esplorativa dell'analisi dei dati (\glossario{EDA}), facilitando così la visualizzazione di schemi e modelli
altrimenti difficilmente individuabili.

\subsection{Caratteristiche degli utenti}
\label{sub:caratteristiche_degli_utenti}

Il prodotto é rivolto ad un'utenza che possieda dati ad alto numero dimensionale in formato CSV o all'interno di un
database, e desideri esplorarne la conformazione e la struttura. Per poter utilizzare l'applicazione \emph{HD Viz} gli
utenti non necessiteranno di autenticazione né di registrazione.
% TODO: Check
% Agli utenti viene fornita una breve guida introduttiva per orientarsi nell'interfaccia e per usare al meglio i vari strumenti offerti da \emph{HD Viz}.

% Non fa parte dell'analisi dei requisiti ma della Progettazione
% \subsection{Architetture del progetto}
% \label{sub:architetture_e_tecnologie_del_progetto}

% Il back end verrà sviluppato in JavaScript con l'ausilio del framework \glossario{Node.js}, inoltre l'applicazione
% immagazzinerà i dati caricati in un database all'interno del server. \\
% Il \glossario{front end} sarà costituito da diverse pagine web accessibili dai principali \glossario{browser} desktop. \\
% La \glossario{UI} verrà sviluppata principalmente in \glossario{HTML} e \glossario{CSS}, verranno inoltre aggiunti elementi dinamici in JavaScript. I grafici che verranno visualizzati saranno sviluppati con JavaScript utilizzando la libreria D3.js.
% Sarà messa a disposizione dell'utente una piccola guida introduttiva per orientarsi e usare al meglio l'applicazione.

\subsection{Vincoli progettuali}
\label{sub:vincoli_progettuali}

L'implementazione del progetto dovrà rispettare i seguenti vincoli obbligatori specificati nel documento
\textsc{Capitolato d'Appalto C4 - HD Viz}, reperibile al link indicato nella sezione \refSec{ssub:normativi}:
\begin{itemize}
    \item Il lato \glossario{front end} dell'applicazione dovrà essere sviluppato prevalentemente con le tecnologie
    \glossario{HTML}, \glossario{CSS} e JavaScript, e dovrà utilizzare la libreria D3.js, per la
    visualizzazione dei dati;
    \item Il back end dell'applicazione dovrà essere sviluppato in Java con server Tomcat,
    oppure in JavaScript con l'ausilio del framework Node.js
    \item I dati dovranno essere forniti al sistema tramite caricamento di file CSV o tramite query da database
    esterno, inoltre essi dovranno poter arrivare ad almeno 15 dimensioni;
    \item Dovranno essere disponibili le seguenti tipologie di visualizzazione:
    \begin{itemize}
        \item Scatter Plot Matrix: disposizione matriciale di \glossario{Scatter Plot} dove in ciascuno di essi compare
        sugli assi una differente coppia di dimensioni;
        \item Force Field: visualizzazione che traduce le distanze tra punti nello spazio multidimensionale in forze
        di attrazione e repulsione proiettate in uno spazio bidimensionale;
        \item Heat Map: visualizzazione che trasforma la distanza tra coppie di punti nello spazio multidimensionale in
        colori di varia intensità. In questo grafico dovrà inoltre essere possibile svolgere l'ordinamento dei dati in
        modo che le strutture presenti siano più visibili all'utente;
        \item Proiezione Lineare Multi Asse: rappresentazione che dispone i punti dello spazio multidimensionale nel
        piano cartesiano e permette all'utente di spostare degli assi delle diverse dimensioni sul piano, in modo da
        favorire l'individuazione di strutture e di raggruppamenti.
    \end{itemize}
\end{itemize}
Inoltre sono stati presentati i seguenti requisiti opzionali:
\begin{itemize}
    \item La possibilità di visualizzare altre tipologie di grafici adatte alla visualizzazione di dati
    pluridimensionali;
    \item La possibilità di utilizzare ulteriori funzioni di calcolo della distanza oltre alla
    \glossario{distanza euclidea} in tutte le visualizzazioni che necessitano del concetto di distanza;
    \item La possibilità di scegliere funzioni di forza diverse rispetto a quelle automaticamente previste nel grafico
    Force Field dalla libreria D3.js;
    \item L'implementazione dell'analisi automatica dei dati per dare evidenza a situazioni di particolare interesse;
    \item L'implementazione dell'utilizzo di algoritmi di preparazione del dato di modo da effettuare le dovute
    trasformazioni precedentemente alla visualizzazione stessa.
\end{itemize}

\subsection{Supporto browser}
\label{sub:supporto_browser}

Durante l'incontro con il proponente (fare riferimento a VE\_2020\_12\_17) è stato specificato che il supporto ai browser meno recenti (es IE8, IE9, IE10) non è richiesto, anzi, ci è stato sconsigliato di non perdere tempo nel cercare di rendere il nostro prodotto retrocompatibile con questi software datati.

Il gruppo, quindi, ha scelto di prendere in considerazione una delle ultime versioni del browser di Google, Google Chrome v87.0.4280 (2020-11-17), che è il browser che verrà usato dal nostro porponente e dai nostri utenti. 
Inoltre è stato deciso di supportare le ultime versioni dei browser più diffusi, quindi dei browser Firefox v85.0.2, Opera v74.0.0, e Safari per Mac v14.0.3.

\end{document}

\subsection{Requisiti di funzionalità}
\label{sub:requisiti_di_funzionalita}

\rowcolors{2}{white!80!lightgray!90}{white}
\renewcommand{\arraystretch}{2} %
\begin{longtable}[H]{| >{\raggedright\bfseries}m{20mm} | >{\raggedright}m{90mm} | >{\centering}m{25mm} | >{\centering\arraybackslash}m{30mm}|}

    \hline
    \rowcolor{lightgray}
    \multicolumn{1} {| >{\centering\bfseries}m{20mm}| } {\textbf{Requisito}}
    & \multicolumn{1} {>{\centering}m{90mm}| } {\textbf{Descrizione}}
    & \multicolumn{1} {>{\centering}m{25mm}| } {\textbf{Importanza}}
    & \multicolumn{1} {>{\centering\arraybackslash}m{30mm}| } {\textbf{Fonte}} \\
    \hline
    
    \endfirsthead%
    
    \hline
    \rowcolor{lightgray}
    \multicolumn{1} {>{\centering\bfseries}m{20mm}| } {\textbf{Requisito}}
    & \multicolumn{1} {>{\centering}m{90mm}| } {\textbf{Descrizione}}
    & \multicolumn{1} {>{\centering}m{25mm}| } {\textbf{Importanza}}
    & \multicolumn{1} {>{\centering\arraybackslash}m{30mm}| } {\textbf{Fonte}} \\
    \hline
    
    \endhead%
    
    \hline
    \rowcolor{lightgray!40}
    \multicolumn{4}{|c|}{\textit{Continua alla pagina successiva}} \\
    \hline
    
    \endfoot%
    
    \endlastfoot%

    RFO1
    & L'utente deve poter creare un ambiente di lavoro
    & Obbligatorio
    & \makecell{
        Capitolato \\
        \hyperref[sub:uc1]{UC1}} \\


    RFO1.1
    & L'utente deve poter inserire dati nel sistema
    & Obbligatorio
    & \makecell{
        Capitolato \\
        \hyperref[ssub:uc1.1]{UC1.1}} \\

    RFO1.2
    & L'utente deve poter effettuare l'inserimento dei dati da file CSV
    & Obbligatorio
    & \makecell{
        Capitolato  \\
        \hyperref[ssub:uc1.2]{UC1.2}} \\

    RFO1.3
    & L'utente deve poter effettuare l'inserimento dei dati da database esterno
    & Obbligatorio
    & \makecell{
        Capitolato \\
        \hyperref[ssub:uc1.3]{UC1.3}} \\

    RFO1.3.1
    &   L'utente deve poter aprire un collegamento con un server per poter
        accedere ad uno dei suoi database
    & Obbligatorio
    & \makecell{ Interno \\  \hyperref[par:uc1.3.1]{UC1.3.1}}\\

    RFO1.3.1.1
    &   L'utente deve poter immettere l'indirizzo del server
    & Obbligatorio
    & \makecell{ Interno \\  \hyperref[spar:uc1.3.1.1]{UC1.3.1.1}}\\

    RFO1.3.1.2
    &   L'utente deve poter immettere il nome di accesso al server
    & Obbligatorio
    & \makecell{ Interno \\  \hyperref[spar:uc1.3.1.2]{UC1.3.1.2}}\\

    RFO1.3.1.3
    &   L'utente deve poter immettere la password di accesso al server
    & Obbligatorio
    & \makecell{ Interno \\  \hyperref[spar:uc1.3.1.3]{UC1.3.1.3}}\\

    RFO1.3.2
    &   L'utente deve poter importare i dati nel sistema mediante la ricerca
        su un database tra quelli disponibili
    & Obbligatorio
    & \makecell{ Interno \\ \hyperref[par:uc1.3.2]{UC1.3.2}}\\

    RFO1.4
    &   L'utente deve poter inserire i metadati di tipo delle dimensioni del dataset
    & Obbligatorio
    & \makecell{ Interno \\ \hyperref[ssub:uc1.4]{UC1.4}}\\

    RFO2
    &  L'utente deve poter creare un grafico
    & Obbligatorio
    & \makecell{ Interno \\ \hyperref[sub:uc2]{UC2}}\\

    RFO2.1
    & L'utente deve poter selezionare la costruzione di un grafico di sua scelta
    & Obbligatorio
    & \makecell{ Capitolato \\ \hyperref[ssub:uc2.1]{UC2.1}}\\

    RFO2.2
    & L'utente deve poter selezionare l'opzione di costruzione di un grafico di tipo Scatter Plot
    Matrix
    & Obbligatorio
    & \makecell{ Capitolato \\   \hyperref[ssub:uc2.2]{UC2.2}}\\

    RFO2.3
    & L'utente deve poter selezionare l'opzione di costruzione di un grafico di tipo Force Field
    & Obbligatorio
    & \makecell{ Capitolato \\  \hyperref[ssub:uc2.3]{UC2.3}}\\

    RFO2.4
    & L'utente deve poter selezionare l'opzione di costruzione di un grafico di tipo Heat Map
    & Obbligatorio
    & \makecell{ Interno \\  \hyperref[ssub:uc2.4]{UC2.4}}\\

    RFO2.5
    & L'utente deve poter selezionare l'opzione di costruzione di un grafico di tipo PLMA
    & Obbligatorio
    & \makecell{Capitolato \\ \hyperref[ssub:uc2.5]{UC2.5}}\\

    RFO2.6
    & L'utente deve poter selezionare l'opzione di costruzione di un grafico di tipo Distance Map
    & Obbligatorio
    & \makecell{Capitolato \\ \hyperref[ssub:uc2.6]{UC2.6}}\\

    RFO3
    & L'utente deve poter modificare i metadati associati al dataset
    & Obbligatorio
    & \makecell{ Interno \\  \hyperref[sub:uc3]{UC3} }\\

    RFO3.1
    & L'utente deve poter selezionare di quale dimensione desidera modificare i metadati
    & Obbligatorio
    & \makecell{ Interno \\\hyperref[ssub:uc3.1]{UC3.1} }\\

    RFO3.2
    & L'utente deve poter modificare i metadati di tipo associati al dataset
    & Obbligatorio
    & \makecell{ Interno \\\hyperref[ssub:uc3.2]{UC3.2} }\\

    RFO3.3
    & L'utente deve poter modificare i metadati di visibilità delle dimensioni del dataset
    & Obbligatorio
    & \makecell{ Interno \\  \hyperref[ssub:uc3.3]{UC3.3} }\\

    RFD4
    & L'utente deve poter modificare la visualizzazione del grafico corrente
    & Desiderabile
    & \makecell{ Capitolato \\ \hyperref[sub:uc4]{UC4} }\\

    RFD4.1
    & L'utente deve poter modificare le proprietà del grafico
    & Desiderabile
    & \makecell{ Capitolato \\ \hyperref[ssub:uc4.1]{UC4.1} }\\

    RFD4.2
    & L'utente deve poter modificare le proprietà della visualizzazione Scatter Plot Matrix
    & Desiderabile
    & \makecell{ Capitolato \\ \hyperref[ssub:uc4.2]{UC4.2} }\\

    RFD4.2.1
    & L'utente deve poter modificare il numero di dimensioni di rappresentabili dalla matrice di Scatter Plot
    & Desiderabile
    & \makecell{ Capitolato \\ \hyperref[par:uc4.2.1]{UC4.2.1} }\\

    RFD4.2.2
    & L'utente deve poter aggiungere un'ulteriore dimensione del dato alla matrice di Scatter Plot mediante colore
    & Desiderabile
    & \makecell{ Verbale \\ \hyperref[par:uc4.2.2]{UC4.2.2} }\\

    RFD4.2.3
    & L'utente deve poter aggiungere un'ulteriore dimensione del dato alla matrice di Scatter Plot mediante brillanza
    & Desiderabile
    & \makecell{ Verbale \\ \hyperref[par:uc4.2.3]{UC4.2.3} }\\

    RFD4.2.4
    & L'utente deve poter selezionare un punto in uno Scatter Plot della matrice e visualizzarlo evidenziato negli
    altri Scatter Plot
    & Desiderabile
    & \makecell{ Interno \\ \hyperref[par:uc4.2.4]{UC4.2.4} }\\

    RFD4.2.5
    & L'utente deve poter selezionare un insieme di punti in uno Scatter Plot della matrice e visualizzarlo evidenziato
    negli altri Scatter Plot
    & Desiderabile
    & \makecell{ Interno \\ \hyperref[par:uc4.2.5]{UC4.2.5} }\\

    RFD4.3
    & L'utente deve poter modificare i grafici con la matrice delle distanze
    & Desiderabile
    & \makecell{ Verbale \\ \hyperref[ssub:uc4.3]{UC4.3} }\\

    RFD4.3.1
    & L'utente deve poter scegliere l'algoritmo di calcolo della distanza
    & Desiderabile
    & \makecell{ Interno \\ \hyperref[par:uc4.3.1]{UC4.3.1} }\\

    RFD4.3.1.1
    & L'utente deve poter scegliere l'algoritmo di calcolo della distanza euclidea
    & Desiderabile
    & \makecell{ Interno \\ \hyperref[par:uc4.3.1]{UC4.3.1} }\\

    RFD4.3.1.2
    & L'utente deve poter scegliere l'algoritmo di calcolo della distanza Manhattan
    & Desiderabile
    & \makecell{ Interno \\ \hyperref[par:uc4.3.1]{UC4.3.1} }\\

    RFD4.3.1.3
    & L'utente deve poter scegliere l'algoritmo di calcolo della distanza Minkowsky
    & Desiderabile
    & \makecell{ Interno \\ \hyperref[par:uc4.3.1]{UC4.3.1} }\\

    RFD4.3.1.4
    & L'utente deve poter scegliere l'algoritmo di calcolo della distanza Canberra
    & Desiderabile
    & \makecell{ Interno \\ \hyperref[par:uc4.3.1]{UC4.3.1} }\\

    RFD4.3.2
    & L'utente deve poter scegliere se eseguire preprocessing dei dati
    & Desiderabile
    & \makecell{ Verbale \\ \hyperref[par:uc4.3.2]{UC4.3.2} }\\

    RFD4.3.2.1
    & L'utente deve poter scegliere se normalizzare i dati
    & Desiderabile
    & \makecell{ Verbale \\ \hyperref[par:uc4.3.2]{UC4.3.2} }\\

    RFD4.3.2.2
    & L'utente deve poter scegliere se standardizzare i dati
    & Desiderabile
    & \makecell{ Verbale \\ \hyperref[par:uc4.3.2]{UC4.3.2} }\\

    RFD4.3.2.3
    & L'utente deve poter scegliere se non eseguire alcun tipo di preprocessing sui dati
    & Desiderabile
    & \makecell{ Verbale \\ \hyperref[par:uc4.3.2]{UC4.3.2} }\\


    RFD4.3.3
    & L'utente deve poter modificare l'influenza di una dimensione
    & Desiderabile
    & \makecell{ Interno \\ \hyperref[par:uc4.3.3]{UC4.3.3} }\\

    RFD4.4
    & L'utente deve poter modificare le proprietà della visualizzazione Force Field
    & Desiderabile
    & \makecell{ Capitolato \\ \hyperref[ssub:uc4.4]{UC4.4} }\\

    RFD4.4.1
    & L'utente deve poter trascinare i nodi visualizzati nella visualizzazione Force field
    & Desiderabile
    & \makecell{ Capitolato \\ \hyperref[par:uc4.4.1]{UC4.4.1} }\\

    RFD4.4.2
    & L'utente deve poter eliminare archi a cui sono associate forze al di fuori di un certo intervallo
    & Desiderabile
    & \makecell{ Capitolato \\ \hyperref[par:uc4.4.2]{UC4.4.2} }\\

    RFD4.4.2.1
    & L'utente deve poter impostare il valore di soglia minimo della forza di attrazione
    & Desiderabile
    & \makecell{ Verbale \\ \hyperref[par:uc4.4.3]{UC4.4.3} }\\

    RFD4.4.2.2
    & L'utente deve poter eliminare gli archi ai quali sono associate forze con inferiori ad una certa soglia
    nella visualizzazione Force Field
    & Desiderabile
    & \makecell{ Verbale \\ \hyperref[par:uc4.4.3]{UC4.4.3} }\\


    RFD4.4.2.3
    & L'utente deve poter impostare il valore di soglia massimo per le forze associate agli archi
    & Desiderabile
    & \makecell{ Verbale \\ \hyperref[par:uc4.4.4]{UC4.4.4} }\\

    RFD4.4.3.4
    & L'utente deve poter eliminare gli archi ai quali sono associate forze superiori ad una certa soglia
    & Desiderabile
    & \makecell{ Verbale \\ \hyperref[par:uc4.4.4]{UC4.4.4} }\\

    RFD4.4.5
    & L'utente deve poter scalare le forze di attrazione
    & Desiderabile
    & \makecell{ Interno \\ \hyperref[par:uc4.4.5]{UC4.4.5} }\\

    % NICE
    RFD4.5
    & L'utente deve poter modificare le proprietà della visualizzazione Distance Map
    & Desiderabile
    & \makecell{ Capitolato \\ \hyperref[ssub:uc4.5]{UC4.5} }\\

    RFD4.5.1
    & L'utente deve poter modificare il gradiente di colore della visualizzazione Distance Map
    & Desiderabile
    & \makecell{ Interno \\ \hyperref[par:uc4.5.1]{UC4.5.1} }\\

    RFD4.5.1.1
    & L'utente deve poter modificare il gradiente di colore in "Blue-Magenta-Yellow"
    & Desiderabile
    & \makecell{ Interno \\ \hyperref[par:uc4.5.1]{UC4.5.1} }\\

    RFD4.5.1.2
    & L'utente deve poter modificare il gradiente di colore in CoolWarm
    & Desiderabile
    & \makecell{ Interno \\ \hyperref[par:uc4.5.1]{UC4.5.1} }\\

    RFD4.5.1.3
    & L'utente deve poter modificare il gradiente di colore in Dim Gray
    & Desiderabile
    & \makecell{ Interno \\ \hyperref[par:uc4.5.1]{UC4.5.1} }\\

    RFO4.5.2
    & L'utente deve poter ordinare la Distance Map
    & Obbligatorio
    & \makecell{ Capitolato \\ \hyperref[par:uc4.5.2]{UC4.5.2} }\\

    RFO4.5.3.1
    & L'utente deve poter ordinare la Distance Map mediante clustering gerarchico
    & Obbligatorio
    & \makecell{ Capitolato \\ \hyperref[par:uc4.5.3]{UC4.5.3} }\\

    RFO4.5.3.2
    & L'utente deve poter associare un dendrogramma al clustering gerarchico nella Distance Map
    & Obbligatorio
    & \makecell{ Capitolato \\ \hyperref[par:uc4.5.3]{UC4.5.3} }\\

    RFD4.5.4
    & L'utente deve poter ripristinare l'ordinamento originario
    & Desiderabile
    & \makecell{ Interno \\ \hyperref[par:uc4.5.4]{UC4.5.4} }\\

    RFF4.5.5
    & L'utente deve poter ordinare le dimensioni rappresentate nella Distance Map per valore
    & Facoltativo
    & \makecell{ Interno \\ \hyperref[par:uc4.5.5]{UC4.5.5} }\\

    RFD4.5.6
    & L'utente deve poter modificare le etichette associate alla Distance Map
    & Desiderabile
    & \makecell{ Interno \\ \hyperref[par:uc4.5.6]{UC4.5.6} }\\

    RFD4.6
    & L'utente deve poter modificare le proprietà della visualizzazione Proiezione Lineare Multi Asse
    & Desiderabile
    & \makecell{ Interno \\  \hyperref[ssub:uc4.6]{UC4.6} }\\

    RFD4.6.1
    & L'utente deve poter aggiungere una dimensione dalla visualizzazione Proiezione Lineare Multi Asse
    & Desiderabile
    & \makecell{ Interno \\  \hyperref[par:uc4.6.1]{UC4.6.1} }\\

    RFD4.6.2
    & L'utente deve poter rimuovere una dimensione dalla visualizzazione Proiezione Lineare Multi Asse
    & Desiderabile
    & \makecell{ Interno \\  \hyperref[par:uc4.6.2]{UC4.6.2} }\\

    RFD4.6.3
    & L'utente deve poter spostare gli assi nella visualizzazione Proiezione Lineare Multi Asse
    & Desiderabile
    & \makecell{ Interno \\  \hyperref[par:uc4.6.3]{UC4.6.3} }\\

    RFD4.7
    & L'utente deve poter modificare la visualizzazione dell'Heat Map
    & Desiderabile
    & \makecell{ Interno \\  \hyperref[ssub:uc4.7]{UC4.7} }\\

    RFD4.7.1
    & L'utente deve poter modificare il gradiente di colore nell'Heat Map
    & Desiderabile
    & \makecell{ Interno \\  \hyperref[par:uc4.7.1]{UC4.7.1} }\\

    RFD4.7.1.1
    & L'utente deve poter modificare il gradiente di colore nell'Heat Map in Blue-Magenta-Yellow
    & Desiderabile
    & \makecell{ Interno \\  \hyperref[par:uc4.7.1]{UC4.7.1} }\\

    RFD4.7.1.2
    & L'utente deve poter modificare il gradiente di colore nell'Heat Map in CoolWarm
    & Desiderabile
    & \makecell{ Interno \\  \hyperref[par:uc4.7.1]{UC4.7.1} }\\

    RFD4.7.1.3
    & L'utente deve poter modificare il gradiente di colore nell'Heat Map in Dim Gray
    & Desiderabile
    & \makecell{ Interno \\  \hyperref[par:uc4.7.1]{UC4.7.1} }\\

    RFD4.7.2.4
    & L'utente deve poter modificare le etichette assocciate agli assi nell'Heat Map
    & Desiderabile
    & \makecell{ Interno \\  \hyperref[par:uc4.7.2]{UC4.7.2} }\\

    RFD4.7.3
    & L'utente deve poter modificare le etichette associate alle righe nell'Heat Map
    & Desiderabile
    & \makecell{ Interno \\  \hyperref[par:uc4.7.3]{UC4.7.3} }\\

    RFD4.7.4
    & L'utente deve poter modificare le etichette associate alle colonne nell'Heat Map
    & Desiderabile
    & \makecell{ Interno \\  \hyperref[par:uc4.7.4]{UC4.7.4} }\\

    RFD4.7.5
    & L'utente deve poter ordinare gli elementi dell'Heat Map
    & Desiderabile
    & \makecell{ Interno \\  \hyperref[par:uc4.7.5]{UC4.7.5} }\\

    RFD4.7.5.6
    & L'utente deve poter ordinare gli elementi dell'Heat Map mediante clustering gerarchico
    & Desiderabile
    & \makecell{ Interno \\  \hyperref[spar:uc4.7.5.6]{UC4.7.5.6} }\\

    RFD4.7.5.7
    & L'utente deve poter modificare l'ordinamento degli elementi ritornando all'ordine originale del dataset
    & Desiderabile
    & \makecell{ Interno \\  \hyperref[spar:uc4.7.5.7]{UC4.7.5.7} }\\

    RFD4.8
    & L'utente deve poter ripristinare la visualizzazione corrente con le proprietà di default
    & Desiderabile
    & \makecell{ Interno \\  \hyperref[ssub:uc4.8]{UC4.8} }\\

    RFO5
    & L'utente deve essere informato in caso di errori
    & Obbligatorio
    & \makecell{ Interno \\  \hyperref[sub:uc5]{UC5} }\\

    RFO6
    & L'utente deve essere informato in caso di errore durante l'inserimento di un file
    & Obbligatorio
    & \makecell{ Interno \\  \hyperref[sub:uc6]{UC6} }\\

    RFO7
    & L'utente deve essere informato in caso di errore durante l'accesso al database
    & Obbligatorio
    & \makecell{ Interno \\  \hyperref[sub:uc7]{UC7} }\\

    RFO8
    & L'utente deve essere informato in caso di errore se il database non contiene alcun dato
    & Obbligatorio
    & \makecell{ Interno \\  \hyperref[sub:uc8]{UC8} }\\

    RFO9
    & L'utente deve essere informato in caso la query usata per recuperare i dati non ritorni nessun dato
    & Obbligatorio
    & \makecell{ Interno \\  \hyperref[sub:uc9]{UC9} }\\

    RFO10
    & L'utente deve essere informato in caso provi a modificare la visibilità di un dato quando non può farlo
    & Obbligatorio
    & \makecell{ Interno \\  \hyperref[sub:uc10]{UC10} }\\
    \hline
    \rowcolor{white}
    \caption{Requisiti funzionali}%
    \label{tab:requisiti_funzionali}
\end{longtable}

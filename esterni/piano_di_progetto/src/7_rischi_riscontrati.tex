\documentclass[../piano_di_progetto.tex]{subfiles}

\begin{document}
\subsection{Periodo RR}
\label{sub:rischi_rr}
\begin{center}
	\begin{longtable}{|p{4cm}|p{6cm}|p{6cm}|}
		\hline
		\rowcolor{lightgray}
		{\textbf{Codice}} & {\textbf{Descrizione}} & {\textbf{Soluzione}} \\
		\hline
		\endfirsthead
	
		\hline
		\rowcolor{lightgray}
		{\textbf{Codice}} & {\textbf{Descrizione}} & {\textbf{Soluzione}} \\
		\hline
		\endhead
		
		\hline
		\multicolumn{3}{|c|}{\emph{Continua alla pagina successiva...}}\\
		\hline
		\endfoot

		\endlastfoot
        RR01 - Errata analisi dei requisiti 
        & Il motivo principale per cui vi è stato ritardo nella prima consegna dei documenti è stato proprio l’errata analisi dei requisiti, alla quale sono conseguite delle correzioni che hanno richiesto più tempo di quanto preventivato.  
        & Dopo la prima consegna dei documenti il gruppo si è premurato di effettuare un nuovo incontro con l’azienda al fine di sciogliere subito i nodi sorti precedentemente e poter proseguire spediti nella giusta direzione. \\
        
        RO02 - Impegni personali
        & A causa di impegni personali non tutti sono riusciti a completare i compiti per le scadenze fissate.
        & Il resto del gruppo ha mitigato supportando dove possibile. \\
        
        RO04 - Gestione della repository
        & Trattandosi del primo progetto che prevede tutti questi componenti, il gruppo si è reso conto che la gestione delle repository andava rivista e migliorata.
        & Dopo la prima consegna i nuovi amministratori si sono premurati di trovare una gestione migliore e più funzionale alle necessità del gruppo, in accordo con tutti. \\

        RO05 - Amministrazione delle risorse
        & Un ulteriore motivo per cui vi è stato ritardo è il fatto che c’è stata una cattiva gestione delle risorse, nello specifico ci sono stati errori nel distribuire i compiti.
        & Il gruppo ha rivisto la distribuzione dei compiti ponderando meglio l’impegno richiesto per ciascuno di essi in relazione con la disponibilità dei componenti. \\
        
        \hline
		\rowcolor{white}
		\caption{Tabella rischi riscontrati nel periodo di RR}
	\end{longtable}

\end{center}

\subsection{Periodo RP}
\label{sub:rischi_rp}
\begin{center}
	\begin{longtable}{|p{4cm}|p{6cm}|p{6cm}|}
		\hline
		\rowcolor{lightgray}
		{\textbf{Codice}} & {\textbf{Descrizione}} & {\textbf{Soluzione}} \\
		\hline
		\endfirsthead
	
		\hline
		\rowcolor{lightgray}
		{\textbf{Codice}} & {\textbf{Descrizione}} & {\textbf{Soluzione}} \\
		\hline
		\endhead
		
		\hline
		\multicolumn{3}{|c|}{\emph{Continua alla pagina successiva...}}\\
		\hline
		\endfoot

		\endlastfoot
        RT01 - Familiarità con gli ambienti di sviluppo
        & Alcuni membri del gruppo hanno avuto difficoltà nel famigliarizzare con le tecnologie che verranno usate per lo sviluppo dell'applicativo.
        & Quando possibile vi è stato supporto ai membri in difficoltà dal resto del gruppo. \\

        RO02 - Impegni personali
        & A causa di impegni personali non tutti sono riusciti a completare i compiti per le scadenze fissate.
        & Il resto del gruppo ha mitigato supportando dove possibile. \\

        RP01 - Inesperienza del team a livello gestionale
        & In seguito ad una rivisitazione della pianificazione per il periodo della RP è emerso che le ore preventivate per i ruoli di Progettista e Programmatore erano sottostimate, mentre quelle degli altri ruoli erano sovrastimate, seppur di poco.
        & Sono state ridistribuite alcune ore da altri ruoli. \\
        \hline
		\rowcolor{white}
		\caption{Tabella rischi riscontrati nel periodo di RP}
	\end{longtable}

\end{center}

\subsection{Periodo RQ}
\label{sub:rischi_rq}
\begin{center}
	\begin{longtable}{|p{4cm}|p{6cm}|p{6cm}|}
		\hline
		\rowcolor{lightgray}
		{\textbf{Codice}} & {\textbf{Descrizione}} & {\textbf{Strategia adottata}} \\
		\hline
		\endfirsthead
	
		\hline
		\rowcolor{lightgray}
		{\textbf{Codice}} & {\textbf{Descrizione}} & {\textbf{Strategia adottata}} \\
		\hline
		\endhead
		
		\hline
		\multicolumn{3}{|c|}{\emph{Continua alla pagina successiva...}}\\
		\hline
		\endfoot

		\endlastfoot
        RT01 - Inesperienza del team a livello tecnico
        & L'inesperienza dei membri del team nell'utilizzo di design pattern visti in teoria a lezione ha portato il gruppo ad utilizzarli in modo parzialmente scorretto. Questo ha portato ad una valutazione congelata della presentazione della Product Baseline con il prof. Cardin. 
		& Il team si è impegnato a risolvere i problemi rilevati in fase di presentazione della Product Baseline. \\
		
		RO05 - Inesperienza del team a livello gestionale
        & Le problematiche legate all'inesperienza dei membri del gruppo porta ad una errata stima delle tempistiche lavorative. Risulta difficile valutare correttamente le tempistiche per completare un determinato compito. 
        & Si cerca di ragionare sempre per stima in eccesso e prendendo in considerazione le esperienze precedenti. \\
		
		RT01 - Inesperienza del team nell'uso del linguaggio di programmazione
        & A causa della poca esperienza con il linguaggio di programmazione usato per la scrittura del codice, i programmatori hanno impiegato più tempo del previsto nella scrittura di quest'ultimo.
		& Ogni membro del team si è documentato e studiato il linguaggio di programmazione in modo autonomo e autosufficiente, aiutando i membri più in difficoltà all'occorrenza.\\
		
		RT01 - Problematica legata ai test di unità con Jest
		& Alcuni membri del team non sono riusciti ad eseguire i test di unità dei moduli Node.js con il framework Jest a causa di un errore.
		& I membri del team che hanno riscontrato l'errore si sono documentati in autonomia, riuscendo a risolvere il problema modificando un parametro su Node.js. Inoltre sono stati notificati tutti i membri del gruppo sul problema e sulla relativa soluzione. \\
		
		RT01 - Difficoltà nella creazione dei diagrammi UML
		& L'inesperienza lavorativa e la scarsa conoscenza dei progettisti dei linguaggi di programmazione orientati agli oggetti si sono trasformati in difficoltà nel momento della scrittura dei diagrammi UML. 
		& I progettisti si impegnano nello studio individuale della struttura del linguaggio di programmazione orientato agli oggetti per la scrittura di diagrammi UML i più corretti possibili. \\

		RO03 - Mancata comunicazione tra i membri del gruppo
		& In questa fase c'è stata una cattiva amministrazione e comunicazione tra i membri del gruppo, che ha portato ad un rilevante ritardo nelle tempistiche.
		& Gli amministratori si sono impegnati nel cercare metodi più efficaci per la gestione delle risorse e delle scadenze per questa fase e sopratutto per la fase successiva. Ogni membro del gruppo si impegna a comunicare in modo più frequente e chiaro con il resto del team. \\
		\hline
		\rowcolor{white}
		\caption{Tabella rischi riscontrati nel periodo di RQ}
	\end{longtable}

\end{center}

\end{document}

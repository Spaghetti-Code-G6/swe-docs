\documentclass[../piano_di_progetto.tex]{subfiles}

\begin{document}

L’ \glossario{analisi dei rischi} viene effettuata con lo scopo di prevenire e affrontare i rischi che si possono presentare nel corso del progetto. Infatti è facile incorrere in problematiche derivanti dalle tempistiche prefissate, oppure dalle tecnologie nuove.\\
I principali rischi sono stati suddivisi nelle seguenti categorie:
\begin{itemize}
	\item Rischi dei requisiti - Codice \textbf{RR}
	\item Rischi tecnologici  - Codice \textbf{RT}
	\item Rischi Organizzativi - Codice \textbf{RO}
	\item Rischi Personali - Codice \textbf{RP}
\end{itemize}
Nei seguenti paragrafi questi rischi verranno classificati secondo le seguenti metriche:
\begin{itemize}
	\item \textbf{Probabilità di occorrenza}: alta frequenza, media frequenza, bassa frequenza
	\item \textbf{Pericolosità del rischio}: alta, medio, bassa
\end{itemize}

\subsection{Rischi dei requisiti}%
\label{sub:rischi_req}

\begin{center}
	\begin{longtable}{|c|p{4.5cm}|c|c|p{4.5cm}|}
		\hline
		\rowcolor{lightgray}
		{\textbf{Codice}} & {\textbf{Descrizione}} & {\textbf{Occorrenza}} & {\textbf{Gravità}} & {\textbf{Mitigazione}}                                                     \\

		\hline
		RR01              &
		Errata analisi dei requisiti: sebbene il proponente abbia descritto dettagliatamente l’obiettivo del progetto e le relative richieste, l’inesperienza del gruppo potrebbe comportare una errata analisi dei requisiti
		                  &
		Media
		                  &
		Alta
		                  &
		Il gruppo si impegna a comunicare con il proponente per poter chiarire eventuali dubbi e accertarsi che l’andamento progettuale sia positivo e rispetti le richieste \\

		RR02              &
		Modifica requisiti: potrebbe verificarsi una modifica dei requisiti minimi o opzionali in corso d’opera
		                  &
		Bassa
		                  &
		Alta
		                  &
		Grazie ad una regolare comunicazione con il proponente il gruppo sarà in grado di rettificare il lavoro svolto nelle tempistiche opportune                           \\
		\hline
		\rowcolor{white}
		\caption{Tabella contenente l'elenco dei rischi dei requisiti}
	\end{longtable}

\end{center}

\newpage

\subsection{Rischi tecnologici}%
\label{sub:rischi_tec}

\begin{center}
	\begin{longtable}{|c|p{4.5cm}|c|c|p{4.5cm}|}
		\hline
		\rowcolor{lightgray}
		{\textbf{Codice}} & {\textbf{Descrizione}} & {\textbf{Occorrenza}} & {\textbf{Gravità}} & {\textbf{Mitigazione}}                                                                                                                                                                                                                \\

		\hline
		RT01              &
		Ambienti di sviluppo: non tutti i componenti hanno conoscenze degli ambienti di sviluppo e linguaggi richiesti dal proponente
		                  &
		Media
		                  &
		Alta
		                  &
		Sarà compito di ciascun componente gestire il proprio tempo da dedicare allo studio autonomo affinché tutti possano raggiungere un livello paritario di preparazione                                                                                                                                                            \\
		
		RT02              &
		Guasti: tutti i componenti hanno dei personal computer con cui lavorare al progetto; potrebbe succedere che qualcuno abbia un guasto hardware o software che ne comporti la perdita di progressi
		                  &
		Bassa
		                  &
		Bassa
		                  &
		Ciascuno si impegna a lavorare su file online grazie all’ampio supporto software che l’Università offre. Nel caso qualcuno sia impossibilitato a proseguire nel proprio lavoro, costui comunicherà tempestivamente il problema al gruppo ed il responsabile ridistribuirà il lavoro da concludere al primo compagno disponibile \\
		\hline
		\rowcolor{white}
		\caption{Tabella contenente l'elenco dei rischi tecnologici}
	\end{longtable}
\end{center}

\subsection{Rischi Organizzativi}%
\label{sub:rischi_org}

\begin{center}
	\begin{longtable}{|c|p{4.5cm}|c|c|p{4.5cm}|}
		\hline
		\rowcolor{lightgray}
		{\textbf{Codice}} & {\textbf{Descrizione}} & {\textbf{Occorrenza}} & {\textbf{Gravità}} & {\textbf{Mitigazione}}\\
		\hline

		\endfirsthead
	
		\hline
		\rowcolor{lightgray}
		{\textbf{Codice}} & {\textbf{Descrizione}} & {\textbf{Occorrenza}} & {\textbf{Gravità}} & {\textbf{Mitigazione}}\\
		\hline
		\endhead
		
		\hline
		\multicolumn{5}{|c|}{\emph{Continua alla pagina successiva...}}\\
		\hline
		\endfoot

		\endlastfoot

		RO01              &
		Incontri: ogni componente ha degli impegni personali, questo comporta che non sempre saranno tutti disponibili ad incontrarsi ad una determinata ora
		                  &
		Media
		                  &
		Medio-alta
		                  &
		Il gruppo si accorda con anticipo per eventuali incontri; qualora dovessero mancare uno o due componenti il responsabile si occuperà di aggiornare gli assenti circa le riunioni svolte                                                         \\
		
		RO02              &
		Impegni personali: dal momento che ognuno ha degli impegni personali potrebbe non essere possibile completare i singoli compiti assegnati entro le tempistiche richieste
		                  &
		Medio-bassa
		                  &
		Alta
		                  &
		Ciascun componente si impegna a svolgere al meglio delle proprie possibilità i singoli compiti assegnati; qualora per incombenze non sia possibile concluderli entro le tempistiche richieste, il singolo si impegna a comunicarlo con anticipo \\
		
		RO03              &
		Accordarsi: Nessun componente del gruppo ha mai lavorato in team così ampi, quindi potrebbe rivelarsi non facile mettere d’accordo 7 soggetti differenti
		                  &
		Media
		                  &
		Media
		                  &
		Il gruppo si impegna a prendere le decisioni rispettando il pensiero di chiunque, e mettendo ogni membro sullo stesso piano. Qualora dovessero sorgere incertezze il gruppo sceglie di adottare la votazione come soluzione dei conflitti.      \\
		
		RO04 		& 
		Gestione della repository: data l'inesperienza a gestire archivi destinati all'uso di più componenti, potrebbero sorgere problemi nello stabilire le giuste modalità di utilizzo & 
		Media 		& 
		Alta 		& 
		L’Amministratore si accorderà con i vari componenti per trovare la soluzione migliore nella gestione della repository, nonché la soluzione migliore per poter svolgere al meglio il lavoro di ciascuno. \\

		RO05 		& 
		Cattiva amministrazione delle risorse: data l’inesperienza del team di sviluppo con progetti di questa natura, è possibile che sorgano problemi nell’amministrazione delle risorse come tempo, costi e suddivisione dei ruoli.  & 
		Media 		& 
		Alta 		& 
		A ciascuna riunione di SpaghettiCode, si controllerà se il lavoro svolto fino a quel momento è pertinente a quanto è stato preventivato; qualora qualcosa non andasse si ridistribuiranno le risorse in modo da rispettare la tabella di marcia e i costi preventivati.\\

		\hline
		\rowcolor{white}
		\caption{Tabella contenente l'elenco dei rischi organizzativi}
	\end{longtable}

\end{center}


\subsection{Rischi Personali}%
\label{sub:rischi_pers}

\begin{center}
	\begin{longtable}{|c|p{4.5cm}|c|c|p{4.5cm}|}
		\hline
		\rowcolor{lightgray}
		{\textbf{Codice}} & {\textbf{Descrizione}} & {\textbf{Occorrenza}} & {\textbf{Gravità}} & {\textbf{Mitigazione}}\\
		\hline

		\endfirsthead
	
		\hline
		\rowcolor{lightgray}
		{\textbf{Codice}} & {\textbf{Descrizione}} & {\textbf{Occorrenza}} & {\textbf{Gravità}} & {\textbf{Mitigazione}}\\
		\hline
		\endhead
		
		\hline
		\multicolumn{5}{|c|}{\emph{Continua alla pagina successiva...}}\\
		\hline
		\endfoot

		\endlastfoot

		RP01		& 
		Inesperienza del team a livello gestionale: non avendo affrontato progetti del genere prima d’ora, i componenti di SpaghettiCode non conoscono bene la natura ruoli che devono intraprendere, i compiti da svolgere e quali di essi competono un determinato ruolo. 	 
					&  
		Media 		& 
		Media 		&  
		Durante le ore di studio personale, ciascun componente si impegnerà a studiare la gerarchia dei ruoli e, in caso di dubbi, ne parlerà con il team di sviluppo oppure direttamente con il Responsabile.\\

		RP02 		& 
		Approvazione errata dei documenti: è possibile che il Responsabile commetta errori nell’approvazione dei documenti, che potrebbero portare alla consegna di documentazione errata o scadente.  
					&  
		Media 		& 
		Alta 		& 
		È necessario correggere tali sviste, andando quindi a sprecare risorse investibili in altri compiti. Il Responsabile deve avere modo di controllare il lavoro prodotto dal proprio team in modo costante e graduale, inoltre colui che copre il ruolo di Responsabile deve assicurarsi che i documenti approvati siano effettivamente validi; in caso di sviste il Verificatore deve saper trovare e correggere gli eventuali errori. \\

		RP03 		& 
		Intesa parziale tra i membri del team di sviluppo 
					& 
		Media 		& 
		Media 		& 
		Il team di sviluppo è formato principalmente da persone che precedentemente non si conoscevano o che hanno avuto poche interazioni tra di loro fino al momento della creazione di quest’ultimo. Nonostante ciò si assume che ognuno abbia la maturità necessaria per poter lavorare in armonia a prescindere da eventuali disguidi o discussioni che possono sorgere. \\


		\hline
		\rowcolor{white}
		\caption{Tabella contenente l'elenco dei rischi personali}
	\end{longtable}

\end{center}

\end{document}
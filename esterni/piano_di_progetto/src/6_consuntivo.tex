\documentclass[../piano_di_progetto.tex]{subfiles}

\begin{document}
In questa sezione verranno illustrati i prospetti orari e relativi costi per le varie fasi di lavoro. Il bilancio si suddivide in:
\begin{itemize}
\item \textbf{Positivo}: se sono state necessarie meno ore di quelle preventivate;
\item \textbf{Paritario}: se sono state svolte effettivamente le ore preventivate;
\item \textbf{Negativo}: se sono state necessarie più ore di quelle preventivate;
\end{itemize}

\subsection{Analisi}%
\label{sub:cons_analisi}
Le ore di lavoro svolte in questa fase sono destinate alla scelta del capitolato e allo studio autonomo, quindi queste ore svolte non verranno rendicontate.\\
Poiché l'analisi dei requisiti è un documento molto importante, ha richiesto l'investimento di molte più ore da parte degli analisti.

\begin{table}[!ht]
	\centering
	\begin{tabular}{|l|c|c|c|c|c|c|c|}
	\hline
	\rowcolor{lightgray}
	\textbf{Nome} & \textbf{Re} & \textbf{Am} & \textbf{An} & \textbf{Pg}  & \textbf{Pr}   & \textbf{Ve} & \textbf{Totale}\\
	\hline
	Contro Daniel Eduardo & 0 & 0 & 16(+3) & 0 & 0 & 15 (+1) & 31(+4) \\
	Fichera Jacopo & 0 & 0 & 16(+3) & 0 & 0 & 15 (+1) & 31(+4) \\
	Kostadinov Samuel & 0 & 10 & 5 & 0 & 0 & 12 & 27 \\			
	Masevski Martin & 4 & 10 & 5 & 0 & 0 & 8 & 27 \\
	Pagotto Manuel & 0 & 0 & 15(+2) & 0 & 0 & 16 (+2) & 31(+4) \\			
	Paparazzo Giorgia & 13 & 0 & 4 & 0 & 0 & 10 & 27 \\
	Rizzo Stefano & 0 & 0 & 16(+4) & 0 & 0 & 15 & 31(+4) \\
	\hline
	\end{tabular}
	\caption{Resoconto orario della fase di Analisi}
\end{table}

\begin{center}
	\begin{longtable}{|l|c|c|}
		\hline
		\rowcolor{lightgray}
		\textbf{Ruolo} & \textbf{Ore} & \textbf{Costo in €}\\
		\hline
		\endhead
		
		\hline
		\multicolumn{3}{|c|}{\emph{Continua alla pagina successiva...}}\\
		\hline
		\endfoot

		\endlastfoot
		Responsabile & 17 & 510,00 \\
		Amministratore & 20 & 400,00 \\
		Analista & 77(+12) & 1.625,00(+300,00 €) \\
		Progettista &    0       & 0 \\
		Programmatore &  0       & 0 \\
		Verificatore &   91(+4)      & 1.365,00 (+60,00€) \\
		\hline
		\textbf{Totale preventivo} & \textbf{189} & \textbf{3.840,00 €} \\
		\hline
		\textbf{Totale consuntivo} & \textbf{205	} & \textbf{4.200,00 €} \\
		\hline
		\textbf{Differenza} & \textbf{+16} & \textbf{+360 €}\\
		\hline
		\rowcolor{white}
		\caption{Resoconto economico nella fase di analisi}
	\end{longtable}
\end{center}

\subsubsection{Conclusioni}%
\label{sub:cons_con_1}
A fronte di una settimana di ritardo nella consegna, si è vista la necessità di aumentare le ore di lavoro di alcuni dei componenti del gruppo, di conseguenza anche il bilancio economico ha visto un aumento di costi. Sebbene il surplus sfori il limite del 5\% stabilito nel \textsc{Piano di Qualifica}, questo costo non intaccherà il budget totale rendicontato perché questa fase non è rendicontata.

\subsubsection{Preventivo a finire}
\label{sub:cons_prev_fine_1}
Rispettando i tempi preventivati non è necessario apportare modifiche al preventivo.\\ \\

\subsection{Consolidamento dei requisiti}%
\label{sub:cons_cons_req}
Le ore di lavoro svolte in questa fase non verranno rendicontate nel preventivo finale in quanto sono considerate ore di investimento per l'approfondimento personale.\\
Nella tabella che segue vengono riportate le ore di lavoro effettive durante il periodo di consolidamento dei requisiti: \\

\begin{table}[!ht]
	\centering
	\begin{tabular}{|l|c|c|c|c|c|c|c|}
	\hline
	\rowcolor{lightgray}
	\textbf{Nome} & \textbf{Re} & \textbf{Am} & \textbf{An} & \textbf{Pg}  & \textbf{Pr}   & \textbf{Ve} & \textbf{Totale}\\
	\hline
	Contro Daniel Eduardo & 0 & 0 & 2 & 0 & 0 & 3 & 5 \\
	Fichera Jacopo & 0 & 0 & 2 & 0 & 0 & 3 & 5 \\
	Kostadinov Samuel & 0 & 1 & 1 & 0 & 0 & 3 & 5 \\			
	Masevski Martin & 1 & 0 & 1 & 0 & 0 & 3 & 5 \\
	Pagotto Manuel & 0 & 0 & 2 & 0 & 0 & 3 & 5 \\			
	Paparazzo Giorgia & 1 & 0 & 1 & 0 & 0 & 3 & 5 \\
	Rizzo Stefano & 0 & 0 & 2 & 0 & 0 & 3 & 5 \\
	\hline
	\textbf{Ore totali ruolo} & \textbf{2} & \textbf{1} & \textbf{11} & \textbf{0} & \textbf{0} & \textbf{21} & \textbf{35} \\
	\hline
	\end{tabular}
	\caption{Resoconto orario della fase di consolidamento dei requisiti}
\end{table}

\begin{center}
	\begin{longtable}{|l|c|c|}
		\hline
		\rowcolor{lightgray}
		\textbf{Ruolo} & \textbf{Ore} & \textbf{Costo in €}\\
		\hline
		\endhead
		
		\hline
		\multicolumn{3}{|c|}{\emph{Continua alla pagina successiva...}}\\
		\hline
		\endfoot

		\endlastfoot
		Responsabile & 	 2 	 & 60,00 \\
		Amministratore & 1 	 & 20,00 \\
		Analista & 		11 	 & 275,00 \\
		Progettista &    0   & 0 \\
		Programmatore &  0   & 0 \\
		Verificatore &   21  & 315,00 \\
		\hline
		\textbf{Totale preventivo} & \textbf{35} & \textbf{€ 670,00} \\
		\hline
		\textbf{Totale consuntivo} & \textbf{35} & \textbf{€ 670,00} \\
		\hline
		\textbf{Differenza} & \textbf{0} & \textbf{€ 0}\\
		\hline
		\rowcolor{white}
		\caption{Resoconto economico della fase di consolidamento dei requisiti}
	\end{longtable}
\end{center}

\subsubsection{Conclusioni}%
\label{sub:cons_con_2}
Ogni ruolo coinvolto in questa fase ha rispettato i tempi, rientrando nelle ore preventivate.

\subsubsection{Preventivo a finire}
\label{sub:cons_prev_fine_2}
Rispettando i tempi preventivati non è necessario apportare modifiche al preventivo. \\ \\

\clearpage
\subsection{Progettazione e codifica della technology baseline}%
\label{sub:cons_prog_tech}
Di seguito vengono riportate le ore di lavoro effettivi durante la fase di progettazione e codifica della technology baseline: \\

\begin{table}[!ht]
	\centering
	\begin{tabular}{|l|c|c|c|c|c|c|c|}
	\hline
	\rowcolor{lightgray}
	\textbf{Nome} & \textbf{Re} & \textbf{Am} & \textbf{An} & \textbf{Pg}  & \textbf{Pr}   & \textbf{Ve} & \textbf{Totale}\\
	\hline
	Contro Daniel Eduardo & 0 & 6(-1) & 0 & 8 & 7(+1) & 9 & 30 \\
	Fichera Jacopo & 0 & 0 & 0 & 9(-1) & 9(+2) & 12(-1) & 30 \\
	Kostadinov Samuel & 0 & 0 & 9(+1) & 10(+2) & 0 & 11(-3) & 30 \\			
	Masevski Martin & 8(-3) & 0 & 0 & 0 & 13(+8) & 9(-5) & 30 \\
	Pagotto Manuel & 8(-2) & 0 & 10(+1) & 0 & 0 & 12(+1) & 30 \\			
	Paparazzo Giorgia & 0 & 0 & 7(-3) & 8(+4) & 7(+1) & 8(-2) & 30 \\
	Rizzo Stefano & 0 & 9 & 0 & 6(+2) & 5(+1) & 10(-3) & 30 \\
	\hline
	\textbf{Ore totali ruolo} & \textbf{16(-5)} & \textbf{15(-1)} & \textbf{26(-1)} & \textbf{41(+7)} & \textbf{41(+13)} & \textbf{71(-13)} & \textbf{210} \\
	\hline
	\end{tabular}
	\caption{Resoconto orario della fase di progettazione e codifica della technology baseline}
\end{table}

\begin{center}
	\begin{longtable}{|l|c|c|}
		\hline
		\rowcolor{lightgray}
		\textbf{Ruolo} & \textbf{Ore} & \textbf{Costo in €}\\
		\hline
		\endhead
		
		\hline
		\multicolumn{3}{|c|}{\emph{Continua alla pagina successiva...}}\\
		\hline
		\endfoot

		\endlastfoot
		Responsabile & 	 16(-5) 	 & 480,00(-150,00) \\
		Amministratore & 15(-1) 	 & 300,00(-20,00) \\
		Analista & 		26(-1) 	 & 650,00(-25,00) \\
		Progettista &    41(+7)   & 902,00(+154,00) \\
		Programmatore &  41(+13)   & 615,00(+195,00) \\
		Verificatore &   71(-13)  & 1.065,00(-195,00) \\
		\hline
		\textbf{Totale preventivo} & \textbf{210} & \textbf{€ 4.053,00} \\
		\hline
		\textbf{Totale consuntivo} & \textbf{210} & \textbf{€ 4.012,00} \\
		\hline
		\textbf{Differenza} & \textbf{0} & \textbf{-€ 41,00}\\
		\hline
		\rowcolor{white}
		\caption{Resoconto economico della fase di progettazione e codifica della technology baseline}
	\end{longtable}
\end{center}

\subsubsection{Conclusioni}%
\label{sub:cons_con_3}
In questa fase il gruppo non è riuscito a smaltire il ritardo accumulato nella fase precedente, per questo questo motivo i documenti vengono consegnati con circa una settimana di ritardo rispetto a quello pianificato. \\
Durante questo periodo è stato rispettato il monte ore preventivato. Le ore totali per ogni ruolo hanno subito una sostanziale modifica, di seguito vengono spiegate le motivazioni:
\begin{itemize}
	\item \textbf{Responsabile}: grazie all'ottima collaborazione dei membri del team il responsabile è riuscito a completare il suo lavoro con un risparmio sulle ore preventivate;
	\item \textbf{Amministratore}: l'amministratore ha portato a termine il lavoro in meno tempo rispetto a quello preventivato, riuscendo a risparmiare un'ora di lavoro;
	\item \textbf{Analista}: l'analista è riuscito a portare a termine il proprio lavoro con minor tempo rispetto a quello preventivato, risparmiando un'ora di lavoro;
	\item \textbf{Progettista}: per una più approfondita e dettagliata possibile progettazione del software, il progettista ha richiesto più ore di quelle inizialmente preventivate, influenzando in minima parte il resoconto economico di questo periodo;
	\item \textbf{Programmatore}: per la realizzazione di un prodotto il più possibile stabile e funzionale i programmatori hanno sforato il monte ore preventivato, influenzando in minima parte il resoconto economico di questo periodo. A fronte di ciò sono riusciti a sviluppare ;
	\item \textbf{Verificatore}: grazie al lavoro attento svolto dai ruoli sopraelencati è stato possibile risparmiare tredici ore di lavoro dal totale preventivato;
\end{itemize} 
In conclusione, in questo periodo, sono stati risparmiati € 41,00 rispetto a quello preventivato.

\subsubsection{Preventivo a finire}
\label{sub:cons_prev_fine_3}
Il bilancio economico viene chiuso in positivo di € 41,00. I fondi risparmiati verranno investiti nei successivi periodi per far fronte ad eventuali ritardi o per l'implementazione di requisiti opzionali. 
\end{document}
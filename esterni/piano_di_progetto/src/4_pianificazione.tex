\documentclass[../piano_di_progetto.tex]{subfiles}

\begin{document}

La pianificazione si è basata sulle scadenze descritte nel sottocapitolo \S\ref{sub:scad}; il gruppo ha scelto di suddividere il periodo che va dalla formazione alla Revisione di Accettazione nelle seguenti fasi:
\begin{itemize}
\item Analisi dei requisiti;
\item Consolidamento dei requisiti;
\item Progettazione architetturale;
\item Progettazione di dettaglio e codifica;
\item Validazione e collaudo.
\end{itemize}
Ad ognuna di queste attività verranno destinate delle risorse di seguito descritte.


\subsection{Analisi}%
\label{sub:analisi}
Periodo: dal 22/10/2020 al 10/01/2021.\\
Questa attività si svolge nel periodo che va dalla formazione del gruppo alla consegna della \glossario{Revisione dei Requisiti}. In questo periodo il gruppo inizia con la visione dei \glossario{capitolati} proposti, e per ognuno di essi traccia un prototipo di \textsc{\glossario{studio di fattibilità}} dove vengono evidenziati gli aspetti positivi e negativi di ciascuno. Nel frattempo, vengono stabilite le prime \glossario{norme di progetto}, utili a fissare degli standard di utilizzo ai quali il gruppo si deve attenere. Successivamente alla scelta del capitolato vengono tracciati i requisiti minimi richiesti dal proponente, inoltre il gruppo incontra l’azienda tramite riunioni online allo scopo di risolvere eventuali dubbi.\\

\begin{figure}[H]
\centering
\includegraphics[width=18cm]{img/01_RR.png}
\caption{ \glossario{Diagramma di Gantt} della fase di analisi dei requisiti}

\end{figure}

\subsection{Consolidamento dei requisiti}%
\label{sub:cons_req}
Periodo: dal 11/01/2021 al 18/01/2021.\\
La fase di consolidamento dei requisiti inizia subito dopo la consegna della \textsc{Revisione dei Requisiti}; prevede l’approfondimento delle tecnologie richieste attraverso lo studio autonomo, l'approfondimento dell'analisi dei requisiti con eventuali aggiornamenti, se necessario. Sarà inoltre previsto qualche contatto con la \emph{Zucchetti S.p.A.} per fugare eventuali dubbi. Per concludere verrà preparata la presentazione. Questa fase si conclude con la presentazione della \textsc{Revisione dei Requisiti} giorno 18/01/2021.

\begin{figure}[H]
\centering
\includegraphics[width=18cm]{img/01_RR_consolidamento.png}
\caption{ \glossario{Diagramma di Gantt} della fase di consolidamento dei requisiti}
\end{figure}

\subsection{Progettazione architetturale}%
\label{sub:prog_arc}
Periodo: dal 19/01/2021 al 01/03/2021.\\
La fase di progettazione architetturale inizia subito dopo la presentazione della Revisione dei Requisiti.\\
In questa fase verranno aggiornati i documenti redatti, verranno aggiornati i requisiti in base ai contatti avuti col proponente, verrà fatto uno studio più approfondito delle tecnologie coinvolte nel progetto, al fine di realizzare un \glossario{Proof of Concept} che farà da base per la fase successiva. Verrà infine scritta la lettera di presentazione al fine di candidarsi alla \textsc{Revisione di Progettazione}. Nella settimana successiva alla consegna della \textsc{Revisione di Progettazione} verrà redatta la presentazione da portare in sede di revisione programmata giorno 08/03/2021. 

\begin{figure}[H]
\centering
\includegraphics[width=18cm]{img/02_RP.png}
\caption{Diagramma di Gantt della fase di progettazione architetturale}
\end{figure}


\subsection{Progettazione di dettaglio e codifica}%
\label{sub:prog_dett}
Periodo: dal 01/03/2021 al 02/04/2021.\\
Questa fase prevede la stesura del codice sulla base dei dettagli precedentemente tracciati; vi saranno aggiustamenti della pianificazione e aggiornamenti dei documenti. Verrà inoltre redatto il \textsc{\glossario{Manuale d’utente}}. I contatti con il proponente si manterranno stretti al fine di poter effettuare i primi rilasci e assicurarsi di essere in linea con le richieste.\\
Per concludere verrà redatta la lettera di presentazione per candidarsi alla \textsc{Revisione di Qualifica}. Nella settimana successiva alla consegna della \textsc{Revisione di Qualifica} verrà redatta la presentazione da portare in sede di revisione programmata giorno 09/04/2021. 

\begin{figure}[H]
\centering
\includegraphics[width=18cm]{img/03_RQ.png}
\caption{Diagramma di Gantt della fase di dettaglio e codifica}
\end{figure}

\subsection{Validazione e Collaudo}%
\label{sub:valid_coll}
Periodo: 03/04/2021 al 03/05/2021.\\
Questa fase terminerà con la \textsc{Revisione di Accettazione}; in questo periodo verranno aggiornati i documenti e verrà verificato il software creato. Saranno impiegati principalmente verificatori e programmatori al fine di effettuare un controllo serrato sul software che verrà rilasciato. Sarà completato il \textsc{Manuale d'utente} e sarà redatta la presentazione e la lettera finale. Nella settimana successiva alla consegna della \textsc{Revisione di Accettazione} verrà redatta la presentazione da portare in sede di revisione programmata giorno 10/05/2021.

\begin{figure}[H]
\centering
\includegraphics[width=18cm]{img/04_RA.png}
\caption{Diagramma di Gantt della fase di validazione e collaudo}
\end{figure}

\end{document}
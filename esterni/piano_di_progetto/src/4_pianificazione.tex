\documentclass[../piano_di_progetto.tex]{subfiles}

\begin{document}

La pianificazione si è basata sulle scadenze descritte nel sottocapitolo \S\ref{sub:scad}; il gruppo ha scelto di suddividere il periodo che va dalla formazione alla Revisione di Accettazione nelle seguenti fasi:
\begin{itemize}
\item \textbf{Analisi};
\item \textbf{Consolidamento dei requisiti};
\item \textbf{Progettazione e codifica della technology baseline};
\item \textbf{Progettazione e codifica di dettaglio};
\item \textbf{Validazione e collaudo}.
\end{itemize}
Ad ognuna di queste attività verranno destinate delle risorse di seguito descritte.

\subsection{Analisi}%
\label{sub:analisi}
Periodo: dal 2020-10-22 al 2021-01-10.\\
Questa attività si svolge nel periodo che va dalla formazione del gruppo alla consegna della \glossario{Revisione dei Requisiti}. In questo periodo il gruppo inizia con la visione dei \glossario{capitolati} proposti, e per ognuno di essi traccia un prototipo di \textsc{\glossario{studio di fattibilità}} dove vengono evidenziati gli aspetti positivi e negativi di ciascuno. Nel frattempo, vengono stabilite le prime \glossario{norme di progetto}, utili a fissare degli standard di utilizzo ai quali il gruppo si deve attenere. Successivamente alla scelta del capitolato vengono tracciati i requisiti minimi richiesti dal proponente, inoltre il gruppo incontra l'azienda tramite riunioni online allo scopo di risolvere eventuali dubbi.\\

\begin{figure}[H]
\centering
\includegraphics[width=18cm]{src/img/gantt/01_RR.png}
\caption{ \glossario{Diagramma di Gantt} della fase di analisi dei requisiti}

\end{figure}


\subsection{Consolidamento dei requisiti}%
\label{sub:cons_req}
Periodo: dal 11-01-2021 al 18-01-2021.\\
La fase di consolidamento dei requisiti inizia subito dopo la fase di analisi e finisce il giorno della presentazione della \textsc{Revisione dei Requisiti}.

\subsubsection{Attività}
\begin{itemize}
    \item \textbf{Consolidamento}: attività che ha lo scopo di migliorare e/o consolidare i requisiti della fase precedente;
    \item \textbf{Preparazione della presentazione}: viene prodotto il materiale necessario all'esposizione che verrà esposta durante la RR;
    \item \textbf{Verifica documenti}: vengono verificati e, se necessario, aggiustati i documenti prodotti nelle fasi precedenti. 
\end{itemize}


%consegna della \textsc{Revisione dei Requisiti}; prevede l’approfondimento delle tecnologie richieste attraverso lo studio autonomo, l'approfondimento dell'analisi dei requisiti con eventuali aggiornamenti, se necessario. Sarà inoltre previsto qualche contatto con la \emph{Zucchetti S.p.A.} per fugare eventuali dubbi. Per concludere verrà preparata la presentazione. Questa fase si conclude con la presentazione della \textsc{Revisione dei Requisiti} giorno 18/01/2021.

\begin{figure}[H]
\centering
\includegraphics[width=18cm]{src/img/gantt/4_2_consolidamento_dei_requisiti.png}
\caption{ Diagramma di Gantt della fase di consolidamento dei requisiti}
\end{figure}

%
%\subsection{Progettazione architetturale}%
%\label{sub:prog_arc}
%Periodo: dal 19/01/2021 al 01/03/2021.\\
%La fase di progettazione architetturale inizia subito dopo la presentazione della Revisione dei Requisiti.\\
%In questa fase verranno aggiornati e corretti i documenti redatti a seguito delle valutazioni ricevute, saranno adattati i requisiti in base ai feedback del proponente e verrà effettuato uno studio più approfondito delle tecnologie coinvolte, al fine di realizzare un \glossario{Proof of Concept} che fungerà da dimostratore della base tecnologica del progetto scelta. 

%Più in dettaglio, verranno aggiornati i seguenti documenti, portandoli alla versione 2.0.0 in corrispondenza della consegna:

%\begin{itemize}
%    \item Norme di Progetto;
%    \item Glossario;
%    \item Analisi dei Requisiti;
%    \item Piano di Progetto;
%    \item Piano di Qualifica.
%\end{itemize}

%Verrà anche preparata la presentazione della Technology Baseline, che servirà al gruppo per comprendere appieno l'utilizzo di tutte le tecnologie coinvolte per lo sviluppo del prodotto. Verrà scelta un'architettura per il codice identificando il design pattern che meglio si adatta alle nostre esigenze. Verrà sviluppato anche un Proof of Concept che servirà a verificare con il proponente le scelte effettuate per la scrittura del software. Ovviamente, sarà solo un prototipo attraverso il quale si mostreranno solo alcuni dei requisiti obbligatori richiesti, e nemmeno completi. Per le prossime fasi, questo prototipo verrà usato come base e migliorato fino al raggiungimento degli obiettivi prefissati.

%Durante la stesura della Technology Baseline sono previsti 2 incrementi:
%\begin{itemize}
%    \item Incremento 1 (dal 2021-02-11 al 2021-02-20): l'obiettivo principale è la codifica di alcuni use case concordati con il proponente. Ci si focalizzerà sull'interazione con la libreria D3.js per la creazione dei grafici Scatter Plot Matrix. In particolare sugli use case: UCO1.2, UCO2.2, UCO3, UCO3.1, UCO3.2, UCO3.3, UC4.2, UC4.2.1, UC4.2.4, UC4.2.5;
%\item Incremento 2 (dal 2021-02-21 al 2021-02-26): l'obiettivo, di questo periodo, sarà l'implementazione degli use case e del miglioramento generale di tutto il resto. Se gli use case citati al punto precedente saranno completati, si proverà ad implementare altri use case: UC4.2.2, UC4.2.3.
%\end{itemize}

%\begin{figure}[H]
%\centering
%\includegraphics[width=18cm]{src/img/gantt/02_RP.png}
%\caption{Diagramma di Gantt della fase di progettazione architetturale}
%\end{figure}


\subsection{Progettazione e codifica della technology baseline}
\label{sub:tech_baseline}
Periodo: dal 2021-01-19 al 2021-03-01. \\ \\
La fase di progettazione architetturale inizia subito dopo la presentazione della Revisione dei Requisiti.\\
In questa fase verranno aggiornati e corretti i documenti redatti a seguito delle valutazioni ricevute, saranno adattati i requisiti in base ai feedback del proponente e verrà effettuato uno studio più approfondito delle tecnologie coinvolte, al fine di realizzare un \glossario{Proof of Concept}.

\subsubsection{Attività}
\begin{itemize}
    \item \textbf{Revisione/Aggiornamento}: revisione e/o aggiornamento di tutti i documenti \emph{v1.0.0} prodotti fino ad ora.
    \item \textbf{Ricerca tecnologie}: scelta di strumenti e tecnologie usate nella successiva fase di codifica del \glossario{Proof of Concept} previo studio autonomo;
    \item \textbf{Progettazione}: progettazione dell'architettura di sistema dell'applicazione;
    \item \textbf{Codifica}: codifica del Proof of Concept. 
    \item \textbf{Verifica}.\\
\end{itemize}
In questo periodo vengono scelti alcuni dei requisiti più rilevanti individuati durante la fase di Analisi e sviluppati in vari incrementi.
I requisti sono stati concordati con il proponente \textsc{Zucchetti S.p.A}, come dichiarato nel documento \textsc{Verbale Esterno 2021-02-17}.%

\subsubsection{I incremento}
Vengono stabiliti i seguenti obiettivi:
\begin{itemize}
    \item Installazione e configurazione dell'ambiente di sviluppo Node.
\end{itemize}
\paragraph{Attività}
\noindent\textbf{Periodo (2021-01-19 - 2021-01-21)}
\begin{itemize}
    \item Progettazione;
    \item Codifica;
    \item Verifica delle funzionalità implementate.
\end{itemize}

\subsubsection{II incremento}
Vengono stabiliti i seguenti obiettivi:
\begin{itemize}
    \item Codifica del front-end dell'applicazione web \emph{HD Viz}.
\end{itemize}
\paragraph{Attività}
\noindent\textbf{Periodo (2021-01-22 - 2021-02-05)}
\begin{itemize}
    \item Progettazione;
    \item Codifica;
    \item Verifica delle funzionalità implementate.
\end{itemize}

\subsubsection{III incremento}
Vengono stabiliti i seguenti obiettivi:
\begin{itemize}
    \item Implementazione sistema di caricamento dati attraverso file CSV;
    \item Implementazione del parser dati e dell'assegnazione automatica delle label.
\end{itemize}
\paragraph{Attività}
\noindent\textbf{Periodo (2021-01-25 - 2021-02-08)}
\begin{itemize}
    \item Progettazione;
    \item Codifica;
    \item Verifica delle funzionalità implementate.
\end{itemize}

\subsubsection{IV incremento}
Vengono stabiliti i seguenti obiettivi:
\begin{itemize}
    \item Implementazione del grafico Scatter Plot Matrix per la visualizzazione dei dati importati.
\end{itemize}
\paragraph{Attività}
\noindent\textbf{Periodo (2021-02-09 - 2021-02-19)}
\begin{itemize}
    \item Progettazione;
    \item Codifica;
    \item Verifica delle funzionalità implementate.
\end{itemize}

\subsubsection{V incremento}
Vengono stabiliti i seguenti obiettivi:
\begin{itemize}
    \item Implementazione sistema di caricamento dati attraverso database.
\end{itemize}
\paragraph{Attività}
\noindent\textbf{Periodo (2021-02-12 - 2021-03-01)}
\begin{itemize}
    \item Progettazione;
    \item Codifica;
    \item Lettera di presentazione;
    \item Consuntivo di periodo;
    \item Verifica delle funzionalità implementate.
\end{itemize}


\begin{figure}[H]
    \centering
    \includegraphics[width=18cm]{src/img/gantt/4_3_tech_baseline.png}
    \caption{Diagramma di Gantt della fase di progettazione della technology baseline}
\end{figure}


%===============================%

%\subsection{Progettazione e codifica di dettaglio}%
%\label{sub:prog_dett}
%Periodo: dal 2021-03-01 al 2021-04-02.\\ \\
%Questa fase prevede la progettazione e l'implementazione delle funzionalità restanti dalla fase precedente; vi saranno aggiustamenti della pianificazione e aggiornamenti dei documenti. Verrà inoltre redatto il \textsc{\glossario{Manuale Utente}} e il \textsc{\glossario{Manuale Sviluppatore}}. \\
%Al completamento delle attività verrà redatta la lettera di presentazione per candidarsi alla \textsc{Revisione di Qualifica}. 
%Nella settimana successiva alla consegna della \textsc{Revisione di Qualifica} verrà redatta la presentazione da portare in sede di revisione programmata giorno 09/04/2021. 
%Questa fase viene divisa in due macro periodi.
%
%\subsubsection{Periodo I - Documenti}
%Questo periodo ha una durata di 11 giorni. Viene collocato subito dopo i risultati della \textsc{Revisione di Progettazione} stimata per il 2021-03-17.
%Le attività che verranno svolte in questo periodo sono:
%\begin{itemize}
%    \item Correzione dei documenti: vengono corretti i documenti secondo le indicazioni fornite al gruppo dal docente;
%    \item Incremento dei documenti: vengono migliorati i documenti con modifiche e aggiunte;
%    \item Verifica: attività svolta alla fine di ogni attività descritte in precedenza.
%\end{itemize}
%
%\subsubsection{Periodo II - Product Baseline e codifica}
%Le attività che verranno svolte in questo periodo sono:
%\begin{itemize}
%    \item Per ogni incremento individuato verranno svolte le attività seguenti: 
%    \begin{itemize}
%        \item Individuazione design pattern;
%        \item Creazione diagrammi attività;
%        \item Creazione diagrammi classi;
%        \item Stesura codice;
%        \item Stesura test;
%        \item Stesura \textsc{Manuale Utente} e \textsc{Manuale Sviluppatore};
%        \item Verifica.
%    \end{itemize}
%    \item Preparazione presentazione.
%\end{itemize}
%
%\begin{figure}[H]
%\centering
%\includegraphics[width=18cm]{src/img/gantt/03_RQ_alternative_2.png}
%\caption{Diagramma di Gantt della fase di progettazione e codifica di dettaglio}
%\end{figure}

%===========%

% \subsubsection{VI incremento}
% Vengono stabiliti i seguenti obiettivi:
% \begin{itemize}
%     \item Implementazione fetch dei dati importati dall'utente.
% \end{itemize}
% \paragraph{Attività}
% \noindent\textbf{Periodo (2021-03-01 - 2021-03-05)}
% \begin{itemize}
%     \item Progettazione;
%     \item Codifica;
%     \item Verifica delle funzionalità implementate.
% \end{itemize}


% \subsubsection{VII incremento}
% Vengono stabiliti i seguenti obiettivi:
% \begin{itemize}
%     \item Implementazione delle funzionalità mancanti al grafico Scatter Plot Matrix.
% \end{itemize}
% \paragraph{Attività}
% \noindent\textbf{Periodo (2021-03-06 - 2021-03-11)}
% \begin{itemize}
%     \item Progettazione;
%     \item Codifica;
%     \item Stesura \textsc{Manuale Utente};
%     \item Verifica delle funzionalità implementate.
% \end{itemize}


% \subsubsection{VIII incremento}
% Vengono stabiliti i seguenti obiettivi:
% \begin{itemize}
%     \item Implementazione annullamento delle modifiche.
% \end{itemize}
% \paragraph{Attività}
% \noindent\textbf{Periodo (2021-03-09 - 2021-03-15)}
% \begin{itemize}
%     \item Progettazione;
%     \item Codifica;
%     \item Stesura \textsc{Manuale Utente};
%     \item Verifica delle funzionalità implementate.
% \end{itemize}


% \subsubsection{IX incremento}
% Vengono stabiliti i seguenti obiettivi:
% \begin{itemize}
%     \item Implementazione del grafico Force Field per la visualizzazione dei dati importati.
%     \item Implementazione del grafico Distance Map per la visualizzazione dei dati importati.
% \end{itemize}
% \paragraph{Attività}
% \noindent\textbf{Periodo (2021-03-14 - 2021-03-20)}
% \begin{itemize}
%     \item Progettazione;
%     \item Codifica;
%     \item Stesura \textsc{Manuale Utente};
%     \item Verifica delle funzionalità implementate.
% \end{itemize}


% %dopo Incremento IX
% \subsubsection{X incremento}
% Vengono stabiliti i seguenti obiettivi:
% \begin{itemize}
%     \item Implementazione modifiche grafici con matrice delle distanze (Force Field e Distance Map).
% \end{itemize}
% \paragraph{Attività}
% \noindent\textbf{Periodo (2021-03-21 - 2021-03-25)}
% \begin{itemize}
%     \item Progettazione;
%     \item Codifica;
%     \item Stesura \textsc{Manuale Utente};
%     \item Verifica delle funzionalità implementate.
% \end{itemize}

% \subsubsection{XI incremento}
% Vengono stabiliti i seguenti obiettivi:
% \begin{itemize}
%     \item Implementazione del grafico Heat Map per la visualizzazione dei dati importati.
% \end{itemize}
% \paragraph{Attività}
% \noindent\textbf{Periodo (2021-03-20 - 2021-03-25)}
% \begin{itemize}
%     \item Progettazione;
%     \item Codifica;
%     \item Stesura \textsc{Manuale Utente};
%     \item Verifica delle funzionalità implementate.
% \end{itemize}


% \subsubsection{XII incremento}
% Vengono stabiliti i seguenti obiettivi:
% \begin{itemize}
%     \item Implementazione del grafico Proiezione Lineare Multi Asse per la visualizzazione dei dati importati.
% \end{itemize}
% \paragraph{Attività}
% \noindent\textbf{Periodo (2021-03-22 - 2021-03-26)}
% \begin{itemize}
%     \item Progettazione;
%     \item Codifica;
%     \item Stesura \textsc{Manuale Utente};
%     \item Verifica delle funzionalità implementate.
% \end{itemize}


% \subsubsection{XIII incremento}
% Vengono stabiliti i seguenti obiettivi:
% \begin{itemize}
%     \item Implementazione della visualizzazione degli errori.
% \end{itemize}
% \paragraph{Attività}
% \noindent\textbf{Periodo (2021-03-27 - 2021-03-01)}
% \begin{itemize}
%     \item Progettazione;
%     \item Codifica;
%     \item Stesura \textsc{Manuale Utente};
%     \item Consuntivo di periodo;
%     \item Lettera di presentazione;
%     \item Verifica delle funzionalità implementate.
% \end{itemize}

% \begin{figure}[H]
%     \centering
%     \includegraphics[width=18cm]{src/img/gantt/03_RQ_alternative.png}
%     \caption{Diagramma di Gantt della fase di progettazione e codifica di dettaglio}
% \end{figure}


%======================================%

\subsection{Progettazione e codifica di dettaglio}%
\label{sub:prog_dett}
\subsubsection{Progettazione iniziale in data 2021-03-10}
Periodo: dal 2021-03-10 al 2021-04-02.\\ \\
Inizialmente in questa fase si era previsto di proseguire il software realizzato in precedenza, tuttavia si è ritenuto opportuno ripartire da zero e realizzare prima un'architettura 
base che aiutasse successivamente nella programmazione. 
In questa fase si realizza la presentazione per la Revisione di Progettazione, stimata per giorno 2021-03-17, e si procede subito con la modifica di alcuni documenti quali
il \textsc{Piano di Progetto} per tracciare una prima pianificazione, e il \textsc{Piano di Qualifica} in quanto vi è necessità di iniziare a stilare obiettivi e metriche relativi al prodotto software.
Complessivamente questa fase è caratterizzata da due processi, quello di \emph{documentazione} e quello di \emph{progettazione e codifica}. Ad ogni incremento segue anche la rispettiva verifica. \\
\textbf{Documenti:}
\begin{itemize}
    \item \textbf{Piano di progetto:}
        \begin{itemize}
            \item \textbf{Incremento I:} Si pianifica finemente la fase corrente, in quanto il tempo è poco e la successiva consegna è molto vicina;
            \item \textbf{Incremento II:} A ridosso della codifica del software si individuano gli incrementi da implementare;
            \item \textbf{Incremento III:} Si correggono eventuali errori segnalati tra gli esiti della Revisione di Progettazione;
        \end{itemize}
        \item \textbf{Piano di qualifica:}
        \begin{itemize}
            \item \textbf{Incremento I:} Si stilano le metriche e gli obiettivi relativi al prodotto software;
            \item \textbf{Incremento II:} Si correggono eventuali errori segnalati tra gli esiti della Revisione di Progettazione;
            \item \textbf{Incremento III:} Si aggiornano i capitoli relativi ai test svolti;
        \end{itemize}
        \item \textbf{Norme di progetto:}
        \begin{itemize}
            \item \textbf{Incremento I:} Si normano i diagrammi UML, i test che verranno effettuati, ed eventuali cambiamenti nell'amministrazione della repository;
            \item \textbf{Incremento II:} Si correggono eventuali errori segnalati tra gli esiti della Revisione di Progettazione;
        \end{itemize}
        \item \textbf{Manuale sviluppatore:}
        \begin{itemize}
            \item \textbf{Incremento I:} Si inizia la stesura generale del manuale sviluppatore;
            \item \textbf{Incremento II:} Quando l'architettura è pronta, si termina la stesura del manuale;
        \end{itemize}
        \item \textbf{Manuale utente:}
        \begin{itemize}
            \item \textbf{Incremento I:} Si inizia la stesura generale del manuale utente;
            \item \textbf{Incremento II:} Quando l'applicativo software è quasi terminato, si termina la stesura del manuale;
        \end{itemize}
\end{itemize}

\textbf{HD-Viz:}
\begin{itemize}
    \item \textbf{Individuazione dell'architettura e creazione diagrammi:}
    \begin{itemize}
        \item \textbf{Incremento I:} Scelta del modello;
        \item \textbf{Incremento II:} Definizione dataset;
        \item \textbf{Incremento III:} Definizione grafici;
        \item \textbf{Incremento IV:} Definizione presenter e interfacce;
        \item \textbf{Incremento V:} Definizione view;
        \item \textbf{Incremento VI:} Definizione interazioni tra i tipi;      
    \end{itemize}

    \item \textbf{Codifica:}
    \begin{itemize}
        \item \textbf{Incremento I:} Sviluppo ambiente di lavoro;
        \item \textbf{Incremento II:} Sviluppo inserimento dati tramite file;
        \item \textbf{Incremento III:} Sviluppo Scatterplot Matrix;
        \item \textbf{Incremento IV:} Sviluppo modifica grafico Scatterplot Matrix;
        \item \textbf{Incremento V:} Sviluppo Distance Map e algoritmi di scelta;
	
        \item \textbf{Incremento VI:} Sviluppo inserimento dati tramite database;
        \item \textbf{Incremento VII:} Sviluppo Force Field;
        \item \textbf{Incremento VIII:} Sviluppo Heat Map;
        \item \textbf{Incremento IX:} Sviluppo PLMA;
        \item \textbf{Incremento X:} Sviluppo modifiche Distance Map;
        \item \textbf{Incremento XI:} Sviluppo modifica colori;
        \item \textbf{Incremento XII:} Sviluppo modifica dimensioni;
        \item \textbf{Incremento XIII:} Sviluppo modifica etichette;
        \item \textbf{Incremento XIV:} Sviluppo visualizzazione errori;
    \end{itemize}
\end{itemize}

\begin{figure}[H]
\centering
\includegraphics[width=18cm]{src/img/gantt/pianificazione_iniz_documenti.jpg}
\caption{Diagramma di Gantt della fase di progettazione e codifica}
\end{figure}

\begin{figure}[H]
    \centering
    \includegraphics[width=18cm]{src/img/gantt/pianificazione_iniz_pb.jpg}
    \caption{Diagramma di Gantt della fase di progettazione e codifica}
    \end{figure}


\subsubsection{Progettazione riveduta in data 2021-04-13}
Periodo: dal 2021-04-13 al data da definire.\\ \\
La pianificazione realizzata inizialmente si è dimostrata inconsistente in quanto i tempi erano troppo stretti. Inoltre essa si basava sul fatto che i risultati sarebbero stati 
positi sia per quanto riguarda gli esiti della Revisione di Progettazione, sia per quanto riguarda i colloqui della Product Baseline. Infatti inizialmente non si sapeva quando sarebbero
avvenuti tali colloqui, e si sperava di concludere tutto per tempo.\\
Il blocco imposto nei colloqui per la Product Baseline ha richiesto di rivedere la pianificazione dei successivi mesi; sia la correzione dell'architettura realizzata, 
sia i vari errori emersi tra gli esisti della Revisione di Progettazione, hanno comportato parecchio ritardo. Il gruppo ha comunque iniziato a scrivere codice e contemporaneamente 
i test necessari sia per valutare il raggiungimento degli obiettivi di qualità, sia per dimostrare effettivamente che l'architettura ideata è valida. \\
Come nella precedente pianificazione, la fase si è composta da due processi, documentazione e programmazione. Per quanto riguarda i documenti, gli incrementi previsti 
sono gli stessi della pianificazione precedente. Per quanto riguarda invece la codifica software si è dovuto rivalutare quanti requisiti sviluppare. \\

\textbf{HD-Viz:}
\begin{itemize}
    \item \textbf{Correzione dell'architettura:}
    \begin{itemize}
        \item \textbf{Incremento I:} Si correggono le interazioni tra i tipi;
        \item \textbf{Incremento II:} Si corregge il server;
        \item \textbf{Incremento III:} Si aggiungono nuovi tipi;
    \end{itemize}
    \item \textbf{Codifica:}
    \begin{itemize}        
        \item \textbf{Incremento I:} Sviluppo ambiente di lavoro;
        \item \textbf{Incremento II:} Sviluppo inserimento dati tramite file;
        \item \textbf{Incremento III:} Sviluppo Scatterplot Matrix;
        \item \textbf{Incremento IV:} Sviluppo modifica grafico Scatterplot Matrix;
        \item \textbf{Incremento V:} Sviluppo Distance Map e algoritmi di scelta;
	
        \item \textbf{Incremento VI:} Sviluppo inserimento dati tramite database;
        \item \textbf{Incremento VII:} Sviluppo Force Field;
        \item \textbf{Incremento VIII:} Sviluppo Heat Map;
        \item \textbf{Incremento IX:} Sviluppo PLMA;
        \item \textbf{Incremento X:} Sviluppo modifiche Distance Map;
        \item \textbf{Incremento XI:} Sviluppo modifica colori;
        \item \textbf{Incremento XII:} Sviluppo modifica dimensioni;
        \item \textbf{Incremento XIII:} Sviluppo modifica etichette;
        \item \textbf{Incremento XIV:} Sviluppo visualizzazione errori;

    \end{itemize}
\end{itemize}


\begin{landscape}
    \begin{sidewaysfigure}
    \centering
    \includegraphics[width=1\textwidth, width=200mm, angle=90]{src/img/gantt/pianificazione_riveduta_RQ.jpg}
    \caption{Diagramma di Gantt della fase di progettazione e codifica}
    \end{sidewaysfigure}
\end{landscape}%}

%======================================%

\subsection{Validazione e Collaudo}%
\label{sub:valid_coll}
Periodo: X al Y \\ \\ 
Questa fase terminerà con la \textsc{Revisione di Accettazione}. In questo periodo unico verranno svolte le seguenti attività:

\begin{itemize}
    \item Codifica: codifica della versione finale del prodotto, realizzando i requisiti opzionali rimanenti;
    \item Correzione e verifica dei documenti precedentemente redatti;
    \item Aggiornamento e verifica manuali: stesura della versione finale del \textsc{Manuale Utente} e \textsc{Manuale Sviluppatore};
    \item Validazione: verifica di conformità rispetto agli obiettivi;
    \item Collaudo: test di collaudo per ogni funzionalità del prodotto;
    \item Preparazione presentazione.
\end{itemize}


\begin{figure}[H]
\centering
\includegraphics[width=18cm]{src/img/gantt/04_RA_alternative.png}
\caption{Diagramma di Gantt della fase di validazione e collaudo}
\end{figure}

\end{document}

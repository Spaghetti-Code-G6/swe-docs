\documentclass[../glossario.tex]{subfiles}


\begin{document}

\subsection*{Back-end}
\phantomsection
\addcontentsline{toc}{subsection}{Back-end}
È la parte che elabora i dati provenienti dal front-end e ne permette il suo funzionamento, quest'ultima si occupa della gestione dell'interfaccia utente.

\subsection*{Blue-Magenta-Yellow}
\addcontentsline{toc}{subsection}{Blue-Magenta-Yellow}
Gradiente di colori che spazia dal blu al magenta al giallo, usato per visualizzare i dati nelle visualizzazioni Distance Map ed Heat Map. 

\subsection*{Branch}
\addcontentsline{toc}{subsection}{Branch}
Ambiente di un repository isolato comprendente un insieme univoco di modifiche al codice e caratterizzato da un nome specifico. Tratto distintivo di Git.

\subsection*{Browser}
\addcontentsline{toc}{subsection}{Browser}
Programma per navigare in Internet che inoltra la richiesta di un documento alla rete e ne consente la visualizzazione una volta arrivato.
    
\end{document}

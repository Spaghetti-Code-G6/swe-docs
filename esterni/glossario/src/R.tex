\documentclass[../glossario.tex]{subfiles}


\begin{document}

\subsection*{Rilascio}
\phantomsection
\addcontentsline{toc}{subsection}{Rilascio}
Consegna al committente della versione definitiva del prodotto software, pronto per essere inserito nell’ambiente di lavoro cui è destinato. 


\subsection*{Repository} 
\addcontentsline{toc}{subsection}{Repository}
Ambiente di un sistema informativo, in cui vengono gestiti i metadati, attraverso tabelle relazionali.

\subsection*{Requisiti} 
\addcontentsline{toc}{subsection}{Requisiti}
Qualità richieste e necessarie per un determinato scopo.

\subsection*{Responsabile di progetto} 
\addcontentsline{toc}{subsection}{Responsabile di progetto}
Colui che pianifica il progetto, assegna le persone ai ruoli giusti e rappresenta il progetto presso il fornitore e il committente. Vedi sezione \S4.2.2.1 delle \textsc{Norme di Progetto}.

\subsection*{Responsive} 
\addcontentsline{toc}{subsection}{Responsive}
Tecnica di web design per la realizzazione di siti in grado di adattarsi graficamente in modo automatico al dispositivo coi quali vengono visualizzati, riducendo al minimo la necessità dell'utente di ridimensionare e scorrere i contenuti.

\subsection*{Rischio} 
\addcontentsline{toc}{subsection}{Rischio}
Potenzialità che un'azione o un'attività scelta (includendo la scelta di non agire) porti a una perdita o ad un evento indesiderabile.

    
\end{document}
\documentclass[../glossario.tex]{subfiles}

\begin{document}

\subsection*{Kanban Board}
\phantomsection
\addcontentsline{toc}{subsection}{Kanban Board}
È uno strumento di base costituito da una bacheca o lavagna, o tabellone detto Kanban Board. Il suo scopo è quello di far emergere le inefficienze e le strozzature nel lavoro del team. L’effetto complessivo è quello di aumentare la produttività. L’introduzione della Kanban Board consente di tenere sotto controllo costantemente il flusso delle lavorazioni perché spesso i problemi dipendono dal non conoscere:
\begin{itemize}
\item a che punto è il lavoro;
\item cosa manca ancora da fare;
\item cosa riduce la produttività.
\end{itemize}

\subsection*{Kebab case }
\phantomsection
\addcontentsline{toc}{subsection}{Kebab case }
Il kebab case (o kebab-case) è la pratica di scrivere gli identificatori separando le parole che li compongono tramite trattini.

\end{document}

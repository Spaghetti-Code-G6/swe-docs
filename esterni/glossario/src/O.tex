\documentclass[../glossario.tex]{subfiles}

\begin{document}

\subsection*{Open source}
\phantomsection
\addcontentsline{toc}{subsection}{Open source}
In informatica, si indica un tipo di software o il suo modello di sviluppo o distribuzione. Un software open source è reso tale per mezzo di una licenza attraverso cui i detentori dei diritti favoriscono la modifica, lo studio, l'utilizzo e la redistribuzione del codice sorgente.


\subsection*{Orientato agli oggetti}
\phantomsection
\addcontentsline{toc}{subsection}{Orientato agli oggetti}
Paradigma di programmazione, che prevede di raggruppare in un'unica entità (la classe) sia le strutture dati che le procedure che operano su di esse, creando appunto un oggetto software.


\end{document}

\documentclass[../glossario.tex]{subfiles}


\begin{document}

\subsection*{CSS}
\phantomsection
\addcontentsline{toc}{subsection}{CSS}
Il CSS (Cascading Style Sheets), è un linguaggio usato per definire la formattazione di documenti HTML, XHTML e XML, ad esempio i siti web e relative pagine web.

\subsection*{CSV}
\phantomsection
\addcontentsline{toc}{subsection}{CSV}
Comma-separated values (abbreviato in CSV) è un formato di file basato su file di testo utilizzato per l'importazione ed esportazione di una tabella di dati.

\subsection*{CamelCase}
\phantomsection
\addcontentsline{toc}{subsection}{CamelCase}
Si tratta di una notazione in cui si scrivono parole composte o frasi unendo tutte le parole tra loro, lasciando le loro iniziali maiuscole. La prima lettera è minuscola. Se fosse maiuscola si parlerebbe di PascalCase.

\subsection*{Canberra}
\phantomsection
\addcontentsline{toc}{subsection}{Canberra}
La distanza di Canberra è una distanza che può venire usata senza ricorrere alle operazioni di standardizzazione. È inoltre poco sensibile all'asimmetria, alla distribuzione delle variabili ed alla presenza di outliers.

\subsection*{Capitolato}
\phantomsection
\addcontentsline{toc}{subsection}{Capitolato}
Documento tecnico redatto dal cliente, generalmente allegato ad un contratto di appalto, in cui sono specificati i vincoli contrattuali per lo sviluppo di determinato prodotto software.


\subsection*{Ciclo di Deming}
\phantomsection
\addcontentsline{toc}{subsection}{Ciclo di Deming}
È un metodo di gestione iterativo in quattro fasi utilizzato per il controllo e il miglioramento continuo dei processi e dei prodotti.


\subsection*{Ciclo di vita}
\phantomsection
\addcontentsline{toc}{subsection}{Ciclo di vita}
Modo in cui una metodologia di sviluppo scompone l'attività di realizzazione di prodotti software in sotto attività fra loro coordinate, il cui risultato finale è la realizzazione del prodotto stesso e tutta la documentazione ad esso associata: fasi tipiche includono lo studio o analisi, la progettazione, la realizzazione, il collaudo, la messa a punto, l'installazione, la manutenzione e l'estensione, il tutto ad opera di uno o più sviluppatori software.

\subsection*{Cloud}
\phantomsection
\addcontentsline{toc}{subsection}{Cloud}
Modello di conservazione dati su computer in rete dove i dati stessi sono memorizzati su molteplici server virtuali generalmente ospitati presso strutture di terze parti o su server dedicati. I dati sono accessibili ovunque e in qualsiasi momento utilizzando una connessione ad Internet.

\subsection*{Consuntivo}
\phantomsection
\addcontentsline{toc}{subsection}{Consuntivo}
Rendiconto dei risultati di un dato periodo di attività di un ente o di un'impresa.

\subsection*{Container}
\phantomsection
\addcontentsline{toc}{subsection}{Container}
Nei sistemi di virtualizzazione che operano a livello di sistema operativo, un container è un’istanza isolata nello spazio utente.

\subsection*{CoolWarm}
\phantomsection
\addcontentsline{toc}{subsection}{CoolWarm}
Gradiente di colori che spazia dal blu al bianco al rosso, usato per visualizzare i dati nelle visualizzazioni Distance Map ed Heat Map.

\subsection*{Constant Case}
\phantomsection
\addcontentsline{toc}{subsection}{Constant Case}
Notazione che prevede una stringa maiuscola con un trattino basso tra le parole.

\subsection*{Criticità}
\phantomsection
\addcontentsline{toc}{subsection}{Criticità}
Distanza troppo breve tra attività dipendenti.

\end{document}

\documentclass[../glossario.tex]{subfiles}

\begin{document}

\subsection*{Gantt, Diagrammi di}
\phantomsection
\addcontentsline{toc}{subsection}{Gantt, Diagrammi di}
Strumenti di supporto alla gestione dei progetti; permettono la rappresentazione grafica di un calendario di attività utile per pianificare, coordinare e tracciare specifiche attività.


\subsection*{Git}
\addcontentsline{toc}{subsection}{Git}
Software di controllo di versione distribuito utilizzabile da interfaccia a riga di comando.

\subsection*{GitHub} 
\addcontentsline{toc}{subsection}{GitHub}
Servizio di hosting per progetti software. È una implementazione dello strumento di controllo versione distribuito Git.

\subsection*{GitHub Projects} 
\addcontentsline{toc}{subsection}{GitHub\_projects}
Insieme di board relative ad un progetto che permettono di organizzare e dare la priorità ai diversi compiti che concernono un'attività

\subsection*{Google Docs} 
\addcontentsline{toc}{subsection}{Google Docs}
Applicazione web gratuita di elaborazione del testo.

\subsection*{Grafico} 
\addcontentsline{toc}{subsection}{Grafico}
Rappresentazione grafica di un fenomeno o di una funzione matematica.

\subsection*{Gulpease, Indice di}
\addcontentsline{toc}{subsection}{Gulpease, Indice di}
Indice di leggibilità di un testo tarato sulla lingua italiana. Rispetto ad altri ha il vantaggio di utilizzare la lunghezza delle parole in lettere anziché in sillabe, semplificandone il calcolo automatico.

    
\end{document}
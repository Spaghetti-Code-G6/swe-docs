\documentclass[../glossario.tex]{subfiles}




\begin{document}

\subsection*{Machine learning}
\phantomsection
\addcontentsline{toc}{subsection}{Machine learning}
Branca dell'intelligenza artificiale che utilizza metodi statistici per migliorare la performance di un algoritmo nell'identificare pattern nei dati.


\subsection*{Manhattan} 
\addcontentsline{toc}{subsection}{Manhattan}
La distanza Manhattan tra due punti è la somma del valore assoluto delle differenze delle coordinate dei due punti.

\subsection*{Manuale d'Utente} 
\addcontentsline{toc}{subsection}{Manuale d'Utente}
Manuale d'uso di un software, destinato all'utente finale.

\subsection*{Metadati}
\addcontentsline{toc}{subsection}{Metadati}
Sono dati che descrivono il contenuto, la struttura e il contesto dei documenti e la loro gestione nel tempo.


\subsection*{Metriche} 
\addcontentsline{toc}{subsection}{Metriche}
Standard per la misura di alcune proprietà del software o delle sue specifiche.

\subsection*{Milestone} 
\addcontentsline{toc}{subsection}{Milestone}
Importante traguardo intermedio nello svolgimento di un progetto.

\subsection*{Minkowski} 
\addcontentsline{toc}{subsection}{Milestone}
La distanza Minkowski è una generalizzazione del calcolo della distanza distanza tra due punti.

\subsection*{Modello} 
\addcontentsline{toc}{subsection}{Modello}
L'oggetto o il termine atto a fornire un conveniente schema di punti di riferimento ai fini della riproduzione o dell'imitazione, talvolta dell'emulazione.

\subsection*{Modello di sviluppo} 
\addcontentsline{toc}{subsection}{Modello di sviluppo}
Principio teorico che indica il metodo da seguire nel progettare e nello scrivere un programma.

    
\end{document}

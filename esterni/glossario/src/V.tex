\documentclass[../glossario.tex]{subfiles}


\begin{document}


\subsection*{Validazione}
\phantomsection
\addcontentsline{toc}{subsection}{Validazione}
La validazione accerta che il prodotto realizzato sia conforme alle attese.

\subsection*{Verifica}
\phantomsection
\addcontentsline{toc}{subsection}{Verifica}
La verifica accerta che l’esecuzione delle attività di processo attuate nel periodo in esame non abbia introdotto errori.

\subsection*{Verificatore}
\phantomsection
\addcontentsline{toc}{subsection}{Verificatore}
Figura che si occupa delle attività di controllo di quanto svolto dagli altri membri del gruppo. Vedi sezione \S4.2.2.6 delle \textsc{Norme di Progetto}.

\subsection*{Versionamento}
\phantomsection
\addcontentsline{toc}{subsection}{Versionamento}
Funzione che dà la possibilità di gestire le varie versioni di uno stesso file e che tiene traccia delle modiche effettuate.

\subsection*{Versionamento Semantico}
\phantomsection
\addcontentsline{toc}{subsection}{Versionamento Semantico}
Strategia di versionamento adottata dal gruppo Spaghetti Code. Maggiori informazioni riguardo la sua definizione ed implementazione si trovano nelle norme ed al riferimento \url{https://semver.org/lang/it/}.

\end{document}

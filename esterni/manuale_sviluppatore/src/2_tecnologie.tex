\documentclass[../manuale_sviluppatore.tex]{subfiles}
\begin{document}
Qui di seguito verranno descritte le tecnologie e le librerie di terze parti utilizzate per il progetto.


\subsection{Tecnologie}

\paragraph{JSON}
JSON è il formato usato per lo scambio di dati fra client e server. contiene un elenco delle dipendenze di origine specificandone la versione. 
È sempre leggibile ad occhio umano e non necessita di alcun processo di compilazione particolare per essere modificato. 

\paragraph{CSV}
Il CSV è un formato di file basato su file di testo, utilizzato
per l’importazione di una tabella di dati.  È il formato richiesto per caricare da locale i dati.

\paragraph{Javascript}
È un linguaggio di programmazione orientato agli oggetti e agli eventi comunemente utilizzato nella programmazione Web lato client (esteso poi anche al lato server). 
È alla base di HD-Viz, infatti viene utilizzato sia per il lato front-end sia per il lato back-end. 

\paragraph{HTML5}
HTML5 è un linguaggio di markup per la strutturazione di pagine web. Viene utilizzato
in HD-Viz insieme a javascript per definire gli elementi strutturali.

\paragraph{CSS3}
Il CSS è un linguaggio usato per definire la formattazione di documenti HTML5, come ad esempio siti web e le relative pagine web. 
L’uso del CSS permette la separazione dei contenuti delle pagine HTML dal loro layout e permette di dare uno stile chiaro e ampliabile nel tempo. 

\paragraph{Node.js}
Node.js è una runtime di JavaScript open-source multipiattaforma usato per l’esecuzione di codice JavaScript. Molti dei suoi moduli base sono scritti in JavaScript. 
È parte integrante della fase di codifica di HD-Viz in quanto ha una funzione primaria.\\
\begin{itemize}
    \item Versione utilizzata: 14.5;
    \item Link per il download: \url{https://nodejs.org/it/download/}
\end{itemize}



\subsection{Librerie di terze parti}

\paragraph{D3.js}
È una libreria JavaScript per creare visualizzazioni dinamiche ed interattive partendo da dati organizzati, visibili attraverso un comune browser. 
In HD-Viz viene utilizzato per la creazione e manipolazione dei grafici.

\paragraph{JQuery}
È una libreria di javascript per applicazioni web; è di tipo open source. Nasce con l'obiettivo di semplificare la selezione, la manipolazione, la gestione degli eventi e l'animazione di elementi DOM in pagine HTML.

\paragraph{Express.js}
Si tratta di un micro-framework per Node.js di tipo open source, è un ambiente di runtime che consente agli sviluppatori di creare tutti i tipi di strumenti e applicazioni lato server in JavaScript.

\paragraph{ESLint}
È uno strumento di analisi statica per il codice JavaScript. È utilizzato per identificare gli errori presenti nel codice senza doverne fare la build, e determina anche la qualità del codice scritto. 
Permette inoltre di essere configurato dallo sviluppatore con regole di analisi personalizzate.

\begin{itemize}
    \item Versione utilizzata: ;
    \item Link Repository NPM: https://www.npmjs.com/package/eslint 
\end{itemize}

\paragraph{Jest}
Framework di test open-source sviluppato da Facebook; viene utilizzato per testare codice scritto in JavaScript. 
\begin{itemize}
    \item Versione utilizzata: ;
    \item Link Repository NPM: https://www.npmjs.com/package/jest 
\end{itemize}

\paragraph{Sequelize}
È un ORM(Object Relational Mapping) che mappa oggetti a database relazionali attraverso lo standard SQL; grazie ad esso funziona con tutti i database.

\paragraph{Selenium}
È un tool open source per l'automazione di web testing. Permette quindi di specificare istruzioni per automatizzare le interazioni con il sito o l’applicazione web inoltre può lanciare test su diversi sistemi operativi e browser, anche in parallelo. 
Inoltre permette l’integrazione dei test su sistemi come Jenkins, TestNG e Maven agevolando i processi agili e di continuous integration.


\end{document}
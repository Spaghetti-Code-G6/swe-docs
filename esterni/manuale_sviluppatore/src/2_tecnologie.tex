\documentclass[../manuale_sviluppatore.tex]{subfiles}
\begin{document}
Qui di seguito verranno descritte le tecnologie e le librerie di terze parti utilizzate per il progetto.


\subsection{Tecnologie}

\paragraph{JSON}
\glossario{JSON} è il formato usato per lo scambio di dati fra client e server. Contiene un elenco delle dipendenze di origine specificandone la versione. 
È sempre leggibile ad occhio umano e non necessita di alcun processo di compilazione particolare per essere modificato. 

\paragraph{CSV}
Il \glossario{CSV} è un formato di file basato su file di testo, utilizzato per l’importazione di una tabella di dati. 
È il formato richiesto per caricare da locale i dati.

\paragraph{JavaScript}
\glossario{JavaScript} è un linguaggio di programmazione orientato agli oggetti e agli eventi comunemente utilizzato nella programmazione Web lato client (esteso poi anche al lato server). 
È alla base di HD-Viz, infatti viene utilizzato sia per il lato front-end sia per il lato back-end. 

\paragraph{HTML5}
\glossario{HTML5} è un linguaggio di markup per la strutturazione di pagine web. 
Viene utilizzato in HD-Viz insieme a JavaScript per definire gli elementi strutturali.

\paragraph{CSS3}
Il \glossario{CSS} è un linguaggio usato per definire la formattazione di documenti HTML5, XHTML e XML, come ad esempio siti web e le relative pagine web. 
L’uso del CSS permette la separazione dei contenuti delle pagine HTML dal loro layout e garantisce uno stile chiaro e ampliabile nel tempo. 

\paragraph{Node.js}
\glossario{Node.js} è una runtime di JavaScript \glossario{open source} multipiattaforma usato per l’esecuzione di codice JavaScript.  
È parte integrante della fase di codifica di HD-Viz in quanto ha una funzione primaria.\\
\begin{itemize}
    \item Versione utilizzata: 14.5;
    \item Link per il download: \url{https://nodejs.org/it/download/}
\end{itemize}



\subsection{Librerie di terze parti}

\paragraph{D3.js}
\glossario{D3.js} è una libreria JavaScript per creare visualizzazioni dinamiche ed interattive partendo da dati organizzati, visibili attraverso browser. 
In HD-Viz viene utilizzato per la creazione e manipolazione dei grafici.

\paragraph{JQuery}
\glossario{JQuery} è una libreria di JavaScript per applicazioni web; è di tipo open source. Nasce con l'obiettivo di semplificare la selezione, la manipolazione, la gestione degli eventi e l'animazione di elementi DOM in pagine HTML.

\paragraph{Express.js}
\glossario{Express} è un framework per Node.js di tipo open source, è un ambiente di runtime che consente agli sviluppatori di creare tutti i tipi di strumenti e applicazioni lato server in JavaScript.

\paragraph{ESLint}
\glossario{ESLint} è uno strumento di analisi statica per il codice JavaScript. È utilizzato per identificare gli errori presenti nel codice senza doverne fare la build; 
permette inoltre di essere configurato dallo sviluppatore con regole di analisi personalizzate.

\begin{itemize}
    \item Versione utilizzata: 7.29;
    \item Link Repository NPM: https://www.npmjs.com/package/eslint 
\end{itemize}

\paragraph{Jest}
\glossario{Jest} è un framework di test open-source sviluppato da Facebook. Viene utilizzato per testare codice scritto in JavaScript. 
\begin{itemize}
    \item Versione utilizzata: 26;
    \item Link Repository NPM: https://www.npmjs.com/package/jest 
\end{itemize}

\paragraph{Sequelize}
\glossario{Sequelize} è un ORM (Object Relational Mapping) che mappa oggetti a database relazionali attraverso lo standard SQL.

\paragraph{Selenium}
\glossario{Selenium} è un tool open source per l'automazione di web testing, permette quindi di specificare istruzioni per automatizzare le interazioni con il sito o l’applicazione web. 
Può lanciare test su diversi sistemi operativi e browser, anche in parallelo, inoltre permette l’integrazione dei test su sistemi come Jenkins, TestNG e Maven agevolando i processi agili e di continuous integration.

\paragraph{ML-HClust}
È una libreria che permette di fare \glossario{clustering} gerarchico. È stata impiegata per realizzare il clustering gerarchico della Distance Matrix.

\paragraph{noUIslider}
È un framework consigliato da Materialize che permette di gestire gli slider nei grafici. 
\end{document}

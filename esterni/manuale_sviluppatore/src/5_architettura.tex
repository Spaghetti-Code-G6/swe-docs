\documentclass[../manuale_sviluppatore.tex]{subfiles}

\begin{document}

\subsection{Design Pattern architetturale: MVP}
Il design pattern architetturale scelto per il prodotto è il \glossario{Model View Presenter}, di 
conseguenza le componenti principali del prodotto sono le seguenti:
\begin{itemize}
	\item \textbf{Model}: tra i model figurano tutti i tipi che hanno il compito di mantenere i dati 
	del dominio, nello specifico vi sono \emph{Dataset} (rappresentante il dataset caricato 
	nell'applicazione), \emph{DistanceMatrix} (rappresentante la matrice delle distanze associata 
	al dataset caricato)e i modelli delle visualizzazioni che preservano la configurazione delle 
	stesse;
	\item \textbf{Presenter}: tra i presenter figurano tutti i tipi che implementano la logica di 
	business e la logica di visualizzazione, in particolare vi sono i presenter delle 
	visualizzazioni, dei \glossario{components} e i manager dell'applicazione che ne gestiscono il 
	corretto funzionamento;
	\item \textbf{View}: le view sono dei tipi sostanzialmente passivi che visualizzano i dati di 
	dominio secondo la configurazione dell'applicazione, ed indirizzano gli eventi ai presenter 
	affinchè essi li elaborino e producano degli aggiornamenti coerenti con li stessi; tra le view 
	vi sono il codice HTML dei grafici e i diversi tipi degli elementi di modifica dei grafici.
\end{itemize}

Nell'architettura di \emph{HD-Viz} ciascun component è composto da un presenter e da una view, 
non necessita di un modello specifico in quanto i dati dipendono dal modello dei grafici, dal 
dataset o dalla matrice delle distanze. Nel caso dei grafici invece la view viene realizzata 
tramite le funzionalità di D3.

\begin{figure}[H]
	\centering
	\includegraphics[width=18cm]{src/img/patternMVP.jpg}
	\caption{Pattern MVP}
\end{figure}

\subsection{Architettura di dettaglio}
\subsubsection{Il client}

Per la definizione dei package del prodotto si è deciso di seguire il principio di progettazione 
\emph{Common Reuse Principle}, di modo da avere tipi che vengono utilizzati assieme all'interno 
dello stesso package.

\begin{figure}[H]
	\centering
	\includegraphics[width=10cm]{src/img/packageDiagramOverview.pdf}
	\caption{Struttura generale HD-Viz}
\end{figure}

I package principali sono:
\begin{itemize}
	\item \textbf{Util}: all'interno del package sono presenti tutti i tipi di supporto necessari 
	nell'applicazione, come per esempio l'implementazione del design pattern observer, largamente 
	utilizzata all'interno del prodotto;
	\item \textbf{Core}: all'interno del package vi sono tutti i tipi relativi alle strutture dati 
	cardine del prodotto, ossia il \emph{Dataset} e la \emph{DistanceMatrix}, oltre 
	all'implementazione del design pattern strategy per la scelta della distanza da utilizzare;
	\item \textbf{HD-Viz}: contiene i manager, tipi che realizzano la \emph{dependency inversion} e 
	garantiscono il corretto funzionamento dell'applicazione;
	\item \textbf{Graphs}: package che contiene i tipi dei modelli, dei presenter e le view dei 
	grafici di HD-Viz;
	\item \textbf{Components}: contiene i tipi dei presenters e delle views dei diversi componenti, 
	ossia quegli elementi che si interfacciano con il \emph{Dataset}, la \emph{DistanceMatrix} o i 
	modelli dei grafici e ne modificano i dati o le proprietà di visualizzazione.
\end{itemize}

Il tipo \emph{HDVizmanager} è il core dell'architettura, esso implementa la \emph{dependency inversion}, 
infatti si occupa di tener traccia dei dati per evitare istanze multiple del \glossario{Dataset}, 
della \emph{DistanceMatrix} e dei grafici una volta creati. 
Nel momento in cui viene modificato il dataset originale per modificare le visualizzazioni, esso 
funge da tramite creando una copia del dataset e usando quest'ultima per le modifiche, cosicchè il 
dataset originale resti invariato.
\emph{HdVizManager} collabora con diversi altri manager che si occupano di gestire aspetti più 
specifici dell'applicazione, come ad esempio \emph{GraphCreatorManager} che gestisce le factories 
per la creazione dei vari modelli dei grafici. 
Il \emph{CurrentGraphManager} verifica se le factories sono in grado di costruire il grafico, 
quindi si occupa di comunicare con il presenter e il modello del grafico relativo.


\begin{figure}[H]
	\centering
	\includegraphics[width=18cm]{src/img/core-hdvizmanager.jpg}
	\caption{HD-Viz Manager}
\end{figure}

Graphs contiene le diverse visualizzazioni che la nostra applicazione metterà a disposizione, Component mette a disposizione una serie di componenti che permettono di aggiungere funzionalità ai grafici.
Questi ultimi modificano le proprietà di visualizzazione dei grafici, dialogando con modelli tramite interfacce. Ciascun component ha un \emph{Presenter} che gestisce la logica e la \emph{View} che renderizza il tutto,
infatti tramite le interfacce ogni grafico implementerà dimensioni e colori a modo proprio.

\begin{figure}[H]
	\centering
	\includegraphics[width=18cm]{src/img/graph-e-hdviz.jpg}
	\caption{Relazioni tra graph e HD-Viz}
\end{figure}


Ciascun grafico eredita il \emph{GraphModel}; dentro il modello salverà le impostazioni del grafico, ad esempio il gradiente dei colori, il dataset, le label delle righe ed altro.

\begin{figure}[H]
	\centering
	\includegraphics[width=18cm]{src/img/graphs-e-components.jpg}
	\caption{Relazioni tra graphs e components}
\end{figure}


Collegando il model al presenter, i vari componenti che modificano la visualizzazione notificheranno il presenter; a sua volte modificherà il model, il quale si aggiornerà di conseguenza, 
e notificherà al presenter dei cambiamenti.
Ciò avverrà tramite i metodi update. In particolare al presenter verrà passato l'EventType, ovvero il tipo di evento occorso; in base a ciò si aggiornerà il modello, e quindi la visualizzazione.

\subsubsection{Il Server}

Il lato server dell'applicativo è stato realizzato mediante l'uso del framework \emph{Express}.

Questo si occupa della creazione e gestione di un servizio HTTP. 

\emph{Express} fornisce inoltre un particolare modulo, \emph{Router}, con il quale è possibile definire dei percorsi 
al fine di gestire delle richieste indipendentemente dalla implementazione attuale del server, questo rende possibile
accodareli all'applicativo express finale.

\begin{figure}[H]
	\centering
	\includegraphics[width=18cm]{src/img/package-server.png}
	\caption{Diagramma package lato server.}
\end{figure}

\par Presentata la struttura del diagramma di package si riconoscono diversi package, la cui descrizione segue.

NB: Al fine di rendere la compresione più chiara vengono riportati frammenti dei diversi diagrammi delle classi, 
quindi verranno eventualmente omesse delle parti laddove non considerate significative.
\newpage

\begin{itemize}
	
	\item \textbf{Core:} Nucleo dell'applicativo, contiene le classi per la gestione del server
	e la definizione di nuovi percorsi da creare mediante la classe astratta \emph{Route}. 
	
	La classe App definisce il funzionamento del server al quale è possibile accodate le 
	possibili implementazioni di \emph{Route} e, solamente in costruzione, i possibili 
	percorsi statici che devono essere resi accessibili. 
	
	\emph{Application} si occupa inoltre della generazione di un ambiento dotato di sessione, essa 
	si basa sull'uso delle librerie esterne \emph{express-session} e \emph{memorystore}
	che combinate creano uno store virtuale di oggetti memorizzati, di tipo \emph{DataSession}, per ogni 
	utente che accede al servizio.
		
	\begin{figure}[H]
		\centering
		\includegraphics[width=18cm]{src/img/server-core.png}
		\caption{Classi del package di Core}
	\end{figure}

	\newpage
	\item \textbf{Csv:} Package per la gestione di file csv caricati da client. 
	
	Le richieste HTTP vengono gestite dalla \emph{CsvRoute} che realizza la classe astratta \emph{Route}.
	
	L'elaborazione, che ricade nella classe \emph{CsvApplication}, avviene mediante l'uso della liberira \emph{fs} di Node.js per la 
	lettura del file dato in input. La classe crea una rappresentazione temporanea del dataset al fine di spedire al client un risultato 
	già manipolabile, creando quindi un \emph{MDataset}.
	
	\begin{figure}[H]
		\centering
		\includegraphics[width=18cm]{src/img/server-csv.png}
		\caption{Classi del package di Csv}
	\end{figure}

	\newpage
	\item \textbf{Database:} Il package per il riperimento dei dati da un database su un dato server.
	
	Il funzionamento si basa sull'uso di configurazioni che vengono date in costruzione alla classe \emph{RelationalDbApplication} 
	che permettono di eseguire query sui database dalle informazioni date, sfruttando l'ORM Sequelize. 
	
	Dalla query eseguita sul server si costruisce un \emph{MDataset} che viene rispedito al client. 

	Tutte le operazioni, come per il package di \emph{Csv}, vengono generate da richieste HTTP entranti alla \emph{Route} del package, 
	in questo particolare caso \emph{RelationalDbRoute}.

	\begin{figure}[H]
		\centering
		\includegraphics[width=18cm]{src/img/server-database.png}
		\caption{Classi del package di Database}
	\end{figure}


	\newpage
	\item \textbf{Configuration e Group:} Il package di Group contiene delle semplici classi per il controllo di gruppi
	di file con lo stesso nome ma diversa estensione in un determinato percorso, questo si presta al reperimento delle 
	configurazioni per accedere ad un database che sono divise in un file .json e uno .sql, per la query. 

	Il \emph{Configuration Collector} dato un \emph{Group} è in grado di ricostruire le \emph{Configuration} che 
	come già detto, sono una rappresentazione di tutte le informazioni necessarie per accedere ad una determinata fonte di dati.
	
	\begin{figure}[H]
		\centering
		\includegraphics[width=18cm]{src/img/server-group.png}
		\caption{Classi del package di Configuration e Group}
	\end{figure}

	\item \textbf{MDataset:} Package conentente una rappresentazione in versione minimal, priva quindi di funzionalità, del dataset
	che viene calcolato a lato server e spedito al client, in particolare usate dai package \emph{Csv, Database}.

	\item \textbf{Session:} Package che contiene la classe per la gestione e il mantenimento della sessione per ogni utente che 
	accede al servizio di HdViz, esso permette di ottenere informazioni sulla propria \emph{DataSession} o di memorizzare delle 
	modifiche avvenute a lato client sulla sessione.

\end{itemize}
\end{document}
\documentclass[../manuale_sviluppatore.tex]{subfiles}

\begin{document}

\subsection{Pattern architetturale: }

\subsection{Architettura di dettaglio}
\emph{HD Viz manager} delega le visualizzazioni ai vari Manager. 
Graphs contiene le diverse visualizzazioni che la nostra applicazione metterà a disposizione, Component mette a disposizione una serie di componenti che permettono di aggiungere funzionalità ai grafici.
Questi ultimi modificano le proprietà di visualizzazione dei grafici, dialogando con modelli tramite interfacce. Ciascun component ha un \emph{Presenter} che gestisce la logica e la \emph{View} che renderizza il tutto,
infatti tramite le interfacce ogni grafico implementerà dimensioni e colori a modo proprio.
\\
-inserire la foto di components, graphs e hdviz manager-
\\

Ciascun grafico eredita il \emph{GraphModel}; dentro il modello salverà le impostazioni del grafico, ad esempio il gradiente colori, il dataset, i label delle righe ecc.
\\
-mettere tutto graphs-
\\

Collegando il model al presenter, i vari componenti che modificano la visualizzazione notificheranno il model, che si aggiornerà, e dirà al presenter di aggiornarsi perchè la visualizzazione è cambiata. 
Ciò avverrà tramite i metodi update. In particolare verrà aggiornato l'EventType.

\end{document}
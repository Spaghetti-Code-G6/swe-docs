\documentclass[../manuale_sviluppatore.tex]{subfiles}

\begin{document}
Questo capitolo ha lo scopo di indicare agli sviluppatori in che modo opera il codice e la sua sintassi. 
Vengono di seguito esposti gli strumenti utilizzati per effettuare i test di analisi statica e dinamica.

\subsection{Jest}
Il framework di testing Jest viene utilizzato per effettuare i test di unità. La configurazione del framework è presente nella directory principale del progetto con il nome \\
\begin{center}
    \verb|npm install --save-dev @babel/plugin-transform-runtime|
\end{center}

Per eseguire i test sarà sufficiente eseguire il comando \\
\begin{center}
\verb|npm test|
\end{center}

\subsection{ESLint}
È uno strumento di analisi del codice statico per identificare i modelli problematici trovati nel codice JavaScript.
Sarà necessario scaricare la repository da GitHub al seguente link: 
\begin{center} \url{git clone git://github.com/eslint/eslint.git} \end{center}
e successivamente installarne il contenuto
\begin{center}
    \verb|cd eslint| \\
    \verb|npm install|
\end{center}

Per effettuare i test sul codice basterà eseguire il comando
\begin{center} \verb|eslint| \end{center}

\end{document}
\documentclass[../manuale_sviluppatore.tex]{subfiles}
\begin{document}

\subsection*{Clustering}
\phantomsection
\addcontentsline{toc}{subsection}{Clustering}
In data mining e statistiche, l'analisi del clustering gerarchico è un metodo di analisi che cerca 
di costruire una gerarchia di cluster, ovvero una struttura ad albero basata sulla gerarchia.

\subsection*{Components}
\phantomsection
\addcontentsline{toc}{subsection}{Components}
Widget per la modifica delle proprietà del dataset, della matrice delle distanze e delle 
visualizzazioni.

\subsection*{CSS}
\phantomsection
\addcontentsline{toc}{subsection}{CSS}
Il CSS (Cascading Style Sheets), è un linguaggio usato per definire la formattazione di documenti 
HTML, XHTML e XML, ad esempio i siti web e relative pagine web.

\subsection*{CSV}
\phantomsection
\addcontentsline{toc}{subsection}{CSV}
Comma-separated values (abbreviato in CSV) è un formato di file basato su file di testo utilizzato 
per l'importazione ed esportazione di una tabella di dati.

\subsection*{D3.js}
\phantomsection
\addcontentsline{toc}{subsection}{D3.js}
Libreria JavaScript per creare visualizzazioni dinamiche ed interattive partendo da dati 
organizzati, visibili attraverso un comune browser.

\subsection*{Dataset}
\phantomsection
\addcontentsline{toc}{subsection}{Dataset}
È una collezione di dati. Più comunemente un dataset costituisce un insieme di dati strutturati in 
forma relazionale, cioè corrisponde al contenuto di una singola tabella di base di dati.

\subsection*{ESLint}
\phantomsection
\addcontentsline{toc}{subsection}{ESLint}
Strumento di analisi statica per il codice JavaScript.

\subsection*{Express}
\phantomsection
\addcontentsline{toc}{subsection}{Express}
Framework per applicazioni web Node.js flessibile e leggero che fornisce una serie di funzioni 
avanzate per le applicazioni web e per dispositivi mobili.

\subsection*{HTML}
\phantomsection
\addcontentsline{toc}{subsection}{HTML}
HyperText Markup Language; è un linguaggio di markup nato per la formattazione e impaginazione di 
documenti ipertestuali.

\subsection*{JavaScript}
\phantomsection
\addcontentsline{toc}{subsection}{JavaScript}
Linguaggio di programmazione orientato agli oggetti e agli eventi, comunemente utilizzato nella 
programmazione Web lato client (esteso poi anche al lato server) per la creazione, in siti web e 
applicazioni web, di effetti dinamici interattivi.

\subsection*{Jest}
\phantomsection
\addcontentsline{toc}{subsection}{Jest}
Framework di test open-source sviluppato da Facebook.

\subsection*{JQuey}
\phantomsection
\addcontentsline{toc}{subsection}{JQuery}
Libreria di JavaScript per applicazioni web.

\subsection*{JSON}
\phantomsection
\addcontentsline{toc}{subsection}{JSON}
Formato usato per lo scambio di dati fra client e server.

\subsection*{MVP}
\phantomsection
\addcontentsline{toc}{subsection}{MVP}
Model View Presenter, pattern architetturale derivato dal Model View Controller (MVC), viene 
utilizzato prevalentemente per costruire interfacce utente.

\subsection*{Node.js}
\phantomsection
\addcontentsline{toc}{subsection}{Node.js}
È una runtime di JavaScript open-source multipiattaforma orientata agli eventi per l'esecuzione di 
codice JavaScript.

\subsection*{Open source}
\phantomsection
\addcontentsline{toc}{subsection}{Open Source}
Termine impiegato in ambito informatico per riferirsi ad un software in cui gli attori rendono 
pubblico il codice sorgente. Tutto questo viene regolamentato tramite l'applicazione delle licenze 
d'uso. I software open-source permettono ai programmatori distanti di coordinarsi e di lavorare 
tutti allo stesso progetto.

\subsection*{Selenium}
\phantomsection
\addcontentsline{toc}{subsection}{Selenium}
Tool open source per l'automazione di web testing.

\subsection*{Sequelize}
\phantomsection
\addcontentsline{toc}{subsection}{Sequelize}
Mappa oggetti a database relazionali attraverso lo standard SQL.

\end{document}

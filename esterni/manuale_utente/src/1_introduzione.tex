\documentclass[../manuale_utente.tex]{subfiles}

\begin{document}

\subsection{Scopo del documento}
\label{sub:scopo_doc}
Lo scopo di questo documento è illustrare tutte le funzionalità del prodotto HD-Viz. L’utente finale in questo modo avrà a disposizione tutte le indicazioni per il corretto uso del software. 

\subsection{Scopo del prodotto}
\label{sub:scopo_prod}
Il capitolato \emph{ C4 HD Viz} richiede lo sviluppo di una web application che abbia come scopo la traduzione di dati con molte dimensioni in grafici che aiutino l’utente a trarre delle interpretazioni e conclusioni sugli stessi. \\
Il gruppo SpaghettiCode si propone di sviluppare per l’azienda \emph{Zucchetti S.p.A.}  il prodotto richiesto.

\subsection{Glossario}
\label{sub:glossario}
All’interno del documento sono presenti termini che possono presentare significati ambigui o incongruenti a seconda del contesto. 
Al fine quindi di evitare l’insorgere di incomprensioni viene fornito un glossario individuabile nell’appendice §A, posta alla fine di questo documento, contenente i suddetti termini e la loro spiegazione. 
Nella seguente documentazione per favorire maggiore chiarezza ed evitare inutili ridondanze tali parole vengono indicate mettendo una "G" a pedice di ogni prima occorrenza del termine che si incontri ad ogni inizio di sezione. 

\end{document}
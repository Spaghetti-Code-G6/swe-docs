\documentclass[../manuale_utente.tex]{subfiles}
\begin{document}


\subsection{Creazione workflow}
    \label{sub:crea_work}
\subsubsection{Caricamento dati}
    \label{subsub:carica_dati}

Il primo passo per poter utilizzare HD-Viz è importare dei dati. È possibile farlo in due modi:
\begin{itemize}
    \item caricamento di dati tramite file \glossario{CSV} da locale
    \item caricamento di dati tramite query da un database esterno
\end{itemize}
In seguito l’utente potrà esplorare e manipolare i grafici al fine di analizzare i dati, trarne conclusioni e similitudini tra essi. \\
Per importare un file da locale è sufficiente andare nella sezione \emph{“Carica file”}  in alto a sinistra, cliccare su \emph{“scegli file”}, 
scegliere un file CSV presente sul proprio dispositivo e poi cliccare su \emph{“Carica”}. 
Per importare dati tramite database bisogna cliccare sul bottone \emph{“Carica”} presente nel riquadro in alto a sinistra, relativo al caricamento database.
In questo modo il database verrà caricato tramite un file di configurazione presente nella cartella  \verb|server/src/dbConfig|

\begin{figure}[H]
	\centering
	\includegraphics[width=18cm]{src/img/introduzione.jpg}
	\caption{Introduzione a Hd-Viz}
\end{figure}

%\begin{figure}[H]
%	\centering
%	\includegraphics[width=18cm]{src/img/seleziona_dataset.jpg}
%	\caption{Seleziona fonte}
%\end{figure}


\subsubsection{Visualizzazione grafico}
    \label{subsub:vis_graf}
Una volta che il file è stato caricato, il grafico principale, lo \glossario{ScatterPlot Matrix}, verrà definito con i dati appena passati. \\
Grazie al menu a tendina in alto a sinistra sarà possibile cambiare grafico passando per esempio al Force Field, all’Heat Map e alla Distance Map.

\begin{figure}
	\centering
	\begin{minipage}{.5\textwidth}
	  \centering
	  \includegraphics[width=.5\linewidth]{img/seleziona_dataset.jpg}
	  \caption{Seleziona fonte}
	  \label{fig:sub1}
	\end{minipage}%
	\begin{minipage}{.5\textwidth}
	  \centering
	  \includegraphics[width=.5\linewidth]{img/seleziona_grafico.jpg}
	  \caption{Seleziona grafico}
	  \label{fig:sub2}
	\end{minipage}
	\caption{Selezione fonte e grafico}
	\label{fig:test}
\end{figure}

\newpage

\paragraph{ScatterPlot Matrix}
    \label{par:vis_scatt}
Si tratta di un grafico che permette di visualizzare la relazione tra due variabili quantitative riportate su uno spazio cartesiano. Ogni unità statistica è rappresentata da un punto posizionato sul grafico in base alle sue coordinate. 
Quindi questo grafico sarà costituito da tanti punti quante sono le unità statistiche oggetto di studio. I valori che assume l’unità statistica per le due variabili rappresentano quindi la posizione dell’unità rispetto agli assi. 
Osservando l’andamento dei punti si può notare come sembra esserci una relazione lineare positiva o negativa. Se il modello di punti sul grafico scende dall'alto a sinistra verso il basso a destra, suggerisce una correlazione negativa. 
Può essere disegnata una linea di andamento (o linea di trend) per studiare la correlazione tra le variabili in esame. Se non c’è relazione tra le due variabili all’aumentare dei valori di una variabile, i valori dell’altra variabile non risulteranno in media né aumentare né diminuire.\\
In questo grafico sarà possibile modificare la dimensione del dataset, la dimensione della matrice, modificarne la dimensione rappresentata mediante tinta, mediante brillanza. Tutti questi parametri si troveranno
in ordine in un menu a sinistra. Ad ogni manipolazione dei parametri inseriti si otterrà subito una modifica del grafico. \\
Qui di seguito per esempio ecco come si mostra uno ScatterPlot Matrix al caricamento dei dati dell'Iris.

\begin{figure}[H]
	\centering
	\includegraphics[width=18cm]{src/img/spm/spm_iris.jpg}
	\caption{ScatterPlot Matrix iris dataset}
\end{figure}

Quando si va a dare colore alla lunghezza dei sepali, il risultato è il seguente:

\begin{figure}[H]
	\centering
	\includegraphics[width=18cm]{src/img/spm/spm_colore_dimensione_sepal.jpg}
	\caption{ScatterPlot Matrix lunghezza del sepalo colorato}
\end{figure}

È possibile inoltre cambiare i colori rappresentati. Nel seguente esempio si è scelto di colorare la dimensione del sepalo, e si è usato come gradiete il CoolWarm.

\begin{figure}[H]
	\centering
	\includegraphics[width=18cm]{src/img/spm/spm_mix_colori.jpg}
	\caption{ScatterPlot Matrix modifica di più colori}
\end{figure}


Qui invece vi è un esempio con un \glossario{dataset} da 1000 records. È possibile cambiare le dimensioni rappresentate escludendone una ed aggiungendone un'altra, come rappresentato in figura.

\begin{figure}[H]
	\centering
	\includegraphics[width=18cm]{src/img/spm/spm_cambio_dim.jpg}
	\caption{ScatterPlot Matrix modifica delle dimensioni rappresentate}
\end{figure}


\paragraph{Distance Map}
    \label{par:vis_distance}
Si occupa di misurare la prossimità fra due unità statistiche adottando diversi algoritmi.\\
Dopo aver selezionato il grafico DistanceMap in alto a sinistra, i vari algoritmi compariranno nel menu sottostante. Qui ad esempio si mostra come appare il dataset dell'Iris
dopo aver selezionato il grafico DistanceMap:

\begin{figure}[H]
	\centering
	\includegraphics[width=18cm]{img/dm/iris_base_dm.jpg}
	\caption{DistanceMap applicata all'Iris}
\end{figure}

Comer per ScatterPlot, è possibile manipolare il dataset tramite le impostazioni presenti a sinistra, nonchè manipolare più impostazioni per ottenere risultati differenti.\\
Di seguito per esempio si mostra come appare il dataset dell'Iris dopo aver selezionato come colore la scala di grigi, e l'ordinamento "Hierarichal Clustering".


\begin{figure}[H]
	\centering
	\includegraphics[width=18cm]{img/dm/iris_clustering_euclideo_dm.jpg}
	\caption{DistanceMap applicata all'Iris}
\end{figure}


\paragraph{Force Field}
    \label{par:vis_force_field}
Il grafico \glossario{Force Field} traduce le distanze nello spazio a molte dimensioni in forze di attrazione e repulsione tra i punti proiettati nello spazio. 
Questo grafico esegue una riduzione dimensionale preservando, o addirittura evidenziando, le strutture presenti nei dati. Più semplicemente, le entità vengono
rappersentate tramite nodi, e le linee che le uniscono rappresentano le connessioni, o forze di attrazione. Al caricamento del grafico i punti si muovono in maniera
casuale in cui i punti si spostano in base alle forze a cui sono soggetti. 

\begin{figure}[H]
	\centering
	\includegraphics[width=18cm]{img/ff/ff_iris_0_1}
	\caption{ForceField applicata all'Iris - caricamento}
\end{figure}

Dopo poco si raggiunge una condizione di equilibrio stabile dove si evidenzia la geometria. 

\begin{figure}[H]
	\centering
	\includegraphics[width=18cm]{img/ff/ff_iris_0_2}
	\caption{ForceField applicata all'Iris - equilibrio}
\end{figure}


Come per gli altri grafici sarà possibile manipolare il dataset ottenendo risultati differenti, come di seguito. 

\begin{figure}[H]
	\centering
	\includegraphics[width=18cm]{img/ff/ff_iris_1}
	\caption{ForceField applicata all'Iris - modifica dataset}
\end{figure}


Il ForceField è caratterizzato dalla possibilità di modificare le forze di attrazione, e il rispettivo range. Ad ogni manipolazione si possono ottenere quindi risultati differenti, 
come rappresentato di seguito:

\begin{figure}[H]
	\centering
	\includegraphics[width=18cm]{img/ff/ff_iris_2}
	\caption{ForceField applicata all'Iris - modifica force intensity}
\end{figure}

\begin{figure}[H]
	\centering
	\includegraphics[width=18cm]{img/ff/ff_iris_3}
	\caption{ForceField applicata all'Iris - modifica force range}
\end{figure}

\begin{figure}[H]
	\centering
	\includegraphics[width=18cm]{img/ff/ff_iris_4}
	\caption{ForceField applicata all'Iris - modifica force range e force intensity}
\end{figure}

È anche possibile selezionare un punto e allontanarlo dal centro:

\begin{figure}[H]
	\centering
	\includegraphics[width=18cm]{img/ff/ff_iris_5}
	\caption{ForceField applicata all'Iris - manipolazione dei punti}
\end{figure}

\paragraph{HeatMap}
    \label{par:vis_heatmap}

Il grafico \glossario{Heat map} trasforma la distanza tra i punti i colori più o meno intensi, facendo così capire quali oggetti sono vicini tra loro e quali sono distanti. \\
Come per gli altri grafici è possibile usufruire delle impostazioni nel menu a sinistra per poter modificare il grafico. Qui di seguito per esempio si mostra la differenza 
tra la Heat Map base, e la Heat Map con colori in scala di grigi:

\begin{figure}[H]
	\centering
	\includegraphics[width=18cm]{img/hm/heat_map_base.jpg}
	\caption{Heat Map base}
\end{figure}

\begin{figure}[H]
	\centering
	\includegraphics[width=18cm]{img/hm/heat_map_base_gray.jpg}
	\caption{Heat Map in scala di grigi}
\end{figure}




\end{document}
\documentclass[../manuale_utente.tex]{subfiles}

\begin{document}

\subsection{Requisiti}
    \label{sec:requisiti}
\subsubsection{Requisiti Hardware consigliati}
    \label{subsub:req_h}

\paragraph{Windows}
    \label{par:Windows_req}

\begin{itemize}
    \item Sistema Operativo: Windows 7 o versioni successive
    \item Processore: Intel o AMD (almeno 4 core?)
    \item RAM: almeno 2 GB di RAM*
    \item Google Chrome v.87.0.0 o superiore, oppure Mozzilla Firefox v.85.0.0 o superiore, oppure Opera v.74.0.0 o superiore, oppure Safari v.14.0.0 o superiore;
    \item Altri requisiti: connessione internet.
\end{itemize}
*In base alla grandezza del dataset, può essere richiesto un quantitativo di RAM superiore.

\paragraph{MacOs}
    \label{par:mac_req}

\begin{itemize}
    \item Sistema Operativo: MacOS Sierra o versioni successive;
    \item Processore: Intel o AMD
    \item RAM: almeno 2 GB di RAM*
    \item Google Chrome v.87.0.0 o superiore, oppure Mozzilla Firefox v.85.0.0 o superiore, oppure Opera v.74.0.0 o superiore, oppure Safari v.14.0.0 o superiore;
    \item Altri requisiti: connessione internet.
\end{itemize}
*In base alla grandezza del dataset, può essere richiesto un quantitativo di RAM superiore.

\subsubsection{Requisiti Software}
    \label{subsub:req_s}
\paragraph{Node.js}
    \label{par:node}
Node.js è una runtime di JavaScript open-source multipiattaforma, event-driven, per l’esecuzione di codice JavaScript. Molti dei suoi moduli base sono scritti in JavaScript. 
\emph{Versione utilizzata: 14.5}.\\
Link per il download: \url{https://nodejs.org/it/download/}

\subsection{Installazione}
    \label{sub:inst}


Per eseguire l’applicazione HD-Viz bisogna seguire i seguenti passaggi:
\begin{itemize}
    \item Installare Node.js versione 14.5;
    \item Scaricare l’applicazione, in formato zip, dalla repository del progetto: \url{https://github.com/Spaghetti-Code-G6/HD-Viz}
    \item Estrarre il contenuto del file in una qualsiasi directory; 
    \item Aprire il prompt dei comandi (Windows) o il terminale (Linux/Mac); 
    \item Posizionarsi tramite il comando cd /percorso nella cartella in cui sono stati estratti i file;
    \item Successivamente sarà necessario eseguire i seguenti comandi:
    \begin{center}
        \verb|npm i| \\ 
        \verb|npx webpack ./server/modules/application.js --mode development| \\ 
        \verb|node ./server/server.js|
    \end{center}
    \item Infine si deve aprire su browser il percorso visibile sul terminale (Es. \url{http://localhost:8085/public/index.html}).
\end{itemize}


\subsection{Segnalazione bug}
Nel caso si riscontrassero bug si prega di segnalarlo al seguente indirizzo:\\
\begin{center}\href{mailto:spaghetti.code.g6@gmail.com}{spaghetti.code.g6@gmail.com}\end{center}


\end{document}
\usepackage[italian]{babel}
\usepackage[margin=20mm, footskip=20pt]{geometry}
\usepackage{graphicx}
\usepackage{subfiles}
\usepackage{hyperref}
\usepackage{nameref}
\usepackage{titlesec}
\usepackage{longtable}
\usepackage[table]{xcolor}
\usepackage{titling}
\usepackage{lastpage}
\usepackage{ifthen}
\usepackage{calc}
\usepackage{soulutf8}
\usepackage{contour}
\usepackage{float}
\usepackage{fancyhdr}
\usepackage{multirow}
\usepackage{pgfgantt}
\usepackage{lscape}
\usepackage{background}
\usepackage{lmodern}
\usepackage{textcomp}
\usepackage{lastpage}
\usepackage[utf8]{inputenc}
\usepackage{makecell}

% definizione dei percorsi in cui cercare immagini
\graphicspath{ {./}
    {./src/img/}
}

% setup della sottolineatura
\setuldepth{Flat}
\contourlength{0.8pt}

\newcommand{\uline}[1]{%
  \ul{{\phantom{#1}}}%
  \llap{\contour{white}{#1}}%
}


% setup dei link
\hypersetup{
  % set true if you want colored links (instead of boxes)
  colorlinks=true,
  % set to all if you want both sections and subsections linked
  linktoc=all,
  % set color for file links
  filecolor=blue,
  % set color for internal links
  linkcolor=black,
  % set url color
  urlcolor=blue,
  % set characters encoding in the bookmarks tab
  pdfencoding=unicode,
}

% setup forma \paragraph e \subparagraph
\titleformat{\paragraph}[hang]{\normalfont\normalsize\bfseries}{\theparagraph}{1em}{}
\titleformat{\subparagraph}[hang]{\normalfont\normalsize\bfseries}{\thesubparagraph}{1em}{}

% setup profondità indice di default
\setcounter{secnumdepth}{5}
\setcounter{tocdepth}{5}

\makeatletter %% non togliere, i comandi che definiscono i placeholder vanno qui
% esempio di utilizzo: \appendToGraphicspath{./img/} (un comando diverso per ogni path da includere)
% N.B.: ci DEVE essere un forward slash alla fine del path, a indicare che è una cartella.
\newcommand\appendToGraphicspath[1]{%
  \g@addto@macro\Ginput@path{{#1}}%
}

\newcommand{\setTitle}[1]{%
  \newcommand{\@placeholderTitle}{#1}%
}
\newcommand{\placeholderTitle}{\@placeholderTitle}

\newcommand{\setUso}[1]{%
  \newcommand{\@uso}{#1}%
}
\newcommand{\uso}{\@uso}

\newcommand{\setVersione}[1]{%
  \newcommand{\@versione}{#1}%
}
\newcommand{\versione}{\@versione}

\newcommand{\disabilitaVersione}{%
  \renewcommand{\setVersione}[1]{}%
  \renewcommand{\versione}{DISABILITATA}
}

\newcommand{\setResponsabile}[1]{%
  \newcommand{\@responsabile}{#1}%
}
\newcommand{\responsabile}{\@responsabile}

\newcommand{\setRedattori}[1]{%
  \newcommand{\@redattori}{#1}%
}
\newcommand{\redattori}{\@redattori}

\newcommand{\setVerificatori}[1]{%
  \newcommand{\@verificatori}{#1}%
}
\newcommand{\verificatori}{\@verificatori}

\newcommand{\setDestinatari}[1]{%
  \newcommand{\@destinatari}{#1}%
}
\newcommand{\destinatari}{\@destinatari}

\newcommand{\setDescrizione}[1]{%
  \newcommand{\@descrizione}{#1}%
}
\newcommand{\descrizione}{\@descrizione}

\newcommand{\setModifiche}[1]{%
  \newcommand{\@modifiche}{#1}%
}

\newcommand{\modifiche}{\@modifiche}
\makeatother %% non togliere, i comandi che definiscono i placeholder vanno qui

% hook per lo script che genera il glossario
\newcommand{\glossario}[1]{#1\textsubscript{G}}

% comandi per rendere opzionali gli elenchi di figure
\newcommand{\elencoFigure}{%
  \renewcommand{\listfigurename}{Elenco delle figure}%
  \listoffigures%
}

\newcommand{\disabilitaElencoFigure}{%
  \renewcommand{\elencoFigure}{}%
}

% comandi per rendere opzionali le tabelle
\newcommand{\elencoTabelle}{%
  \renewcommand{\listtablename}{Elenco delle tabelle}%
  \listoftables%
}

\newcommand{\disabilitaElencoTabelle}{%
  \renewcommand{\elencoTabelle}{}%
}

% comando per riferirsi ad una sezione
\newcommand{\refSec}[1]{%
  \hyperref[#1]{\S\ref*{#1}}%
}

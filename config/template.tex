\pagenumbering{gobble}


% variabile prima pagina (serve nell'istruzione dopo per stampare il bg solo nella prima pagina)
\newif\iffirstpage
\firstpagetrue

% setta l'immagina di bg
\backgroundsetup{
	scale=1,
	opacity=0.2,
	placement=top,
	contents={%
			\iffirstpage
				\includegraphics[width=\paperwidth]{datascience_og_colori.png}%
				\global\firstpagefalse
			\fi
		}%
}

\begin{titlepage}% per non stampare il numero della pagina

	\centering % allinea al centro la pagina
	\hspace{0.05\textwidth}% spazio tra linea e testo
	% lasciare questa riga per il corretto funzionamento di \parbox
	\parbox[b]{0.4\textwidth}{% cambiando la larghezza del testo il paragrafo si muove a destra o a sinistra 
	{\hspace{0.05\textwidth}\includegraphics[width=4cm,height=4cm]{logo_colori.png}}\\[3\baselineskip] % logo
	{\Huge\bfseries SpaghettiCode}\\ [\baselineskip] %titolo
	{\texttt{spaghetti.code.g6@gmail.com}}\\[\baselineskip]\\[4\baselineskip] % 
	{\Large\textsc\mbox{\placeholderTitle{}}}\\[4\baselineskip] % nome del documento
	{\begin{tabular}{r|l}
		\hline                                  \\
		% testo in grassetto
		\textbf{Versione}     & \versione{}     \\
		\rule{0pt}{3ex}%  EXTRA vertical height 
		\textbf{Approvazione} & \responsabile{} \\
		\rule{0pt}{3ex}%  EXTRA vertical height 
		\textbf{Redazione}    & \redattori{}    \\
		\rule{0pt}{3ex}%  EXTRA vertical height 
		\textbf{Verifica}     & \verificatori{} \\
		\rule{0pt}{3ex}%  EXTRA vertical height 
		\textbf{Uso}          & \uso{}          \\
		\rule{0pt}{3ex}%  EXTRA vertical height 
		\textbf{Destinato a}  & \destinatari{}
		\ifthenelse{\equal{\uso}{Esterno}}{
		\\ & Zucchetti S.p.A.
		}{}
	\end{tabular}}\\[4\baselineskip]

	}

	{\bfseries Descrizione}\\
	{\descrizione{}}\\[1\baselineskip]



\end{titlepage}

\newgeometry{textheight=660pt, lmargin=2cm, tmargin=2cm, rmargin=2cm}

% setup di header e footer nelle pagine senza numero
\fancypagestyle{nopage}{%
	\fancyhf{}%
	\fancyhead[R]{\includegraphics[width=1.3cm]{logo_colori.png}}%
	\fancyhead[L]{\emph{SpaghettiCode}\\\placeholderTitle{}}%
}
% setup di header e footer nelle pagine col numero
\fancypagestyle{usual}{%
	\fancyhf{}%
	\fancyhead[R]{\includegraphics[width=1.3cm]{logo_colori.png}}%
	\fancyhead[L]{\emph{SpaghettiCode}\\\placeholderTitle{}}%
	\fancyfoot[R]{\thepage\ di~\pageref{LastPage}}%
}
\setlength{\headheight}{1.8cm}

\newpage
\pagestyle{nopage}

\setcounter{table}{-1}

%REGISTRO DELLE MODIFICHE

\section*{Registro delle modifiche}
\label{sec:registro_delle_modifiche}

\rowcolors{2}{white!80!lightgray!90}{white}
\renewcommand{\arraystretch}{2} % allarga le righe con dello spazio sotto e sopra

\begin{longtable}
	[H]{|>{\centering\bfseries}m{2cm}|>{\centering}m{3.5cm}|>{\centering}m{2.5cm}|>{\centering}m{3cm}|>{\centering\arraybackslash}m{5cm}|}
	
	\hline
	\rowcolor{lightgray}
	{\textbf{Versione}} & {\textbf{Nominativo}} & {\textbf{Ruolo}} & {\textbf{Data}} & {\textbf{Descrizione}} \\
	\hline
	\endfirsthead
	
	\hline
	\rowcolor{lightgray}
	{\textbf{Versione}} & {\textbf{Nominativo}} & {\textbf{Ruolo}} & {\textbf{Data}} & {\textbf{Descrizione}} \\
	\hline
	\endhead
	
	\hline
 	\multicolumn{5}{|c|}{\emph{Continua alla pagina successiva......}}\\
	
	\endfoot
	\hline
	\endlastfoot
	
	\modifiche{}

\end{longtable}
% section registro_delle_modifiche (end)

\newpage
\thispagestyle{nopage}
\pagenumbering{roman}
\tableofcontents

\elencoFigure{}%
\elencoTabelle{}%

\newpage
\pagestyle{usual}
\pagenumbering{arabic}
